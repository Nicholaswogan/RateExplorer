\documentclass[12pt,landscape]{article}
\usepackage{geometry}                % See geometry.pdf to learn the layout options. There are lots.
\geometry{letterpaper}                   % ... or a4paper or a5paper or ... 
%\geometry{landscape}                % Activate for for rotated page geometry
%\usepackage[parfill]{parskip}    % Activate to begin paragraphs with an empty line rather than an indent
\usepackage{graphicx}
\usepackage{amssymb}
\usepackage{longtable}
\usepackage{natbib}
%\usepackage{booktabs}
\usepackage{epstopdf}
\DeclareGraphicsRule{.tif}{png}{.png}{`convert #1 `dirname #1`/`basename #1 .tif`.png}
\newcounter{reaction}
\newcounter{photo}

\begin{document}

%\landscape
%\end{document}
\setlongtables % keeps the width uniform across both pages
% \footnotesize{
\begin{longtable}{l lcl l p{3.5cm} } 
%\caption{All Reactions}\\
 & {\large\bf Table 2.}  & & {\large\bf Reactions} & & \\
\hline
 & {\large\strut Reactants$^a$}  &  & {\large Products} & {\large Rate$^b$} & {\large Reference} \\
\hline \hline 
\endfirsthead
\hline
 & {\large\strut Reactants$^a$}  &  & {\large Products} & {\large Rate$^b$} & {\large Reference} \\
\hline %\hline 
\endhead 
\multicolumn{6}{l}{\bf H, O}\\
\refstepcounter{reaction}\label{R1} R\arabic{reaction}  & H            + H            + M & $\!\!\!\rightarrow$ &  H$_2$        + M &$  8.8\!\times\! 10^{-33} \left(T/298 \right)^{-0.6}$ & Ba92\\
        & H            + H           &$\!\!\!\rightarrow$&  H$_2$         &$  1.0\!\times\! 10^{-12}$ & note  \\
\refstepcounter{reaction}\label{R2} R\arabic{reaction}   & O            + H            + M & $\!\!\!\rightarrow$ &  OH           + M &$  4.3\!\times\! 10^{-32} \left(T/298 \right)^{-1.0}$ & Ts86\\
         & O            + H           &$\!\!\!\rightarrow$&  OH            &$  1.0\!\times\! 10^{-12}$ & note  \\
\refstepcounter{reaction}R\arabic{reaction}   & H$_2$        + O           & $\!\!\!\rightarrow$ &  OH           + H      & $  3.5\!\times\! 10^{-13} \left(T/298\right)^{ 2.67}e^{ -3160/T}$ & Ba92\\
\refstepcounter{reaction}R\arabic{reaction}   & H            + OH           + M & $\!\!\!\rightarrow$ &  H$_2$O       + M &$  6.6\!\times\! 10^{-32} \left(T/298 \right)^{-2.1}$ & Ja03\\
             & H            + OH           & $\!\!\!\rightarrow$ &  H$_2$O       &$  2.7\!\times\! 10^{-10} e^{   -75/T}$ & Co85\\
 \refstepcounter{reaction}R\arabic{reaction}   & H$_2$        + OH          & $\!\!\!\rightarrow$ &  H$_2$O       + H       & $  1.6\!\times\! 10^{-12} \left(T/298\right)^{ 1.6}e^{ -1660/T}$ & Ba92\\
 \refstepcounter{reaction}R\arabic{reaction}   & OH           + OH          & $\!\!\!\rightarrow$ &  H$_2$O       + O   & $  1.7\!\times\! 10^{-12} \left(T/298\right)^{ 1.14}e^{   -50/T}$ & Ba92, Li91\\
 \refstepcounter{reaction}R\arabic{reaction}   & OH           + OH          & $\!\!\!\rightarrow$ &  O$_2$        + H$_2$   & $  3.3\!\times\! 10^{-12} \left(T/298\right)^{ 0.51}e^{-25400/T}$ & Ka05\\
 \refstepcounter{reaction}R\arabic{reaction}   & O            + O            + M & $\!\!\!\rightarrow$ &  O$_2$        + M &$  5.2\!\times\! 10^{-35} e^{+900/T}$ & Ts86\\
           & O            + O           &$\!\!\!\rightarrow$&  O$_2$        &$  1.0\!\times\! 10^{-12}$ & note  \\
 \refstepcounter{reaction}R\arabic{reaction}  & O            + OH          &$\!\!\!\rightarrow$ &  O$_2$        + H                                       & $  2.4\!\times\! 10^{-11} e^{  -353/T}$ & Ba92\\


 \multicolumn{6}{l}{\bf HO$_2$}\\
  \refstepcounter{reaction}R\arabic{reaction} &  H  +     O$_2$ + M &$\!\!\!\rightarrow$ &      HO$_2$ + M & $ 3.0\!\times\! 10^{-32} \left(T/298 \right)^{-1.23}e^{+100/T} $   &  Ya21 \\     
          & H  +     O$_2$ &$\!\!\!\rightarrow$ &    HO$_2$  & $ 5.3\!\times\! 10^{-11} \left(T/298 \right)^{0.60}e^{+125/T}  $    & Ya21 \\  
 \refstepcounter{reaction}R\arabic{reaction} & H  +   HO$_2$  &$\!\!\!\rightarrow$ &   H$_2$  +   O$_2$   & $ 7.1\!\times\! 10^{-11}e^{-710/T}  $  & Ba92 \\  
 \refstepcounter{reaction}R\arabic{reaction} & H  +   HO$_2$  &$\!\!\!\rightarrow$ &   H$_2$O  +   O   & $ 5.0\!\times\! 10^{-11}e^{-870/T}  $  & Ba92 \\  
 \refstepcounter{reaction}R\arabic{reaction} & H  +   HO$_2$  &$\!\!\!\rightarrow$ &   OH  +   OH   & $ 7.6\!\times\! 10^{-11}\left(T/298 \right)^{0.2}  $  & Ya21 \\  
 \refstepcounter{reaction}R\arabic{reaction} & OH  + HO$_2$   &$\!\!\!\rightarrow$ &   H$_2$O  +   O$_2$   & $ 4.8\!\times\! 10^{-11} e^{+250/T} $  & At04 \\  
 \refstepcounter{reaction}R\arabic{reaction} & O  + HO$_2$   &$\!\!\!\rightarrow$ &  OH  +   O$_2$   & $ 2.7\!\times\! 10^{-11} e^{+240/T} $  & At04 \\  

 \multicolumn{6}{l}{\bf O$_3$}\\
  \refstepcounter{reaction}R\arabic{reaction} &  O  +     O$_2$ + M &$\!\!\!\rightarrow$ &   O$_3$ + M & $ 6.0\!\times\! 10^{-34} \left(T/298 \right)^{-2.3}  $   &  De97 \\     
          & O  +     O$_2$ &$\!\!\!\rightarrow$ &   O$_3$  & $ 2.8\!\times\! 10^{-11}  $    &  De97 \\  
 \refstepcounter{reaction}R\arabic{reaction} & O  +  O$_3$  &$\!\!\!\rightarrow$ &  O$_2$   +  O$_2$   & $ 8.0\!\times\! 10^{-12} e^{-2060/T} $  & At04 \\  
 \refstepcounter{reaction}R\arabic{reaction} & H  +  O$_3$  &$\!\!\!\rightarrow$ &  OH   +  O$_2$   & $ 1.4\!\times\! 10^{-10} e^{-470/T} $  & De97 \\  
 \refstepcounter{reaction}R\arabic{reaction} & OH  +  O$_3$  &$\!\!\!\rightarrow$ &  HO$_2$   +  O$_2$   & $ 1.6\!\times\! 10^{-12} e^{-940/T} $  & De97 \\  
% \refstepcounter{reaction}R\arabic{reaction} & O$_3$  +  O($^1$D)  &$\!\!\!\rightarrow$ &  O$_2$   +  O + O & $ 1.2\!\times\! 10^{-10} $   & De97 \\  

 \multicolumn{6}{l}{\bf H$_2$O$_2$}\\
  \refstepcounter{reaction}R\arabic{reaction} &  OH  +     OH + M &$\!\!\!\rightarrow$ &      H$_2$O$_2$ + M & $ 6.9\!\times\! 10^{-31} \left(T/298 \right)^{-0.8}  $   &  At97 \\     
          & OH  +     OH &$\!\!\!\rightarrow$ &   H$_2$O$_2$  & $ 2.6\!\times\! 10^{-11}  $    &  At04 \\  
  \refstepcounter{reaction}R\arabic{reaction} &  HO$_2$  + HO$_2$ + M &$\!\!\!\rightarrow$ &      H$_2$O$_2$ + O$_2$ +  M & $ 2.1\!\times\! 10^{-33} e^{+920/T}  $   & At97  \\     
          & HO$_2$  + HO$_2$ &$\!\!\!\rightarrow$ &   H$_2$O$_2$ +   O$_2$  & $ 3.0\!\times\! 10^{-13} e^{+460/T} $    & JPL19 \\  
 \refstepcounter{reaction}R\arabic{reaction} & H$_2$O$_2$  + H &$\!\!\!\rightarrow$ &  H$_2$ + HO$_2$   & $ 2.8\!\times\! 10^{-12} e^{-1890/T} $  & Ba92 \\  
 \refstepcounter{reaction}R\arabic{reaction} & H$_2$O$_2$  + H &$\!\!\!\rightarrow$ &  H$_2$O + OH   & $ 1.7\!\times\! 10^{-11} e^{-1890/T} $  & Ba92 \\  
 \refstepcounter{reaction}R\arabic{reaction} & H$_2$O$_2$  + OH &$\!\!\!\rightarrow$ &  H$_2$O + HO$_2$   & $ 2.9\!\times\! 10^{-12} e^{-160/T} $  &  At04 \\  
 \refstepcounter{reaction}R\arabic{reaction} & H$_2$O$_2$  + O &$\!\!\!\rightarrow$ &  OH  + HO$_2$   & $ 1.4\!\times\! 10^{-12} e^{-2000/T} $  & At04 \\  

\multicolumn{6}{l}{\bf CO, CO$_2$}\\
 \refstepcounter{reaction}\label{RCO} R\arabic{reaction}   & C            + O      +M     &$\!\!\!\rightarrow$&  CO           + M           &$  5.8\!\times\! 10^{-29}\left(T/298 \right)^{-3.34}$   &  rev-Ba92 \\
           & C            + O           &$\!\!\!\rightarrow$&  CO            &$  1.0\!\times\! 10^{-12}$ &  note \\
 \refstepcounter{reaction}R\arabic{reaction}  & C            + O$_2$       &$\!\!\!\rightarrow$ &  CO           + O       & $  1.6\!\times\! 10^{-11}$ & Ba92\\
 \refstepcounter{reaction}R\arabic{reaction}   & CO           + O            + M & $\!\!\!\rightarrow$ &  CO$_2$       + M &$  1.7\!\times\! 10^{-33} e^{ -1510/T}$ & Ts86\\
           & CO           + O             & $\!\!\!\rightarrow$ &  CO$_2$        &$  1.0\!\times\! 10^{-14} e^{ -1630/T}$ & To84\\  %Toby et al 1984
 \refstepcounter{reaction}R\arabic{reaction}   & CO           + OH          & $\!\!\!\rightarrow$ &  CO$_2$       + H        & $  1.8\!\times\! 10^{-14} \left(T/298\right)^{ 1.89}e^{  583/T}$ & Li07\\
 \refstepcounter{reaction}R\arabic{reaction}   & O$_2$            + CO      &$\!\!\!\rightarrow$ &  CO$_2$           + O       & $  3.65\!\times\! 10^{-13} e^{-15155/T}$ & Liu22\\
% \refstepcounter{reaction}R\arabic{reaction}   & O            + CO$_2$      &$\!\!\!\rightarrow$ &  CO           + O$_2$       & $  2.8\!\times\! 10^{-11} e^{-26500/T}$ & Ts86\\
  \refstepcounter{reaction}R\arabic{reaction}   & C            + OH          & $\!\!\!\rightarrow$ &  CO           + H           & $  1.1\!\times\! 10^{-10} \left(T/298 \right)^{ 0.50}$ & Mi97\\
\refstepcounter{reaction}R\arabic{reaction} & HO$_2$  + CO &$\!\!\!\rightarrow$ &  OH  +   CO$_2$   & $ 2.5\!\times\! 10^{-10} e^{-11900/T} $  & Ts86\\  

\multicolumn{6}{l}{\bf HCO, H$_2$CO}\\
 \refstepcounter{reaction}\label{RHCO} R\arabic{reaction}   & H   + CO     + M & $\!\!\!\rightarrow$ &  HCO    + M &$  5.2\!\times\! 10^{-33} \left(T/298 \right)^{-0.66} e^{  -825/T}$ & rev-Fr02\\
           & H            + CO          &$\!\!\!\rightarrow$&  HCO     &$  2.0\!\times\! 10^{-13} e^{  -1370/T}$ &  Ar81\\

 \refstepcounter{reaction}R\arabic{reaction}  & HCO          + O           &$\!\!\!\rightarrow$ &  H            + CO$_2$   & $  5.0\!\times\! 10^{-11}$ & Ts86\\
 \refstepcounter{reaction}R\arabic{reaction}  & HCO          + O           &$\!\!\!\rightarrow$ &  OH           + CO          & $  5.0\!\times\! 10^{-11}$ & Ts86\\
 \refstepcounter{reaction}R\arabic{reaction}  & OH           + HCO         &$\!\!\!\rightarrow$ &  CO           + H$_2$O     & $  1.7\!\times\! 10^{-10}$ & Ba92\\
 \refstepcounter{reaction}R\arabic{reaction}  & HCO          + H           &$\!\!\!\rightarrow$ &  CO           + H$_2$          & $  1.8\!\times\! 10^{-10}$ & Ba92\\
 \refstepcounter{reaction}R\arabic{reaction}  & HCO          + O$_2$           &$\!\!\!\rightarrow$ &  CO           + HO$_2$      & $  5.2\!\times\! 10^{-12}$ & At01 \\
\refstepcounter{reaction}\label{RH2CO} R\arabic{reaction}  & H            + HCO          + M&$\!\!\!\rightarrow$& H$_2$CO      + M &$  3.2\!\times\! 10^{-30} \left(T/298 \right)^{-2.57}e^{ -215/T}$ & Ei98\\
           & H            + HCO         &$\!\!\!\rightarrow$&  H$_2$CO       &$  3.0\!\times\! 10^{-10}$ & rev-Tr05\\

 \refstepcounter{reaction}\label{RHCHO} R\arabic{reaction}   &  H$_2$        + CO     +M   &$\!\!\!\rightarrow$ &   H$_2$CO      + M    & $  1.4\!\times\! 10^{-33} e^{-32900/T}$ & rev-Tr05\\
          &  H$_2$        + CO         &$\!\!\!\rightarrow$ &   H$_2$CO          & $  5.5\!\times\! 10^{-12} e^{-35900/T}$ & rev-Tr05\\
 \refstepcounter{reaction}R\arabic{reaction}  & HCO          + HCO         &$\!\!\!\rightarrow$ &  H$_2$CO      + CO   & $  4.5\!\times\! 10^{-11}$ & Ba92\\
 \refstepcounter{reaction}R\arabic{reaction}   & H$_2$CO      + H        & $\!\!\!\rightarrow$ &  HCO      + H$_2$   & $  1.5\!\times\! 10^{-11} \left(T/298\right)^{ 1.05}e^{ -1650/T}$ & Ba92 \\
 \refstepcounter{reaction}R\arabic{reaction}   & H$_2$CO      + OH   & $\!\!\!\rightarrow$ &  HCO     + H$_2$O  & $  4.8\!\times\! 10^{-12} \left(T/298\right)^{ 1.18}e^{   225/T}$ & Ba92 \\
 \refstepcounter{reaction}R\arabic{reaction}   & H$_2$CO      + O    &$\!\!\!\rightarrow$ &  OH      + HCO     & $  1.77\!\times\! 10^{-11} \left(T/298\right)^{ 0.57}e^{ -1390/T}$ & Ba92\\
 \refstepcounter{reaction}R\arabic{reaction} & H$_2$CO  + HO$_2$ &$\!\!\!\rightarrow$ &  H$_2$O$_2$  +   HCO  & $ 1.05\!\times\! 10^{-13} \left(T/298 \right)^{2.5} e^{-5140/T} $  & Ei98 \\  

\multicolumn{6}{l}{\bf CH}\\
% \multicolumn{6}{l}{\bf CH$_n$}\\
 \refstepcounter{reaction}\label{RCH} R\arabic{reaction}  & H      + C    + M        &$\!\!\!\rightarrow$ &  CH           + M        & $  5.0\!\times\! 10^{-34}$ &  rev De92\\
         & H      + C            &$\!\!\!\rightarrow$ &  CH          & $  1.0\!\times\! 10^{-11}$ & rev De92\\
 \refstepcounter{reaction}R\arabic{reaction}   & O            + CH          & $\!\!\!\rightarrow$ &  OH           + C     & $  1.73\!\times\! 10^{-11} \left(T/298\right)^{ 0.50}e^{ -4000/T}$ & Mi97\\
 \refstepcounter{reaction}R\arabic{reaction}  & O            + CH          &$\!\!\!\rightarrow$ &  CO           + H     & $  6.6\!\times\! 10^{-11}$ & Ba92 \\
 \refstepcounter{reaction}R\arabic{reaction}  & CH           + O$_2$       &$\!\!\!\rightarrow$ &  CO           + OH     & $  8.3\!\times\! 10^{-11}$ & Li84 \\
 \refstepcounter{reaction}R\arabic{reaction}   & CH           + CO$_2$      &$\!\!\!\rightarrow$ &  CO           + HCO     & $  5.9\!\times\! 10^{-12} e^{  -350/T}$ & Ba92\\
 \refstepcounter{reaction}R\arabic{reaction}   & CH           + H           &$\!\!\!\rightarrow$ &  H$_2$        + C              & $  1.3\!\times\! 10^{-10} e^{  -806/T}$ & Ha93 \\

\multicolumn{6}{l}{\bf CH$_2$}\\
 \refstepcounter{reaction}R\arabic{reaction}   & H$_2$        + C    + M  &$\!\!\!\rightarrow$&  CH$_2$       + M & $  7.0\!\times\! 10^{-32}$ & Hu75 \\
           & H$_2$        + C       &$\!\!\!\rightarrow$&  CH$_2$         &$  2.1\!\times\! 10^{-11}$ & Ha93 \\
 \refstepcounter{reaction}\label{RCH2} R\arabic{reaction}   & H        + CH + M          &$\!\!\!\rightarrow$&  CH$_2$       + M &$  7.0\!\times\! 10^{-32} $ &  assumed \\
           & H        + CH          &$\!\!\!\rightarrow$&  CH$_2$         &$  2.10\!\times\! 10^{-11}$ & assumed   \\
% \refstepcounter{reaction}R\arabic{reaction}   & CH  + H$_2$  & $\!\!\!\rightarrow$ &   CH$_2$  + H  & $ 2.4\!\times\! 10^{-10} e^{ -1760/T}$ & Ba92 \\ 
 \refstepcounter{reaction}R\arabic{reaction}   & H  + CH$_2$  & $\!\!\!\rightarrow$ &   H$_2$  + CH  & $ 2.7\!\times\! 10^{-10}$ & Ga17 \\ 
 
 \refstepcounter{reaction}R\arabic{reaction}   & OH           + CH$_2$      & $\!\!\!\rightarrow$ &  H$_2$CO      + H     & $  9.5\!\times\! 10^{-11} \left(T/298\right)^{ 0.12}e^{    81/T}$ & Ja07\\
 \refstepcounter{reaction}R\arabic{reaction}   & OH           + CH$_2$      & $\!\!\!\rightarrow$ &  H$_2$O       + CH   & $  1.4\!\times\! 10^{-13} \left(T/298\right)^{ 2.02}e^{ -3420/T}$ & Ja07\\
 \refstepcounter{reaction}R\arabic{reaction}  & CH$_2$       + O           &$\!\!\!\rightarrow$ &  HCO           + H              & $  1.2\!\times\! 10^{-10}$ & Ba92\\
 \refstepcounter{reaction}R\arabic{reaction}  & O            + CH$_2$      &$\!\!\!\rightarrow$ &  CO           + H$_2$        & $  8.0\!\times\! 10^{-11}$ & Ba92\\
 \refstepcounter{reaction}\label{RCH2+O2} R\arabic{reaction}   & O$_2$        + CH$_2$      &$\!\!\!\rightarrow$ &  H$_2$CO  + O                  & $  4.1\!\times\! 10^{-11} e^{  -750/T}$ & Ba92\\

\multicolumn{6}{l}{\bf CH$_3$}\\
 \refstepcounter{reaction}R\arabic{reaction}   & CH           + H$_2$        + M & $\!\!\!\rightarrow$ &  CH$_3$       + M &$  6.8\!\times\! 10^{-31} \left(T/298 \right)^{-2.30}$ & Be84 \\  % 3-body
           & CH           + H$_2$          & $\!\!\!\rightarrow$ &  CH$_3$         &$  2.0\!\times\! 10^{-10} \left(T/298 \right)^{0.15}$ & Fu97a\\
 \refstepcounter{reaction}R\arabic{reaction}  & H  + CH$_2$ + M & $\!\!\!\rightarrow$ &  CH$_3$ + M &$  1.0\!\times\! 10^{-30} \left(T/298 \right)^{-2.0}$ & assumed\\  
           & H   + CH$_2$   & $\!\!\!\rightarrow$ &  CH$_3$         &$  1.0\!\times\! 10^{-10} $ &  assumed \\
 \refstepcounter{reaction}R\arabic{reaction}   & CH$_2$       + H$_2$           &$\!\!\!\rightarrow$ &  CH$_3$       + H                     & $  1.7\!\times\! 10^{-11} e^{ -4780/T}$ & Lu10\\
% \refstepcounter{reaction}R\arabic{reaction}   & CH$_3$       + H           &$\!\!\!\rightarrow$ &  CH$_2$       + H$_2$                     & $  1.0\!\times\! 10^{-10} e^{ -7600/T}$ & Ba92\\
 \refstepcounter{reaction}R\arabic{reaction}  & CH$_3$       + O           &$\!\!\!\rightarrow$ &  H$_2$CO      + H                                       & $  9.0\!\times\! 10^{-11}$ & Ba92, Xu15\\
 \refstepcounter{reaction}R\arabic{reaction}  & CH$_3$       + O           &$\!\!\!\rightarrow$ &  HCO      + H$_2$                                       & $  6.0\!\times\! 10^{-11}$ & Ba92, Xu15\\
  
 \refstepcounter{reaction}R\arabic{reaction}  & CH$_2$       + HCO         &$\!\!\!\rightarrow$ &  CH$_3$       + CO  & $  3.0\!\times\! 10^{-11}$ & Ts86\\
 \refstepcounter{reaction}R\arabic{reaction}   & CH$_2$       + CH$_2$      &$\!\!\!\rightarrow$ &  CH           + CH$_3$          & $  4.0\!\times\! 10^{-10} e^{ -5000/T}$ & Fr84\\
 \refstepcounter{reaction}R\arabic{reaction} & O$_3$  +  CH$_3$   &$\!\!\!\rightarrow$ &  H$_2$CO   +  HO$_2$   & $ 5.1\!\times\! 10^{-12} e^{-210/T} $  & At01 \\  

 \multicolumn{6}{l}{\bf CH$_4$}\\
 \refstepcounter{reaction}R\arabic{reaction}   & CH$_3$     + H            + M & $\!\!\!\rightarrow$ &  CH$_4$       + M &$  6.5\!\times\! 10^{-29} \left(T/298 \right)^{-2.17}$ & Go08\\
           & CH$_3$     + H           &$\!\!\!\rightarrow$&  CH$_4$       + M &$  3.5\!\times\! 10^{-10}$ & Go08\\

 \refstepcounter{reaction}R\arabic{reaction}   & CH$_4$     + H   & $\!\!\!\rightarrow$ &  CH$_3$    + H$_2$   & $  4.36\!\times\! 10^{-13} \left(T/298\right)^{ 3.16}e^{ -4410/T}$ &  Sut02 \\

 \refstepcounter{reaction}R\arabic{reaction}   & CH$_4$     + O        & $\!\!\!\rightarrow$ &  CH$_3$    + OH  & $  2.43\!\times\! 10^{-12} \left(T/298\right)^{ 2.2}e^{ -3780/T}$ & Mu21 \\

 \refstepcounter{reaction}R\arabic{reaction}   & CH$_4$       + OH  & $\!\!\!\rightarrow$ &  CH$_3$       + H$_2$O & $  8.8\!\times\! 10^{-13} \left(T/298\right)^{ 1.83}e^{ -1400/T}$ & Ba92\\
 \refstepcounter{reaction}R\arabic{reaction}   & CH$_3$    + HCO  & $\!\!\!\rightarrow$ &  CH$_4$       + CO & $  4.4\!\times\! 10^{-11} $ & Mu87\\
 \refstepcounter{reaction}R\arabic{reaction}   & CH$_3$    + H$_2$CO  &    $\!\!\!\rightarrow$ &  CH$_4$       + HCO & $  4.9\!\times\! 10^{-15} \left(T/298\right)^{4.4}e^{ -2450/T}$ &  Li03\\
 \refstepcounter{reaction}R\arabic{reaction}   & CH$_2$    + CH$_4$  & $\!\!\!\rightarrow$ &  CH$_3$   + CH$_3$ & $  7.1\!\times\! 10^{-12}  e^{ -5050/T}$ & Bo85 \\
 \refstepcounter{reaction}R\arabic{reaction} & H$_2$O$_2$  + CH$_3$ &$\!\!\!\rightarrow$ &  HO$_2$  + CH$_4$   & $ 2.0\!\times\! 10^{-14} e^{+300/T} $  & Ts86 \\  

\multicolumn{6}{l}{\bf CH$_3$O}\\  % needs updating
% \refstepcounter{reaction}\label{RCH3O} R\arabic{reaction}  & H      + H$_2$CO      + M&$\!\!\!\rightarrow$& CH$_3$O      + M &$  1.9\!\times\! 10^{-29} \left(T/298 \right)^{-3.00}e^{ -3360/T}$ & rev-Hi01\\
%          & H      + H$_2$CO       & $\!\!\!\rightarrow$ &  CH$_3$O       &$  8.0\!\times\! 10^{-10} e^{ -3160/T}$ &  rev-Ra03 \\
 \refstepcounter{reaction}R\arabic{reaction}  & H      + H$_2$CO      + M&$\!\!\!\rightarrow$& CH$_3$O      + M &$  4.5\!\times\! 10^{-31} \left(T/298 \right)^{-3.0}e^{ -2900/T}$ & Xu15\\
          & H      + H$_2$CO       & $\!\!\!\rightarrow$ &  CH$_3$O       &$  2.7\!\times\! 10^{-09} \left(T/298 \right)^{-5.0} e^{ -4000/T}$ &  Xu15 \\
 \refstepcounter{reaction}R\arabic{reaction}   & CH$_3$  + OH   & $\!\!\!\rightarrow$ &  CH$_3$O  + H   & $  4.5\!\times\! 10^{-14} \left(T/298\right)^{ 1.00}e^{ -6010/T}$ & Ja07\\
 % \refstepcounter{reaction}R\arabic{reaction}   & H            + CH$_3$O     &$\!\!\!\rightarrow$ &  CH$_3$       + OH     & $  7.7\!\times\! 10^{-11} e^{  -375/T}$ & this one is doubtful\\
% \refstepcounter{reaction}\label{R80} R\arabic{reaction}   & CH$_3$O   + H  &$\!\!\!\rightarrow$ &  H$_2$CO      + H$_2$     & $  3.3\!\times\! 10^{-11} e^{  -375/T}$ & Mo77, Dob91\\
\refstepcounter{reaction}R\arabic{reaction}   & CH$_3$O   + H  &$\!\!\!\rightarrow$ &  H$_2$CO      + H$_2$     & $  3.14\!\times\! 10^{-10} \left(T/298\right)^{-0.58}e^{  -855/T}$ & Li04\\
 \refstepcounter{reaction}R\arabic{reaction}   &  CH$_3$O  + CO   &$\!\!\!\rightarrow$ &  CH$_3$ + CO$_2$     & $  1.33\!\times\! 10^{-11} e^{  -5940/T}$ & Hi00\\
 \refstepcounter{reaction}R\arabic{reaction}   &  CH$_3$O  + O$_2$   &$\!\!\!\rightarrow$ & H$_2$CO + HO$_2$     & $  7.2\!\times\! 10^{-14} e^{  -1080/T}$ &  At01  \\
 
 \refstepcounter{reaction}R\arabic{reaction}   &  CH$_3$O  + O   &$\!\!\!\rightarrow$ & H$_2$CO + OH     & $  2.5\!\times\! 10^{-11} $ &   Ba92  \\
 \refstepcounter{reaction}R\arabic{reaction}   &  CH$_3$O  + OH   &$\!\!\!\rightarrow$ & H$_2$CO + H$_2$O     & $  9.0\!\times\! 10^{-16}  \left(T/298 \right)^{2.5} e^{ -950/T}$ &  Ja09   \\
\refstepcounter{reaction}R\arabic{reaction} & CH$_3$O  +   HCO   &$\!\!\!\rightarrow$ &     H$_2$CO   +  H$_2$CO    &   $3.0\!\times\! 10^{-11}$ & invented \\  

\multicolumn{6}{l}{\bf H$_2$COH}\\
% \refstepcounter{reaction}R\arabic{reaction}   & H         + H$_2$CO      + M & $\!\!\!\rightarrow$ &  H$_2$COH     + M &$  3.0\!\times\! 10^{-33} e^{  -600/T}$ &  rev-Hi89  \\
%             & H       + H$_2$CO         & $\!\!\!\rightarrow$ &  H$_2$COH        &$  3.0\!\times\! 10^{-14} e^{  -600/T}$ & rev-Tsu81\\
 \refstepcounter{reaction}R\arabic{reaction}   & H         + H$_2$CO      + M & $\!\!\!\rightarrow$ &  H$_2$COH     + M &$  3.0\!\times\! 10^{-32} \left(T/298 \right)^{-1.2} e^{  -2900/T}$ &  Xu15  \\
             & H       + H$_2$CO         & $\!\!\!\rightarrow$ &  H$_2$COH        &$  3.0\!\times\! 10^{-12} e^{  -3500/T}$ & Xu15 \\
 \refstepcounter{reaction}R\arabic{reaction}  & H          + H$_2$COH    &$\!\!\!\rightarrow$ &  H$_2$CO      + H$_2$    & $  1.7\!\times\! 10^{-11}$ & Cr92\\
 \refstepcounter{reaction}R\arabic{reaction}  & CH$_3$   + OH    & $\!\!\!\rightarrow$ &  H$_2$COH     + H & $  3.2\!\times\! 10^{-12} \left(T/298\right)^{ 1.00}e^{ -1600/T}$ & Ja07\\
% R60  & H          + H$_2$COH    &$\!\!\!\rightarrow$ &  CH$_3$       + OH         & $  5.0\!\times\! 10^{-11}$ & Ts87\\
 \refstepcounter{reaction}R\arabic{reaction}  & HCO     + H$_2$COH    &$\!\!\!\rightarrow$ &  H$_2$CO      + H$_2$CO    & $  3.0\!\times\! 10^{-11}$ & Ts87\\
 \refstepcounter{reaction}R\arabic{reaction}   &  H$_2$COH  + O   &$\!\!\!\rightarrow$ & H$_2$CO + OH     & $  7.0\!\times\! 10^{-11} $ &    Ts87 \\
 \refstepcounter{reaction}R\arabic{reaction}   &  H$_2$COH  + OH   &$\!\!\!\rightarrow$ & H$_2$CO + H$_2$O     & $  2.1\!\times\! 10^{-11} $ &   Ma18  \\

% BREAK
\multicolumn{6}{l}{\bf CH$_3$OH}\\
\refstepcounter{reaction}\label{RCH3OH}R\arabic{reaction}   & H       + CH$_3$O   +M  &$\!\!\!\rightarrow$ &  CH$_3$OH + M    & $  1.6\!\times\! 10^{-29}\left(T/298 \right)^{0.24} e^{  26.5/T}$ & note\\
          & H       + CH$_3$O       &$\!\!\!\rightarrow$ &  CH$_3$OH       & $  1.6\!\times\! 10^{-10}\left(T/298 \right)^{0.24} e^{  26.5/T}$ & Ja07\\
\refstepcounter{reaction}R\arabic{reaction}   & H            + H$_2$COH   +M  &$\!\!\!\rightarrow$ &  CH$_3$OH + M    & $ 2.9\!\times\! 10^{-29}\left(T/298 \right)^{0.04} $ & note\\
          & H            + H$_2$COH      &$\!\!\!\rightarrow$ &  CH$_3$OH   & $ 2.9\!\times\! 10^{-10}\left(T/298 \right)^{0.04} $ & Ja07 \\
\refstepcounter{reaction}R\arabic{reaction}   & H$_2$     + H$_2$CO   +M  &$\!\!\!\rightarrow$ &  CH$_3$OH + M    & $ 2.3\!\times\! 10^{-30} e^{ -35100/T} $ & note\\
          & H$_2$     + H$_2$CO      &$\!\!\!\rightarrow$ &  CH$_3$OH    & $ 2.3\!\times\! 10^{-11} e^{ -35100/T} $ &  Ja07  \\
\refstepcounter{reaction}R\arabic{reaction}   & OH      + CH$_3$    +M  &$\!\!\!\rightarrow$ &  CH$_3$OH + M    & $  2.25\!\times\! 10^{-24}\left(T/298 \right)^{-8.2} $ &  Ba94\\
           & OH     + CH$_3$     &$\!\!\!\rightarrow$ &  CH$_3$OH      & $  1.0\!\times\! 10^{-10} $ &  Ba94 \\
 \refstepcounter{reaction}R\arabic{reaction}   & H  + CH$_3$OH    & $\!\!\!\rightarrow$ &  CH$_3$O      + H$_2$  & $  1.67\!\times\! 10^{-14} \left(T/298\right)^{ 4.0}e^{ -3600/T}$ & Me11,Jo99\\
\refstepcounter{reaction}R\arabic{reaction}   & H   + CH$_3$OH    & $\!\!\!\rightarrow$ &  H$_2$COH     + H$_2$  & $ 1.59\!\times\! 10^{-13} \left(T/298\right)^{ 3.3}e^{ -1700/T}$ & Me11,Jo99\\
% \refstepcounter{reaction}R\arabic{reaction}   & H + CH$_3$OH & $\!\!\!\rightarrow$ &  CH$_3$       + H$_2$O  & $  1.7\!\times\! 10^{-13} e^{ -3600/T}$ & 2\% at 1000 K, No89\\
 \refstepcounter{reaction}R\arabic{reaction}   & CH$_3$OH    + O   & $\!\!\!\rightarrow$ &  OH   + H$_2$COH   & $ 3.6\!\times\! 10^{-11} e^{  -2590/T}$ & Fa82,Ka81,Gr81\\
 \refstepcounter{reaction}R\arabic{reaction}   & CH$_3$OH    + OH   & $\!\!\!\rightarrow$ &  H$_2$O   + CH$_3$O   & $  4.6\!\times\! 10^{-13} \left(T/298\right)^{ 2.00}e^{  -757/T}$ & Li96\\
 \refstepcounter{reaction}R\arabic{reaction}   & CH$_3$OH  + OH   & $\!\!\!\rightarrow$ &  H$_2$O   + H$_2$COH  & $  2.1\!\times\! 10^{-13} \left(T/298\right)^{ 2.00}e^{   423/T}$ & Li96\\
 \refstepcounter{reaction}R\arabic{reaction}   & CH$_3$   + CH$_3$OH    & $\!\!\!\rightarrow$ &  CH$_4$   + CH$_3$O  & $  2.6\!\times\! 10^{-16} \left(T/298\right)^{ 4.70}e^{ -2910/T}$ & Jo99\\
 \refstepcounter{reaction}R\arabic{reaction}   & CH$_3$  + CH$_3$OH    & $\!\!\!\rightarrow$ &  CH$_4$   +H$_2$COH &  $  1.4\!\times\! 10^{-15} \left(T/298\right)^{ 4.90}e^{ -3380/T}$ &Jo99 \\
 \refstepcounter{reaction}\label{reduced}R\arabic{reaction}  & HCO  + CH$_3$O   &$\!\!\!\rightarrow$ &  CH$_3$OH  + CO & $  1.5\!\times\! 10^{-10}$ & Ts86  \\
 \refstepcounter{reaction}R\arabic{reaction}  & HCO   + H$_2$COH    &$\!\!\!\rightarrow$ &  CH$_3$OH     + CO      & $  6.7\!\times\! 10^{-11}$ & Ts87\\
 \refstepcounter{reaction}R\arabic{reaction}   & CH$_3$O + CH$_3$O  & $\!\!\!\rightarrow$ &  CH$_3$OH   +H$_2$CO &  $  1.0\!\times\! 10^{-10} $ &  Ts86\\
\refstepcounter{reaction}\label{RCH3O+H2COH}R\arabic{reaction} & CH$_3$O + H$_2$COH  &$\!\!\!\rightarrow$ & CH$_3$OH  + H$_2$CO & $3.0\!\times\! 10^{-11}$ & note\\
\refstepcounter{reaction}\label{RH2COH+H2COH}R\arabic{reaction} & H$_2$COH  +  H$_2$COH   &$\!\!\!\rightarrow$ &   CH$_3$OH  +  H$_2$CO    &   $3.0\!\times\! 10^{-11}$ & note \\

 \multicolumn{6}{l}{\bf CH$_3$O$_2$}\\
  \refstepcounter{reaction}R\arabic{reaction} &  CH$_3$  +   O$_2$   + M &$\!\!\!\rightarrow$ &   CH$_3$O$_2$ + M & $ 7.1\!\times\! 10^{-31}   \left(T/298 \right)^{-3.0}  $   & Fe06  \\     
          & CH$_3$  +   O$_2$  &$\!\!\!\rightarrow$ &    CH$_3$O$_2$   & $ 1.8\!\times\! 10^{-12}  $    &  At97\\  
\refstepcounter{reaction}R\arabic{reaction} & CH$_3$O$_2$  + H   &$\!\!\!\rightarrow$ & CH$_3$O + OH  & $ 1.2\!\times\! 10^{-10}  $ & Ts86, Zh17 \\  
\refstepcounter{reaction}R\arabic{reaction} & CH$_3$O$_2$  + H   &$\!\!\!\rightarrow$ & CH$_4$ + O$_2$  & $ 1.2\!\times\! 10^{-11} \left(T/298 \right)^{1.0}  e^{-8350/T} $ &  Bo04\\  
\refstepcounter{reaction}R\arabic{reaction} & CH$_3$O$_2$  + O   &$\!\!\!\rightarrow$ & CH$_3$O + O$_2$  & $ 6.0\!\times\! 10^{-11} $ & Ts86 \\  
\refstepcounter{reaction}R\arabic{reaction} & CH$_3$O$_2$  + OH   &$\!\!\!\rightarrow$ & CH$_3$OH + O$_2$  & $ 1.0\!\times\! 10^{-11} $ &  Zh19 \\  
\refstepcounter{reaction}R\arabic{reaction} & CH$_3$O$_2$  + OH   &$\!\!\!\rightarrow$ & CH$_3$O + HO$_2$  & $ 8.0\!\times\! 10^{-11} $ &  Zh19 \\  
\refstepcounter{reaction}R\arabic{reaction} & CH$_3$O$_2$  + CH$_3$   &$\!\!\!\rightarrow$ & CH$_3$O + CH$_3$O  & $ 4.0\!\times\! 10^{-11} $ &  Ts86 \\  
\refstepcounter{reaction}R\arabic{reaction} & CH$_3$O$_2$  + NO   &$\!\!\!\rightarrow$ & CH$_3$O + NO$_2$  & $ 2.8\!\times\! 10^{-12} e^{+285/T} $ & At01 \\  

%BREAK

\multicolumn{6}{l}{\bf C$_2$H}\\
 \refstepcounter{reaction}R\arabic{reaction}   & O  + C$_2$   & $\!\!\!\rightarrow$ &  CO   + C  & $  5.0\!\times\! 10^{-11} \left(T/298 \right)^{ 0.50}$ & Mi97\\
 \refstepcounter{reaction}R\arabic{reaction}   & CH   + C  & $\!\!\!\rightarrow$ &  C$_2$    + H  & $  1.7\!\times\! 10^{-10} \left(T/298 \right)^{ 0.50}$ & Mi97\\
 \refstepcounter{reaction}R\arabic{reaction}   & C$_2$  + H$_2$   &$\!\!\!\rightarrow$ &  C$_2$H   + H   & $  1.1\!\times\! 10^{-10} e^{ -4000/T}$ & Kr97\\
 \refstepcounter{reaction}R\arabic{reaction}   & C$_2$  + CH$_4$   &$\!\!\!\rightarrow$ &  C$_2$H   + CH$_3$   & $  5.0\!\times\! 10^{-11} e^{ -4000/T}$ & Mo96\\
 \refstepcounter{reaction}R\arabic{reaction}  & O       + C$_2$H      &$\!\!\!\rightarrow$ &  CO    + CH     & $  1.7\!\times\! 10^{-11}$ & Wa84\\
 \refstepcounter{reaction}R\arabic{reaction}  & C$_2$H       + O$_2$       &$\!\!\!\rightarrow$ &  HCO   + CO   & $  3.0\!\times\! 10^{-11}$ & Ba92\\
 
% this next reaction is missing
 \refstepcounter{reaction}R\arabic{reaction}   & CH$_2$  + C   & $\!\!\!\rightarrow$ &  C$_2$H   + H   & $  5.0\!\times\! 10^{-11} \left(T/298 \right)^{ 0.50}$ & Mi97\\
% \refstepcounter{reaction}R\arabic{reaction}  & CH$_3$   + C  &$\!\!\!\rightarrow$ &  C$_2$H    + H$_2$   & $  5.0\!\times\! 10^{-11}$ & \\


\multicolumn{6}{l}{\bf C$_2$H$_2$}\\
 \refstepcounter{reaction}\label{RC2H2}R\arabic{reaction}   & C$_2$H       + H            + M & $\!\!\!\rightarrow$ &  C$_2$H$_2$   + M &$  6.0\!\times\! 10^{-29} \left(T/298 \right)^{-1.80}$ & note  \\
            & C$_2$H       + H           &$\!\!\!\rightarrow$&  C$_2$H$_2$   + M &$  3.0\!\times\! 10^{-10}$ & Ts86\\
 \refstepcounter{reaction}R\arabic{reaction}   & CH$_2$       + CH$_2$      &$\!\!\!\rightarrow$ &  C$_2$H$_2$   + H$_2$          & $  2.0\!\times\! 10^{-11} e^{  -400/T}$ & Ba92\\
  \refstepcounter{reaction}R\arabic{reaction}   & C$_2$H    + H$_2$     & $\!\!\!\rightarrow$ &  C$_2$H$_2$   + H    & $  2.3\!\times\! 10^{-12} \left(T/298\right)^{ 2.22}e^{  -461/T}$ & Ei03\\
 \refstepcounter{reaction}R\arabic{reaction}  & C$_2$H  + CH$_4$  &$\!\!\!\rightarrow$ &  C$_2$H$_2$   + CH$_3$  & $  4.2\!\times\! 10^{-12} \left(T/298\right)^{1.6}e^{  -300/T}$ & Ma11,Bo20\\
 \refstepcounter{reaction}R\arabic{reaction}   & C$_2$H$_2$   + O       & $\!\!\!\rightarrow$ &  CH$_2$       + CO      & $  1.8\!\times\! 10^{-12} \left(T/298\right)^{ 2.00}e^{  -956/T}$ & Ei03\\
 \refstepcounter{reaction}R\arabic{reaction}  & C$_2$H       + OH          &$\!\!\!\rightarrow$ &  CH$_2$       + CO        & $  3.0\!\times\! 10^{-11}$ & Ts86\\
 \refstepcounter{reaction}R\arabic{reaction}  & C$_2$H  + H$_2$O  & $\!\!\!\rightarrow$ &  OH  + C$_2$H$_2$  & $  1.0\!\times\! 10^{-12} \left(T/298\right)^{2.7}e^{ -1120/T}$ & rev Ei03\\
 % \refstepcounter{reaction}R\arabic{reaction}   & C$_2$H$_2$  + OH & $\!\!\!\rightarrow$ &  C$_2$H  + H$_2$O   & $  7.5\!\times\! 10^{-12} \left(T/298\right)^{2.0}e^{ -7050/T}$ & Ei03\\
%   \refstepcounter{reaction}R\arabic{reaction}   & C$_2$H$_2$   + OH     &$\!\!\!\rightarrow$ &  CH$_3$       + CO      & $  6.3\!\times\! 10^{-18}  \left(T/298\right)^{4.0} e^{ -10100/T}$ & Mi88\\
%  there does not seem to be any evidence for this
 \refstepcounter{reaction}R\arabic{reaction} & HO$_2$ + C$_2$H  &$\!\!\!\rightarrow$ &  C$_2$H$_2$  +   O$_2$   & $ 3.0\!\times\! 10^{-11} $  & invented\\  


\multicolumn{6}{l}{\bf C$_2$H$_3$}\\
 \refstepcounter{reaction}\label{RC2H3}R\arabic{reaction} & H     + C$_2$H$_2$   + M&$\!\!\!\rightarrow$& C$_2$H$_3$   + M &$ 3.3\!\times\! 10^{-30} e^{ -740/T}$ & Ba92\\
    & H   + C$_2$H$_2$    & $\!\!\!\rightarrow$ &  C$_2$H$_3$    &$  9.0\!\times\! 10^{-12} e^{ -1220/T}$ & Wa84\\
 \refstepcounter{reaction}R\arabic{reaction}  & CH$_2$  + CH$_2$  &$\!\!\!\rightarrow$ &  C$_2$H$_3$   + H             & $  2.0\!\times\! 10^{-10}e^{ -400/T}$ & Ba92\\  % this is "products"
% the actual products are C2H2 and H + H.  I think C2H3 + H is best workaround
% \refstepcounter{reaction}R\arabic{reaction}  & CH$_2$  + CH$_2$  &$\!\!\!\rightarrow$ &  C$_2$H$_2$   + H$_2$       & $  2.0\!\times\! 10^{-10}$ & Ba92\\  % this is 10\% channel
 % \refstepcounter{reaction}R\arabic{reaction}  & CH$_2$  + CH$_2$  &$\!\!\!\rightarrow$ &  C$_2$H$_2$   + H + H      & $  2.0\!\times\! 10^{-10}$ & Ba92 - actual channel \\
 \refstepcounter{reaction}R\arabic{reaction}  & H    + C$_2$H$_3$  &$\!\!\!\rightarrow$ &  H$_2$        + C$_2$H$_2$             & $  2.0\!\times\! 10^{-11}$ & Ba92\\
  \refstepcounter{reaction}R\arabic{reaction}  & C$_2$H$_3$   + OH          &$\!\!\!\rightarrow$ &  C$_2$H$_2$   + H$_2$O         & $  1.0\!\times\! 10^{-12}$ & Kny17 \\
\refstepcounter{reaction}R\arabic{reaction}  & C$_2$H$_3$   + OH          &$\!\!\!\rightarrow$ &  CH$_4$   + HCO         & $  2.5\!\times\! 10^{-12} \left(T/298\right)^{-0.7}$ & Kny17 \\
 \refstepcounter{reaction}R\arabic{reaction}  & O            + C$_2$H$_3$  &$\!\!\!\rightarrow$ &  OH   + C$_2$H$_2$    & $  5.0\!\times\! 10^{-12}\left(T/298\right)^{0.2}e^{+215/T}$ & Ha05\\
 \refstepcounter{reaction}R\arabic{reaction}  & O            + C$_2$H$_3$  &$\!\!\!\rightarrow$ &  CH$_3$       + CO        & $  5.0\!\times\! 10^{-11}$ & La04\\
 \refstepcounter{reaction}R\arabic{reaction}  & O$_2$        + C$_2$H$_3$  &$\!\!\!\rightarrow$ &  H$_2$CO      + HCO          & $  9.0\!\times\! 10^{-12}$ & Ba92\\
  \refstepcounter{reaction}R\arabic{reaction}  & O$_2$        + C$_2$H$_3$  &$\!\!\!\rightarrow$ &  C$_2$H$_2$   + HO$_2$      & $  6.6\!\times\! 10^{-12}\left(T/298\right)^{-1.26}e^{-1660/T}$ & Ma98 \\
 \refstepcounter{reaction}R\arabic{reaction}  & CH$_3$       + C$_2$H$_3$  &$\!\!\!\rightarrow$ &  C$_2$H$_2$   + CH$_4$       & $  3.0\!\times\! 10^{-11}$ & La04\\
 \refstepcounter{reaction}R\arabic{reaction}  & CH$_2$       + C$_2$H$_3$  &$\!\!\!\rightarrow$ &  C$_2$H$_2$   + CH$_3$        & $  3.0\!\times\! 10^{-11}$ & Ba92\\

 \refstepcounter{reaction}R\arabic{reaction}  & C$_2$H       + C$_2$H$_3$  &$\!\!\!\rightarrow$ &  C$_2$H$_2$   + C$_2$H$_2$       & $  1.6\!\times\! 10^{-12}$ & Ts86\\


\multicolumn{6}{l}{\bf C$_2$H$_4$}\\
% \refstepcounter{reaction}R\arabic{reaction}   & H        + C$_2$H$_3$   + M & $\!\!\!\rightarrow$ &  C$_2$H$_4$   + M &$  1.8\!\times\! 10^{-27} \left(T/298\right)^{ -4.5} $ &  rev-Ba94\\  % I made this up
 \refstepcounter{reaction}\label{RC2H4}R\arabic{reaction}   & H        + C$_2$H$_3$   + M & $\!\!\!\rightarrow$ &  C$_2$H$_4$   + M &$  1.0\!\times\! 10^{-28} \left(T/298\right)^{ -2.0} $ &  note \\   
          & H     + C$_2$H$_3$    & $\!\!\!\rightarrow$ &  C$_2$H$_4$   &$  2.0\!\times\! 10^{-10}$ & Ha05\\
 \refstepcounter{reaction}\label{RC2H2+H2}R\arabic{reaction}   & C$_2$H$_2$   + H$_2$        + M & $\!\!\!\rightarrow$ &  C$_2$H$_4$   + M &$  2.5\!\times\! 10^{-33} e^{-14900/T}$ & rev-Ba92 \\
             & C$_2$H$_2$   + H$_2$      & $\!\!\!\rightarrow$ &  C$_2$H$_4$     &$  5.0\!\times\! 10^{-13} e^{-19600/T}$ & Ts86\\
 \refstepcounter{reaction}R\arabic{reaction}  & CH$_2$       + CH$_3$      &$\!\!\!\rightarrow$ &  C$_2$H$_4$   + H   & $  7.0\!\times\! 10^{-11}$ & Ba92\\
 \refstepcounter{reaction}\label{RCH+CH4}R\arabic{reaction}  & CH           + CH$_4$      &$\!\!\!\rightarrow$ &  C$_2$H$_4$   + H              & $  7.6\!\times\! 10^{-11}$ & Fl02\\
 % \refstepcounter{reaction}R\arabic{reaction}   & C$_2$H$_4$  + H & $\!\!\!\rightarrow$ &  C$_2$H$_3$   + H$_2$ & $ 5.0\!\times\! 10^{-12} \left(T/298\right)^{ 1.93}e^{-6520/T}$ & Kny96\\
 \refstepcounter{reaction}R\arabic{reaction}   & C$_2$H$_3$   + H$_2$ & $\!\!\!\rightarrow$ &  C$_2$H$_4$  + H & $ 3.14\!\times\! 10^{-12} \left(T/298\right)^{ 0.7}e^{-3630/T}$ & Kny96\\
 \refstepcounter{reaction}R\arabic{reaction}   & O    + C$_2$H$_4$  & $\!\!\!\rightarrow$ &  HCO     + CH$_3$    & $  8.2\!\times\! 10^{-13} \left(T/298 \right)^{ 2.08}$ & Ba92\\
 \refstepcounter{reaction}R\arabic{reaction}  & OH  + C$_2$H$_4$  & $\!\!\!\rightarrow$ &  C$_2$H$_3$ + H$_2$O  & $5.42\!\times\! 10^{-15} \left(T/298\right)^{ 4.2}e^{ +433/T}$ & Se06b\\
% \refstepcounter{reaction}R\arabic{reaction}  & OH  + C$_2$H$_4$  & $\!\!\!\rightarrow$ &  H$_2$CO + CH$_3$  & $4.87\!\times\! 10^{-17} \left(T/298\right)^{3.34}e^{ +1400/T}$ & Se06b\\  % low pressure limit
 \refstepcounter{reaction}R\arabic{reaction}   & C$_2$H$_4$   + CH$_3$      & $\!\!\!\rightarrow$ &  CH$_4$       + C$_2$H$_3$   & $  1.57\!\times\! 10^{-14} \left(T/298\right)^{ 3.7}e^{ -4780/T}$ & Ts86 \\
 \refstepcounter{reaction}R\arabic{reaction}  & C$_2$H$_3$   + C$_2$H$_3$  &$\!\!\!\rightarrow$ &  C$_2$H$_2$   + C$_2$H$_4$      & $  2.4\!\times\! 10^{-11}$ & La04\\


\multicolumn{6}{l}{\bf C$_2$H$_5$}\\
 \refstepcounter{reaction}\label{RC2H5}R\arabic{reaction}   & H        + C$_2$H$_4$   + M & $\!\!\!\rightarrow$ &  C$_2$H$_5$   + M &$  7.7\!\times\! 10^{-30} e^{ -380/T}$ & Ba94\\
          & H     + C$_2$H$_4$      & $\!\!\!\rightarrow$ &  C$_2$H$_5$      &$  9.0\!\times\! 10^{-12} \left(T/298 \right)^{1.75} e^{ -605/T}$ & Mi05\\
 \refstepcounter{reaction}R\arabic{reaction}  & H            + C$_2$H$_5$  &$\!\!\!\rightarrow$ &  CH$_3$       + CH$_3$      & $  6.0\!\times\! 10^{-11}$ & Ba92\\
 \refstepcounter{reaction}R\arabic{reaction}  & H     + C$_2$H$_5$  &$\!\!\!\rightarrow$ &  C$_2$H$_4$   + H$_2$          & $  3.0\!\times\! 10^{-12}$ & Ts86\\
 \refstepcounter{reaction}R\arabic{reaction}  & O            + C$_2$H$_5$  &$\!\!\!\rightarrow$ &  H$_2$CO      + CH$_3$      & $  1.1\!\times\! 10^{-10}$ & Ba92\\
 \refstepcounter{reaction}R\arabic{reaction}  & C$_2$H$_5$   + OH          &$\!\!\!\rightarrow$ &  C$_2$H$_4$   + H$_2$O            & $  4.0\!\times\! 10^{-11}$ & Ts86\\
 \refstepcounter{reaction}R\arabic{reaction}  & CH$_2$       + C$_2$H$_5$  &$\!\!\!\rightarrow$ &  C$_2$H$_4$   + CH$_3$      & $  3.0\!\times\! 10^{-11}$ & Ba92\\
 \refstepcounter{reaction}R\arabic{reaction}  & CH$_3$       + C$_2$H$_5$  &$\!\!\!\rightarrow$ &  C$_2$H$_4$   + CH$_4$        & $  1.9\!\times\! 10^{-12}$ & Ba92\\

\multicolumn{6}{l}{\bf C$_2$H$_6$}\\
 \refstepcounter{reaction}R\arabic{reaction} & CH$_3$  + CH$_3$   + M&$\!\!\!\rightarrow$& C$_2$H$_6$   + M &$  1.17\!\times\! 10^{-25} \left(T/298 \right)^{-3.75}e^{  -494/T}$ & Wa03a\\
            & CH$_3$       + CH$_3$      &$\!\!\!\rightarrow$&  C$_2$H$_6$     &$  6.0\!\times\! 10^{-11}$ & Ba94\\
 \refstepcounter{reaction}R\arabic{reaction}   & O       + C$_2$H$_6$  & $\!\!\!\rightarrow$ &  C$_2$H$_5$   + OH    & $  8.6\!\times\! 10^{-12} \left(T/298\right)^{ 1.50}e^{ -2920/T}$ & Ba92\\
 \refstepcounter{reaction}R\arabic{reaction}   & OH     + C$_2$H$_6$  & $\!\!\!\rightarrow$ &  C$_2$H$_5$   + H$_2$O    & $  1.1\!\times\! 10^{-12} \left(T/298\right)^{ 2.00}e^{  -435/T}$ & Ba92\\
 \refstepcounter{reaction}R\arabic{reaction}   & H      + C$_2$H$_6$  & $\!\!\!\rightarrow$ &  H$_2$  + C$_2$H$_5$  & $  4.2\!\times\! 10^{-13} \left(T/298\right)^{ 3.50}e^{ -2600/T}$ & Ts86\\
 \refstepcounter{reaction}R\arabic{reaction}   & CH$_3$       + C$_2$H$_6$  & $\!\!\!\rightarrow$ &  C$_2$H$_5$   + CH$_4$      & $  7.2\!\times\! 10^{-15} \left(T/298\right)^{ 4.00}e^{ -4170/T}$ & Ts86\\
  \refstepcounter{reaction}R\arabic{reaction}  & C$_2$H       + C$_2$H$_6$  &$\!\!\!\rightarrow$ &  C$_2$H$_2$   + C$_2$H$_5$           & $  3.6\!\times\! 10^{-11}$ & La90\\
 \refstepcounter{reaction}R\arabic{reaction}  & C$_2$H$_5$   + C$_2$H$_5$  &$\!\!\!\rightarrow$ &  C$_2$H$_4$   + C$_2$H$_6$          & $  2.4\!\times\! 10^{-12}$ & Ba92\\
 \refstepcounter{reaction}R\arabic{reaction}   & C$_2$H$_6$   + C$_2$H$_3$  & $\!\!\!\rightarrow$ &  C$_2$H$_4$   + C$_2$H$_5$      & $  1.5\!\times\! 10^{-13} \left(T/298\right)^{ 3.30}e^{ -5280/T}$ & Ts86\\


\multicolumn{6}{l}{{\bf C$_2$H$_2$OH} and {\bf CH$_2$CHO}}\\
 \refstepcounter{reaction}\label{RC2H2OH}R\arabic{reaction}   & C$_2$H$_2$   + OH + M & $\!\!\!\rightarrow$ &  C$_2$H$_2$OH + M &$  5.6\!\times\! 10^{-30} \left(T/298 \right)^{-2.00}$ & Ba92\\  %stabilized C2H2OH will be important only at high pressures, but OH is important only at low pressures. In either case the chief radical is atomic hydrogen.  
    & C$_2$H$_2$   + OH   & $\!\!\!\rightarrow$ &  C$_2$H$_2$OH   &$  2.3\!\times\! 10^{-12} e^{  -230/T}$ & Ba92\\ % estimated heat of formation for C$_2$H$_2$OH for reverse reaction from Lai92. 
 \refstepcounter{reaction}\label{RC2H2OH+M}R\arabic{reaction}   & C$_2$H$_2$OH + M & $\!\!\!\rightarrow$ &  CH$_2$CHO + M &$  1.0\!\times\! 10^{-11} e^{ -10000/T}$ &  note \\ %The activation barrier against rearrangement to CH2CHO may be considerable.   We assume a rate with a barrier of order 100 kJ/mol, which is consistent with analogous barriers such as that between CH2CHO and CH3CO or in the re-arrangement of HCN adducts (Dean Bozelli).

 \refstepcounter{reaction}\label{RO+CH2CHO}R\arabic{reaction}   & O + CH$_2$CHO   & $\!\!\!\rightarrow$ & CH$_3$ + CO$_2$  & $  8.0\!\times\! 10^{-11} $ &  note  \\  
  \refstepcounter{reaction}\label{RH+C2H2OH}R\arabic{reaction}   & C$_2$H$_2$OH + H & $\!\!\!\rightarrow$ &  C$_2$H$_2$ + H$_2$O &$  3.5\!\times\! 10^{-12} e^{ -350/T}$ & note \\ 
\refstepcounter{reaction}R\arabic{reaction} & C$_2$H$_3$  + OH   &$\!\!\!\rightarrow$ & CH$_2$CHO  + H  & $5.6\!\times\! 10^{-12}\left(T/298 \right)^{0.26} e^{+220/T}$ & Kny17\\  

\multicolumn{6}{l}{\bf CH$_2$CO}\\
 \refstepcounter{reaction}\label{RC2H2OH+H}R\arabic{reaction}   & C$_2$H$_2$OH + H & $\!\!\!\rightarrow$ &  CH$_2$CO + H$_2$ & $  1.0\!\times\! 10^{-10}e^{ -1000/T}$ & note \\  
 \refstepcounter{reaction}R\arabic{reaction}   & C$_2$H$_2$OH + O & $\!\!\!\rightarrow$ &  CH$_2$CO + OH &$  5.0\!\times\! 10^{-11}e^{ -1000/T} $ &  note \\  
 \refstepcounter{reaction}\label{RC2H2OH+OH}R\arabic{reaction} & C$_2$H$_2$OH + OH & $\!\!\!\rightarrow$ & CH$_2$CO + H$_2$O &$  5.0\!\times\! 10^{-11}e^{ -500/T} $ & note \\  
 \refstepcounter{reaction}\label{RCH2CHO+H}R\arabic{reaction}   & H + CH$_2$CHO   & $\!\!\!\rightarrow$ & H$_2$ + CH$_2$CO &$  2.0\!\times\! 10^{-11} $ &  note \\  
   \refstepcounter{reaction}R\arabic{reaction}   & O + CH$_2$CHO   & $\!\!\!\rightarrow$ & OH + CH$_2$CO &$  2.0\!\times\! 10^{-11} $ &  note \\  
\refstepcounter{reaction}\label{RCH2CHO+OH}R\arabic{reaction}  & OH + CH$_2$CHO   & $\!\!\!\rightarrow$ & H$_2$O + CH$_2$CO &$  2.0\!\times\! 10^{-11} $ & note  \\  
   \refstepcounter{reaction}\label{RO+CH2CHO}R\arabic{reaction}   & O + CH$_2$CHO   & $\!\!\!\rightarrow$ & HCO + H$_2$CO &$  1.0\!\times\! 10^{-10} $ &  note \\  

 \refstepcounter{reaction}R\arabic{reaction}   & C$_2$H$_2$   + OH & $\!\!\!\rightarrow$ &  CH$_2$CO   + H &$  5.0\!\times\! 10^{-17} \left(T/298 \right)^{4.5}e^{ +500/T}$ & Mi89, Wo94\\
\refstepcounter{reaction}R\arabic{reaction}   & CH$_2$CO   + O      & $\!\!\!\rightarrow$ &  CH$_2$ + CO$_2$ &$  3.8\!\times\! 10^{-12} e^{  -680/T}$ & Ba92 \\ % products assumed
\refstepcounter{reaction}R\arabic{reaction}   & CH$_2$CO   + OH    & $\!\!\!\rightarrow$ &  H$_2$COH + CO & $  7.2\!\times\! 10^{-12}  $ & Gr94, Oe92 \\ % Fraction from Gr94, rate from Oe92. 
\refstepcounter{reaction}R\arabic{reaction}   & CH$_2$CO   + OH    & $\!\!\!\rightarrow$ &  CH$_3$ + CO$_2$ &$  5.0\!\times\! 10^{-12}  $ & Fa00, Oe92 \\
\refstepcounter{reaction}R\arabic{reaction}   & CH$_2$CO   + H      & $\!\!\!\rightarrow$ &  CH$_3$ + CO  &$  5.0\!\times\! 10^{-12}  \left(T/298 \right)^{1.45} e^{  -1400/T}$ & Se06a \\
 \refstepcounter{reaction}R\arabic{reaction}   & CH$_2$CO   + H       + M & $\!\!\!\rightarrow$ &  CH$_2$CHO + M &$  1.0\!\times\! 10^{-31} \left(T/298 \right)^{1.43} e^{  -3050/T}$ & assumed   \\   
            & CH$_2$CO   + H     & $\!\!\!\rightarrow$ &  CH$_2$CHO   &$  1.14\!\times\! 10^{-11} \left(T/298 \right)^{1.43} e^{  -3050/T}$ & Se06a\\
\refstepcounter{reaction}R\arabic{reaction}   & CH$_2$CO   + CH$_2$      & $\!\!\!\rightarrow$ &  C$_2$H$_4$ + CO &$  1.6\!\times\! 10^{-12} \left(T/298 \right)^{1.5} e^{  -4000/T}$ & Sa19 \\ 
\refstepcounter{reaction}R\arabic{reaction}   & CH$_2$CO   + CH$_3$      & $\!\!\!\rightarrow$ &  C$_2$H$_5$ + CO &$  1.0\!\times\! 10^{-13} \left(T/298 \right)^{2.3}  e^{  -5360/T}$ & Sem18 \\ 

\multicolumn{6}{l}{\bf C$_2$H$_3$OH}\\
 \refstepcounter{reaction}\label{RC2H3OH}R\arabic{reaction}   & C$_2$H$_2$OH + H + M & $\!\!\!\rightarrow$ &  C$_2$H$_3$OH  + M &  $  3.0\!\times\! 10^{-31} \left(T/298 \right)^{-4.0} e^{  -600/T} $ &  note\\ 
    & C$_2$H$_2$OH + H   & $\!\!\!\rightarrow$ &  C$_2$H$_3$OH   &$  7.0\!\times\! 10^{-11} \left(T/298 \right)^{-0.7} $ & note, Ba92?\\ %  
  \refstepcounter{reaction}R\arabic{reaction} &  C$_2$H$_3$  +    OH + M &$\!\!\!\rightarrow$ &   C$_2$H$_3$OH + M & $ 3.0\!\times\! 10^{-29}  \left(T/298 \right)^{-0.7} $   &  Kny17 \\     
          & C$_2$H$_3$  +    OH  &$\!\!\!\rightarrow$ &   C$_2$H$_3$OH   & $ 9.0\!\times\! 10^{-11}  \left(T/298 \right)^{-0.7} $    &  Kny17\\  
\refstepcounter{reaction}R\arabic{reaction} & C$_2$H$_3$OH  + H   &$\!\!\!\rightarrow$ & CH$_2$CHO  + H$_2$  & $ 3.0\!\times\! 10^{-13}  \left(T/298 \right)^{3.0} e^{-3600/T} $ & Ra11  \\  
\refstepcounter{reaction}R\arabic{reaction} & C$_2$H$_3$OH  + O   &$\!\!\!\rightarrow$ & CH$_2$CHO  + OH  & $ 3.0\!\times\! 10^{-13}  \left(T/298 \right)^{3.0} e^{-3600/T} $ & note \\  
\refstepcounter{reaction}\label{RC2H3OH+O}R\arabic{reaction} & C$_2$H$_3$OH  +  O   &$\!\!\!\rightarrow$ &     H$_2$CO  +   H$_2$CO    & $ 3.0\!\times\! 10^{-13}  \left(T/298 \right)^{3.0} e^{-3600/T} $ & note \\  
\refstepcounter{reaction}\label{RC2H3OH+OH}R\arabic{reaction} & C$_2$H$_3$OH + OH  &$\!\!\!\rightarrow$ &  C$_2$H$_2$OH  + H$_2$O  & $ 1.4\!\times\! 10^{-13}  \left(T/298 \right)^{1.5} e^{+204/T}$ & note \\  

\multicolumn{6}{l}{\bf CH$_3$CO}\\
\refstepcounter{reaction}R\arabic{reaction} & C$_2$H$_3$  + OH   &$\!\!\!\rightarrow$ & CH$_3$CO  + H  & $ 2.9\!\times\! 10^{-11}  \left(T/298 \right)^{-1.0}e^{-195/T} $ & Kny17 \\  
\refstepcounter{reaction}R\arabic{reaction} & CH$_2$CHO +  M  &$\!\!\!\rightarrow$ &  CH$_3$CO  +  M    & $ 1.0\!\times\! 10^{-11}  e^{-10000/T}$  & assumed \\ 

\refstepcounter{reaction}R\arabic{reaction} & CH$_3$  +   CO + M &$\!\!\!\rightarrow$ &  CH$_3$CO   + M   &  $ 7.8\!\times\! 10^{-29}  \left(T/298 \right)^{-7.6} e^{-5530/T}$  & Ts86 \\
     & CH$_3$  +   CO  &$\!\!\!\rightarrow$ &  CH$_3$CO   &  $ 8.4\!\times\! 10^{-13}  e^{-3460/T}$  & Ba94 \\
\refstepcounter{reaction}R\arabic{reaction} & CH$_2$CO  + H  + M  &$\!\!\!\rightarrow$ &   CH$_3$CO    + M    &   $1.38\!\times\! 10^{-30}e^{-1320/T}$ & Ya08 \\  
     & CH$_2$CO  + H    &$\!\!\!\rightarrow$ &   CH$_3$CO     &  $3.7\!\times\! 10^{-12} \left(T/298 \right)^{1.61}e^{-1320/T}$ & Se06 \\        
\refstepcounter{reaction}\label{RCH3CO+H}R\arabic{reaction} & CH$_3$CO +  H  &$\!\!\!\rightarrow$ &   CH$_2$CO  + H$_2$  &  $2.0\!\times\! 10^{-11}$ &Bar91, Ohm90\\   
\refstepcounter{reaction}R\arabic{reaction} & CH$_3$CO  + H  &$\!\!\!\rightarrow$ &  CH$_3$  +   HCO   &  $3.5\!\times\! 10^{-11}$ &Bar91, Ohm90\\   
\refstepcounter{reaction}\label{RO+CH3CO}R\arabic{reaction} & CH$_3$CO  + O   &$\!\!\!\rightarrow$ &   CH$_3$  +   CO$_2$  &  $8.0\!\times\! 10^{-11}$ & Ba94 \\  
\refstepcounter{reaction}\label{RCH3CO+O}R\arabic{reaction} & CH$_3$CO +  O   &$\!\!\!\rightarrow$ &  CH$_2$CO +  OH   &  $2.0\!\times\! 10^{-11}$ & Ba94 \\  
\refstepcounter{reaction}R\arabic{reaction} & CH$_3$CO +  OH   &$\!\!\!\rightarrow$ &   CH$_2$CO +  H$_2$O   &  $2.0\!\times\! 10^{-11}$ & Ts86 \\   
\refstepcounter{reaction}\label{RCH3CO+OH}R\arabic{reaction} & CH$_3$CO  + OH  &$\!\!\!\rightarrow$ &    CH$_4$   +  CO$_2$     &  $2.0\!\times\! 10^{-11}$ & note \\  
\refstepcounter{reaction}R\arabic{reaction} & CH$_3$CO  + CH$_3$  &$\!\!\!\rightarrow$ &   CH$_4$  +   CH$_2$CO   &  $1.0\!\times\! 10^{-11}$ & assumed \\    
\refstepcounter{reaction}R\arabic{reaction} & CH$_3$O  +  CH$_3$CO &$\!\!\!\rightarrow$ & CH$_3$OH +  CH$_2$CO   &  $1.0\!\times\! 10^{-11}$ & assumed \\  

\multicolumn{6}{l}{\bf CH$_3$CHO}\\
\refstepcounter{reaction}R\label{RC2H3OH+M}\arabic{reaction} & C$_2$H$_3$OH +  M    &$\!\!\!\rightarrow$ &    CH$_3$CHO  + M    & $ 1.0\!\times\! 10^{-11}  e^{-10000/T}$  & note \\   
\refstepcounter{reaction}R\arabic{reaction} & CH$_3$CO  + H  + M &$\!\!\!\rightarrow$ &  CH$_3$CHO   + M  &  $2.0\!\times\! 10^{-29}$ & assumed\\
     & CH$_3$CO  + H  &$\!\!\!\rightarrow$ &  CH$_3$CHO    &  $1.5\!\times\! 10^{-10} \left(T/298 \right)^{0.16}$ & Kny17\\ 
\refstepcounter{reaction}R\arabic{reaction} & CH$_2$CHO  + H + M   &$\!\!\!\rightarrow$ &    CH$_3$CHO  + M    &   $1.3\!\times\! 10^{-29}e^{+60/T}$ & assumed \\  
     & CH$_2$CHO  + H   &$\!\!\!\rightarrow$ &    CH$_3$CHO    &   $1.3\!\times\!10^{-10}  \left(T/298 \right)^{0.18}e^{+60/T}$ &Kny17 \\  
\refstepcounter{reaction}R\arabic{reaction} & CH$_3$   +  HCO + M &$\!\!\!\rightarrow$ &   CH$_3$CHO   + M   &  $4.4\!\times\! 10^{-29}$ & assumed \\  
     & CH$_3$   +  HCO  &$\!\!\!\rightarrow$ &   CH$_3$CHO   &  $4.4\!\times\! 10^{-11}$ & Kny17 \\   
\refstepcounter{reaction}R\arabic{reaction} & C$_2$H$_5$   + O   &$\!\!\!\rightarrow$ &    CH$_3$CHO + H   &  $1.3\!\times\! 10^{-10}$ & Ts86\\   
\refstepcounter{reaction}R\arabic{reaction} & H$_2$CO  +  CH$_3$CO  &$\!\!\!\rightarrow$ &  CH$_3$CHO + HCO   &   $3.0\!\times\! 10^{-13}e^{-6500/T}$ & Ts86 \\ 
 
\refstepcounter{reaction}R\arabic{reaction} & HCO  +   CH$_3$CO  &$\!\!\!\rightarrow$ & CH$_3$CHO + CO   &  $1.5\!\times\! 10^{-11}$ & Ts86\\ 
\refstepcounter{reaction}R\arabic{reaction} & CH$_3$O +   CH$_3$CO &$\!\!\!\rightarrow$ &  CH$_3$CHO + H$_2$CO  & $  1.0\!\times\! 10^{-11}$ & Ts86\\   
\refstepcounter{reaction}R\arabic{reaction} & CH$_3$CHO + H    &$\!\!\!\rightarrow$ &   CH$_3$CO +  H$_2$   &  $ 5.27\!\times\! 10^{-13}  \left(T/298 \right)^{2.58} e^{-615/T}$  & Wa84\\ 
\refstepcounter{reaction}R\arabic{reaction} & CH$_3$CHO + H   &$\!\!\!\rightarrow$ &    CH$_2$CHO  + H$_2$   &  $ 6.6\!\times\! 10^{-13}  \left(T/298 \right)^{3.1} e^{-2620/T}$  & Si10\\ 
\refstepcounter{reaction}R\arabic{reaction} & CH$_3$CHO  + OH   &$\!\!\!\rightarrow$ &   CH$_3$CO +  H$_2$O  &   $5.5\!\times\! 10^{-12}e^{+300/T}$ & At01 \\ 
\refstepcounter{reaction}R\arabic{reaction} & CH$_3$CHO  + O    &$\!\!\!\rightarrow$ &   CH$_3$CO  + OH   &   $1.8\!\times\! 10^{-11}e^{-1100/T}$ & De97 \\ 

\multicolumn{6}{l}{\bf C$_2$H$_4$OH}\\
\refstepcounter{reaction}\label{RC2H4OH}R\arabic{reaction}   & C$_2$H$_4$   + OH   + M & $\!\!\!\rightarrow$ &  C$_2$H$_4$OH + M &$  2.1\!\times\! 10^{-29} \left(T/298 \right)^{-4.45}$ & Cl06\\
            & C$_2$H$_4$   + OH     & $\!\!\!\rightarrow$ &  C$_2$H$_4$OH   &$  5.0\!\times\! 10^{-12} e^{  148/T}$ & Cl06\\

\refstepcounter{reaction}R\arabic{reaction}   & CH$_3$ + H$_2$CO  + M & $\!\!\!\rightarrow$ &  C$_2$H$_4$OH + M &$  3.0\!\times\! 10^{-36} \left(T/298 \right)^{4.98}e^{ -564/T}$ & assumed\\ % assumed 
      & CH$_3$ + H$_2$CO  & $\!\!\!\rightarrow$ &  C$_2$H$_4$OH   &$  3.0\!\times\! 10^{-17}\left(T/298 \right)^{4.98} e^{ -564/T}$ & Ch03, Se06b\\
      % old 1868 K barrier used Ch03 rate for ethoxy formation, inserts an 11 kJ/mol activation barrier (Se06b) between ethoxy and C2H4OH
 
\refstepcounter{reaction}\label{RH+C2H4OH}R\arabic{reaction} & C$_2$H$_4$OH  +  H  &$\!\!\!\rightarrow$ &  CH$_3$CHO  +  H$_2$  &  $3.5\!\times\! 10^{-12}\left(T/298 \right)^{1.29}e^{-1740/T}$ &Xu11, note\\ 
\refstepcounter{reaction}R\arabic{reaction} & C$_2$H$_4$OH  +  H   &$\!\!\!\rightarrow$ &  CH$_3$  +  H$_2$COH  &  $3.7\!\times\! 10^{-11}e^{-600/T}$ &Xu11, note\\ 
\refstepcounter{reaction}R\arabic{reaction} & C$_2$H$_4$OH   +   H   &$\!\!\!\rightarrow$ &    C$_2$H$_4$     +   H$_2$O   &   $3.5\!\times\! 10^{-12}e^{-350/T}$ &Xu11, note \\ 
\refstepcounter{reaction}R\arabic{reaction} & C$_2$H$_4$OH  +  H   &$\!\!\!\rightarrow$ &  C$_2$H$_3$OH  +  H$_2$   &   $3.3\!\times\! 10^{-11}e^{-500/T}$ &Xu11, note\\  
\refstepcounter{reaction}R\arabic{reaction} & C$_2$H$_4$OH + O   &$\!\!\!\rightarrow$ &  H$_2$CO   +  H$_2$COH  &   $1.5\!\times\! 10^{-10}$ &Sk18, note\\ 
\refstepcounter{reaction}\label{RO+C2H4OH}R\arabic{reaction} & C$_2$H$_4$OH + O   &$\!\!\!\rightarrow$ &  CH$_3$CHO    +  OH  &   $1.0\!\times\! 10^{-11}e^{-1500/T}$ & note\\
\refstepcounter{reaction}R\arabic{reaction} & C$_2$H$_4$OH + O   &$\!\!\!\rightarrow$ &  C$_2$H$_3$OH    +  OH  &   $3.0\!\times\! 10^{-11}e^{-500/T}$ & note\\ 
\refstepcounter{reaction}\label{ROH+C2H4OH}R\arabic{reaction} & C$_2$H$_4$OH + OH   &$\!\!\!\rightarrow$ &  CH$_3$CHO  +  H$_2$O  &   $1.0\!\times\! 10^{-11}e^{-1500/T}$ & note\\  
\refstepcounter{reaction}R\arabic{reaction} & C$_2$H$_4$OH + OH  &$\!\!\!\rightarrow$ &   C$_2$H$_3$OH    +  H$_2$O  &   $3.0\!\times\! 10^{-11}e^{-500/T}$ & note\\

\multicolumn{6}{l}{\bf C$_2$H$_5$OH}\\
\refstepcounter{reaction}\label{RC2H4OH+H}R\arabic{reaction} & C$_2$H$_4$OH  +  H + M  &$\!\!\!\rightarrow$ & C$_2$H$_5$OH  + M  &  $ 3.0\!\times\! 10^{-31}\left(T/298 \right)^{-4.0}e^{-600/T}$  & Xu11\\
     & C$_2$H$_4$OH   +   H    &$\!\!\!\rightarrow$ &   C$_2$H$_5$OH   &  $ 7.0\!\times\! 10^{-11}  \left(T/298 \right)^{-0.7} $  & Xu11\\   
\refstepcounter{reaction}\label{RC2H5+OH}R\arabic{reaction} & C$_2$H$_5$   +  OH  + M  &$\!\!\!\rightarrow$ &   C$_2$H$_5$OH  + M  &  $ 2.7\!\times\! 10^{-25} $  & Fa93 \\
     & C$_2$H$_5$ + OH     &$\!\!\!\rightarrow$ &   C$_2$H$_5$OH   &  $ 8.25\!\times\! 10^{-11}  \left(T/298 \right)^{-0.61}  e^{+38.4/T} $  & Si10, Fa93\\   
\refstepcounter{reaction}R\arabic{reaction} & CH$_3$   +  H$_2$COH  + M  &$\!\!\!\rightarrow$ &   C$_2$H$_5$OH  + M  &  $ 1.0\!\times\! 10^{-26} $  & assumed \\
     & CH$_3$ + H$_2$COH     &$\!\!\!\rightarrow$ &   C$_2$H$_5$OH   &  $9.44\!\times\! 10^{-11}\left(T/298 \right)^{-1.9}e^{-229/T} $  & Si10 \\   
% \refstepcounter{reaction}R\arabic{reaction} & H      +     C$_2$H$_5$OH &$\!\!\!\rightarrow$ & CH$_3$CHOH   +   H$_2$ & $6.24\!\times\! 10^{-13}\left(T/298 \right)^{2.68}e^{-1470//T}$ & Si10 \\   
\refstepcounter{reaction}R\label{RC2H4OH+H}\arabic{reaction} & H + C$_2$H$_5$OH &$\!\!\!\rightarrow$ & C$_2$H$_4$OH + H$_2$ &  $7.91\!\times\! 10^{-13}\left(T/298 \right)^{2.81}e^{-3780/T}$ & Si10 \\   
\refstepcounter{reaction}R\arabic{reaction} & C$_2$H$_5$OH   +   O    &$\!\!\!\rightarrow$ &   OH     +     C$_2$H$_4$OH   &   $1.6\!\times\! 10^{-13}\left(T/298 \right)^{3.23}e^{-2340/T}$ & Wu07\\ 
\refstepcounter{reaction}R\arabic{reaction} & C$_2$H$_5$OH   +   OH   &$\!\!\!\rightarrow$ &   H$_2$O    +     C$_2$H$_4$OH   &  $ 1.4\!\times\! 10^{-13}  \left(T/298 \right)^{1.5} e^{+204/T}$  & Olm16\\  
\refstepcounter{reaction}R\arabic{reaction} & C$_2$H$_5$OH  +  CH$_3$  &$\!\!\!\rightarrow$ &   C$_2$H$_4$OH  +  CH$_4$  & $ 1.06\!\times\! 10^{-13}\left(T/298 \right)^{3.45} e^{-5540/T}$ & Olm16\\  

\multicolumn{6}{l}{\bf C$_4$H$_n$}\\
 \refstepcounter{reaction}R\arabic{reaction}   & C$_2$H  + C$_2$H$_2$  & $\!\!\!\rightarrow$ &  C$_4$H$_2$   + H   & $1.3\!\times\! 10^{-10} \left(T/298\right)^{ 0.24}e^{+37/T}$ & Ei03\\
\refstepcounter{reaction}\label{R163}R\arabic{reaction}   & C$_2$        + C$_2$H$_2$   &$\!\!\!\rightarrow$ &  C$_4$H  + H & $  3.5\!\times\! 10^{-10} $ &Pa08 \\
 \refstepcounter{reaction}\label{RC4H2}R\arabic{reaction}   & H   + C$_4$H   + M & $\!\!\!\rightarrow$ &  C$_4$H$_2$   + M &$  2.9\!\times\! 10^{-28}\left(T/298 \right)^{-1.5}$ & invented\\
          & H     + C$_4$H  & $\!\!\!\rightarrow$ &  C$_4$H$_2$   &$  1.1\!\times\! 10^{-10} \left(T/298 \right)^{-1.5}$ &  invented \\
 \refstepcounter{reaction}\label{R161}R\arabic{reaction}   & H$_2$        + C$_4$H      &$\!\!\!\rightarrow$ &  C$_4$H$_2$   + H     & $  7.5\!\times\! 10^{-11} e^{ -1560/T}$ & Mo96\\
 \refstepcounter{reaction}\label{R162}R\arabic{reaction}   & C$_4$H$_2$   + C$_2$H      &$\!\!\!\rightarrow$ &  C$_4$H       + C$_2$H$_2$  & $  3.0\!\times\! 10^{-11} e^{ -5000/T}$ & invented\\
 
\refstepcounter{reaction}R\arabic{reaction} & H     +      C$_4$H$_2$ + M &$\!\!\!\rightarrow$ &  C$_4$H$_3$   + M &  $ 5.9\!\times\! 10^{-25}  \left(T/298 \right)^{-8.9} e^{-1260/T}$  & Kl05\\ 
      & H     +      C$_4$H$_2$ + M &$\!\!\!\rightarrow$ &  C$_4$H$_3$   + M   &  $ 5.2\!\times\! 10^{-11}  \left(T/298 \right)^{1.2} e^{-882/T}$  & Kl05 \\  

\refstepcounter{reaction}R\arabic{reaction} & H    +       C$_4$H$_3$  &$\!\!\!\rightarrow$ &  H$_2$    +      C$_4$H$_2$  &   $1.4\!\times\! 10^{-14}\left(T/298 \right)^{3.68}e^{-3140/T}$ & Za17\\   
\refstepcounter{reaction}R\arabic{reaction} & H     +      C$_4$H$_3$ &$\!\!\!\rightarrow$ &   C$_2$H$_2$    +    C$_2$H$_2$  &   $5.0\!\times\! 10^{-14 } \left(T/298 \right)^{2.2}e^{-4560/T}$ & Ha05\\ 

\refstepcounter{reaction}R\arabic{reaction} & H    +       C$_4$H$_3$  + M &$\!\!\!\rightarrow$ &  C$_4$H$_4$   + M    &  $ 1\!\times\! 10^{-27} $  & assumed \\
        & H    +       C$_4$H$_3$  &$\!\!\!\rightarrow$ &  C$_4$H$_4$   &   $1.5\!\times\! 10^{-10} \left(T/298 \right)^{0.12} e^{+53/T}$ & Za17 \\ 

\refstepcounter{reaction}R\arabic{reaction} & O    +       C$_4$H$_3$  &$\!\!\!\rightarrow$ &  C$_4$H$_2$   +     OH  &   $1.0\!\times\! 10^{-11}e^{-900/T}$ & assumed \\  
\refstepcounter{reaction}R\arabic{reaction} & OH    +      C$_4$H$_3$  &$\!\!\!\rightarrow$ &  C$_4$H$_2$    +    H$_2$O   &   $2.0\!\times\! 10^{-11}e^{-500/T}$ & assumed \\

\refstepcounter{reaction}R\arabic{reaction} & H    +       C$_4$H$_4$  &$\!\!\!\rightarrow$ &  C$_4$H$_3$     +   H$_2$    &   $3.0\!\times\! 10^{-12}e^{-2000/T}$ & Gh88\\  
\refstepcounter{reaction}R\arabic{reaction} & O    +       C$_4$H$_4$  &$\!\!\!\rightarrow$ &  C$_4$H$_3$    +    OH       &   $5.0\!\times\! 10^{-11}e^{-900/T}$ &  Cv87\\
\refstepcounter{reaction}R\arabic{reaction} & OH    +      C$_4$H$_4$  &$\!\!\!\rightarrow$ &  C$_4$H$_3$     +   H$_2$O    &   $6.4\!\times\! 10^{-12}e^{+480/T}$ & So15\\


\multicolumn{6}{l}{\bf HCCO}\\
 \refstepcounter{reaction}R\arabic{reaction} & CO  +  CH  + M &$\!\!\!\rightarrow$ &   HCCO  + M   & $ 4.2\!\times\! 10^{-30} \left(T/298 \right)^{-1.9}$  & Fu98\\   
          & CO  +  CH   &$\!\!\!\rightarrow$ &  HCCO      & $ 1.7\!\times\! 10^{-10} \left(T/298 \right)^{-0.4}$  & Fu98\\ 
 \refstepcounter{reaction}R\arabic{reaction} & C$_2$H$_2$  + O   &$\!\!\!\rightarrow$ &    H    +  HCCO    & $ 1.2\!\times\! 10^{-12} \left(T/298 \right)^{2.0}e^{-956/T}$  & Ei03\\  
 \refstepcounter{reaction}R\arabic{reaction}   & CH$_2$CO   + CH$_3$      & $\!\!\!\rightarrow$ &  CH$_4$ + HCCO &$  1.5\!\times\! 10^{-11}  e^{  -7000/T}$ & Se18 \\ 
\refstepcounter{reaction}R\arabic{reaction} & H   +   HCCO  &$\!\!\!\rightarrow$ &  CH$_2$  +  CO      & $ 1.7\!\times\! 10^{-10} $  & Gl00\\
 \refstepcounter{reaction}R\arabic{reaction} & H$_2$  +  HCCO  &$\!\!\!\rightarrow$ &  CH$_2$CO  + H    & $ 2.2\!\times\! 10^{-11} e^{-2000/T}$  & Ca03\\  
 %  products assumed.  This is slightly exothermic
 \refstepcounter{reaction}R\arabic{reaction} & O  +   HCCO  &$\!\!\!\rightarrow$ &  CH  +  CO$_2$     & $ 4.9\!\times\! 10^{-11} e^{-561/T}$  & Pe95\\  
 \refstepcounter{reaction}R\arabic{reaction} & O  +  HCCO  &$\!\!\!\rightarrow$ &  HCO +   CO           & $ 1.1\!\times\! 10^{-10} $  & Ba92\\ % products are substituted for CO + CO + H
 \refstepcounter{reaction}R\arabic{reaction} & CH$_3$  +  HCCO  &$\!\!\!\rightarrow$ &  C$_2$H$_4$ +   CO    & $ 3.3\!\times\! 10^{-12} $  & Hi96\\
 \refstepcounter{reaction}R\arabic{reaction} & C$_2$H +  O$_2$   &$\!\!\!\rightarrow$ &   O  +   HCCO      & $ 1.0\!\times\! 10^{-12} $  & Ts86\\

\multicolumn{6}{l}{\bf C$_3$H$_4$\label{C3H4}}\\
\refstepcounter{reaction}R\arabic{reaction} & C$_2$H$_2$   +     CH$_2$  + M &$\!\!\!\rightarrow$ &   C$_3$H$_4$  + M   &   $2.0\!\times\! 10^{-28} e^{-3330/T}$ & assumed\\  
     & C$_2$H$_2$   +     CH$_2$  &$\!\!\!\rightarrow$ &   C$_3$H$_4$   &   $2.0\!\times\! 10^{-11} e^{-3330/T}$ & Ba94 \\   
\refstepcounter{reaction}R\arabic{reaction} & CH$_2$CO  +  C$_2$H$_2$   &$\!\!\!\rightarrow$ &    C$_3$H$_4$  +   CO   &   $2.0\!\times\! 10^{-11} e^{-12000/T}$ & assumed \\    

\refstepcounter{reaction}R\arabic{reaction} & C$_3$H$_4$    +    O   &$\!\!\!\rightarrow$ &    HCO    +     C$_2$H$_3$ &   $3.7\!\times\! 10^{-12} e^{-1950/T}$ & Va16\\  
\refstepcounter{reaction}R\arabic{reaction} & C$_3$H$_4$    +    O   &$\!\!\!\rightarrow$ &    CH$_3$    +     HCCO   &   $3.4\!\times\! 10^{-12}  \left(T/298 \right)^{0.9}e^{-2170/T}$ & Va16 \\    
\refstepcounter{reaction}R\arabic{reaction} & C$_3$H$_4$    +    O    &$\!\!\!\rightarrow$ &   C$_2$H$_4$    +    CO   &   $1.7\!\times\! 10^{-11} \left(T/298 \right)^{-0.57} e^{-2220/T}$ & Va16 \\   
\refstepcounter{reaction}R\arabic{reaction} & C$_3$H$_4$   +    H    &$\!\!\!\rightarrow$ &   C$_2$H$_2$    +    CH$_3$   &   $2.5\!\times\! 10^{-11} \left(T/298 \right)^{1.7} e^{-4130/T}$ & Hi89\\  
%\refstepcounter{reaction}R\arabic{reaction} & C$_3$H$_4$   +    H    &$\!\!\!\rightarrow$ &   C$_2$H$_2$    +    CH$_3$   &   $1.6\!\times\! 10^{-11} \left(T/298 \right)^{1.8} e^{-4000/T}$ & NIST fit\\  

\multicolumn{6}{l}{\bf C$_3$H$_6$}\\
\refstepcounter{reaction}R\arabic{reaction} & CH$_3$   +   C$_2$H$_3$  + M &$\!\!\!\rightarrow$ &    C$_3$H$_6$   + M      &   $1.0\!\times\! 10^{-29} $ & assumed \\  
        & CH$_3$   +   C$_2$H$_3$   &$\!\!\!\rightarrow$ &    C$_3$H$_6$        &   $1.2\!\times\! 10^{-10} $ & St00\\ 
\refstepcounter{reaction}R\arabic{reaction} & C$_2$H$_4$   +  CH$_2$  + M &$\!\!\!\rightarrow$ &     C$_3$H$_6$   + M     &   $9.2\!\times\! 10^{-31} e^{+530/T}$ & Po13\\   
         & C$_2$H$_4$   +  CH$_2$  &$\!\!\!\rightarrow$ &     C$_3$H$_6$    &   $9.2\!\times\! 10^{-12} e^{+530/T}$ & Po13\\  
\refstepcounter{reaction}R\arabic{reaction} & CH$_2$CO  + C$_2$H$_4$   &$\!\!\!\rightarrow$ &    C$_3$H$_6$   +  CO   &   $2.0\!\times\! 10^{-11}  e^{-12000/T}$ & assumed\\   
\refstepcounter{reaction}R\arabic{reaction} & C$_3$H$_6$   +  H     &$\!\!\!\rightarrow$ &     C$_2$H$_4$   +  CH$_3$   &   $2.1\!\times\! 10^{-12} \left(T/298 \right)^{1.5} e^{-1010/T}$ & NIST fit\\  
\refstepcounter{reaction}R\arabic{reaction} & C$_3$H$_6$   +  O    &$\!\!\!\rightarrow$ &      HCO    +  C$_2$H$_5$    &   $8.0\!\times\! 10^{-14} \left(T/298 \right)^{2.2} e^{+457/T}$ & Cv87\\   

\refstepcounter{reaction}R\arabic{reaction} & C$_3$H$_6$   +  O    &$\!\!\!\rightarrow$ &      CH$_2$CHO  + CH$_3$   &   $3.0\!\times\! 10^{-13} \left(T/298 \right)^{2.2} e^{+457/T} $ & Cv87 \\   
\refstepcounter{reaction}R\arabic{reaction} & C$_3$H$_6$   +  O     &$\!\!\!\rightarrow$ &     CH$_3$CO  +  CH$_3$   &   $1.7\!\times\! 10^{-13} \left(T/298 \right)^{2.2} e^{+457/T}$ & Cv87 \\   
\refstepcounter{reaction}R\arabic{reaction} & C$_3$H$_6$   +  O    &$\!\!\!\rightarrow$ &      H$_2$CO   +  C$_2$H$_4$    &   $4.0\!\times\! 10^{-13} \left(T/298 \right)^{2.2} e^{+457/T}$ & Cv87 \\ 



\multicolumn{6}{l}{\bf N$_2$,NO}\\
 \refstepcounter{reaction}R\arabic{reaction}   & N            + N  + M       &$\!\!\!\rightarrow$&  N$_2$        + M &$  1.3\!\times\! 10^{-32}$ & Kn88\\
          & N            + N           &$\!\!\!\rightarrow$&  N$_2$         &$  1.0\!\times\! 10^{-12}$ & note  \\
 \refstepcounter{reaction}R\arabic{reaction}   & N            + O$_2$       & $\!\!\!\rightarrow$ &  NO           + O     & $  4.5\!\times\! 10^{-12} \left(T/298\right)^{ 1.00}e^{ -3270/T}$ & Ba94\\
 \refstepcounter{reaction}R\arabic{reaction}  & N      + NO &$\!\!\!\rightarrow$ &  N$_2$  + O & $  3.1\!\times\! 10^{-11}$ & At89\\
 \refstepcounter{reaction}R\arabic{reaction}   & N    + OH     &$\!\!\!\rightarrow$ &  NO    + H  & $  3.8\!\times\! 10^{-11} e^{   -85/T}$ & At89\\
%  \refstepcounter{reaction}R\arabic{reaction}  & N     + CO$_2$      &$\!\!\!\rightarrow$ &  NO           + CO   & $  0 $ & Fe98\\
 \refstepcounter{reaction}R\arabic{reaction}   & N   + O + M   &$\!\!\!\rightarrow$&  NO  + M &$  5.5\!\times\! 10^{-33}e^{ +155/T}$ &  Ca73\\
          & N     + O        &$\!\!\!\rightarrow$&  NO      &$  1.0\!\times\! 10^{-12} $ &  assumed \\
 \refstepcounter{reaction}R\arabic{reaction}   & C            + NO          & $\!\!\!\rightarrow$ &  CO           + N  & $  4.7\!\times\! 10^{-11}$ & De91\\

\multicolumn{6}{l}{\bf NH}\\
 \refstepcounter{reaction}R\arabic{reaction}   & N            + H + M           &$\!\!\!\rightarrow$&  NH  + M &$  4.3\!\times\! 10^{-32}\left(T/298\right)^{-1.0} $ &  O+H\\
          & N            + H        &$\!\!\!\rightarrow$&  NH      &$  1.0\!\times\! 10^{-12} $ &  note\\
 \refstepcounter{reaction}R\arabic{reaction}  & O            + NH          &$\!\!\!\rightarrow$ &  OH           + N               & $  1.2\!\times\! 10^{-11}$ & Co91\\
 \refstepcounter{reaction}R\arabic{reaction} & O            + NH          &$\!\!\!\rightarrow$ &  NO           + H                & $  1.2\!\times\! 10^{-10}$ & Co91 \\
 \refstepcounter{reaction}R\arabic{reaction}  & H            + NH          &$\!\!\!\rightarrow$ &  N            + H$_2$         & $  3.2\!\times\! 10^{-12}$ & Ad05\\
 \refstepcounter{reaction}R\arabic{reaction}   & N            + NH          & $\!\!\!\rightarrow$ &  N$_2$        + H           & $  2.0\!\times\! 10^{-11} \left(T/298\right)^{ 0.51}e^{    -10/T}$ & Cd05\\
 \refstepcounter{reaction}R\arabic{reaction}   & NH           + OH          & $\!\!\!\rightarrow$ &  H$_2$O       + N        & $  3.1\!\times\! 10^{-12} \left(T/298 \right)^{ 1.20}$ & Co91\\
 \refstepcounter{reaction}\label{R191}R\arabic{reaction}  & NH           + OH          &$\!\!\!\rightarrow$ &  NO           + H$_2$      & $  3.3\!\times\! 10^{-11}$ & Co91\\
 \refstepcounter{reaction}R\arabic{reaction}  & NH           + NO          &$\!\!\!\rightarrow$ &  N$_2$        + OH                                      & $  1.0\!\times\! 10^{-11}$ & Ba94 \\
 \refstepcounter{reaction}R\arabic{reaction}   & NH    + O$_2$    &$\!\!\!\rightarrow$ &  NO    + OH  & $  6.7\!\times\! 10^{-14} \left(T/298 \right)^{0.79} e^{  -600/T}$ & Rom96\\
 \refstepcounter{reaction}\label{RNO+CH}R\arabic{reaction}   & CH           + NO          &$\!\!\!\rightarrow$ &  NH           + CO                                      & $  1.0\!\times\! 10^{-10} $ & Ge99, note\\
 \refstepcounter{reaction}R\arabic{reaction}   & N            + CH$_2$      &$\!\!\!\rightarrow$ &  NH           + CH                                      & $  1.0\!\times\! 10^{-12} e^{-20400/T}$ & Mi97\\
 \refstepcounter{reaction}\label{R196}R\arabic{reaction}  & NH           + CH$_2$      &$\!\!\!\rightarrow$ &  N            + CH$_3$                                  & $  3.0\!\times\! 10^{-11}$ & Mo96\\
 \refstepcounter{reaction}\label{R197}R\arabic{reaction}  & NH           + CH$_3$      &$\!\!\!\rightarrow$ &  N            + CH$_4$                                  & $  1.0\!\times\! 10^{-11}$ & Mo96\\
% R325  & N            + C$_2$H$_3$  &$\!\!\!\rightarrow$ &  CH$_3$CN          + H                                  & $  0$ & note\\
 \refstepcounter{reaction}\label{R198}R\arabic{reaction}  & N  + C$_2$H$_3$  &$\!\!\!\rightarrow$ &  NH + C$_2$H$_2$  & $  1.2\!\times\! 10^{-11}$ & Pa96\\
 \refstepcounter{reaction}\label{R199}R\arabic{reaction}  & N            + C$_2$H$_5$  &$\!\!\!\rightarrow$ &  NH           + C$_2$H$_4$                              & $  5.5\!\times\! 10^{-11}$ &  St95, Ya05\\
 \refstepcounter{reaction}\label{R200}R\arabic{reaction}  & NH           + C$_2$H$_3$  &$\!\!\!\rightarrow$ &  N            + C$_2$H$_4$                              & $  1.0\!\times\! 10^{-11}$ & Mo96\\
 \refstepcounter{reaction}\label{R201}R\arabic{reaction}  & NH           + C$_2$H$_5$  &$\!\!\!\rightarrow$ &  N            + C$_2$H$_6$                              & $  1.0\!\times\! 10^{-11}$ & Mo96\\

\multicolumn{6}{l}{\bf NH$_2$}\\
 \refstepcounter{reaction}\label{RNH2}R\arabic{reaction} & NH + H + M &$\!\!\!\rightarrow$& NH$_2$  + M &$ 1.0\!\times\! 10^{-32} \left(T/298\right)^{ -2.0} $ & assume \\
          & NH  + H   &$\!\!\!\rightarrow$&  NH$_2$  &$  1.0\!\times\! 10^{-12} $ & assume \\  % just assume something reasonable
 \refstepcounter{reaction}R\arabic{reaction}   & NH           + NH          & $\!\!\!\rightarrow$ &  NH$_2$       + N      & $  3.75\!\times\! 10^{-15} \left(T/298\right)^{ 3.88}e^{-170/T}$ & Kl09\\
  \refstepcounter{reaction}R\arabic{reaction}   & H            + NH$_2$      &$\!\!\!\rightarrow$ &  NH           + H$_2$     & $  1.0\!\times\! 10^{-10} e^{ -4450/T}$ & Ro94\\
 \refstepcounter{reaction}R\arabic{reaction}  & O            + NH$_2$      &$\!\!\!\rightarrow$ &  NH           + OH           & $  1.2\!\times\! 10^{-11}$ & Co91\\
 \refstepcounter{reaction}R\arabic{reaction}  & O            + NH$_2$      &$\!\!\!\rightarrow$ &  NO           + H$_2$      & $  8.2\!\times\! 10^{-11}$ & Co91\\
  \refstepcounter{reaction}R\arabic{reaction}   & OH     + NH$_2$      & $\!\!\!\rightarrow$ &  NH    + H$_2$O   & $  7.8\!\times\! 10^{-13} \left(T/298\right)^{ 1.50}e^{   230/T}$ & Co91\\
\refstepcounter{reaction}R\arabic{reaction}   & NH$_2$   + NO    & $\!\!\!\rightarrow$ &  N$_2$    + H$_2$O     & $ 6.0\!\times\! 10^{-11} \left(T/298 \right)^{-2.37}e^{ -437/T}$ & So01\\
 
  \refstepcounter{reaction}\label{R209}R\arabic{reaction}  & NH$_2$   + C$_2$H   &$\!\!\!\rightarrow$ &  NH   + C$_2$H$_2$     & $  4.0\!\times\! 10^{-11}$ & Mo96\\
  \refstepcounter{reaction}\label{R210}R\arabic{reaction}  & NH  + C$_2$H$_3$  &$\!\!\!\rightarrow$ &  NH$_2$   + C$_2$H$_2$  & $  3.0\!\times\! 10^{-11}$ & Mo96\\
  \refstepcounter{reaction}\label{R211}R\arabic{reaction}  & NH + C$_2$H$_5$  &$\!\!\!\rightarrow$ &  NH$_2$   + C$_2$H$_4$    & $  3.0\!\times\! 10^{-11}$ & Mo96\\
  \refstepcounter{reaction}\label{R212}R\arabic{reaction}  & NH  + C$_2$H$_6$  &$\!\!\!\rightarrow$ &  NH$_2$   + C$_2$H$_5$     & $  1.16\!\times\! 10^{-10}e^{ -8420/T}$ & RH94, rev-Xu99\\


\multicolumn{6}{l}{\bf NH$_3$}\\
\refstepcounter{reaction}R\arabic{reaction}   & NH$_2$       + H    + M &$\!\!\!\rightarrow$&  NH$_3$       + M &$  4.4\!\times\! 10^{-30} \left(T/298 \right)^{-1.7}$ & Alt15\\
             & NH$_2$       + H           &$\!\!\!\rightarrow$&  NH$_3$   &$  2.7\!\times\! 10^{-11}$ & Pa79\\
  \refstepcounter{reaction}R\arabic{reaction}   & NH    + H$_2$ +M   &$\!\!\!\rightarrow$&  NH$_3$       + M &$  0.0 $ & assume \\
              & NH      + H$_2$          &$\!\!\!\rightarrow$&  NH$_3$   &$  0.0 $ & assume \\
 \refstepcounter{reaction}R\arabic{reaction}   & H$_2$    + NH$_2$      & $\!\!\!\rightarrow$ &  NH$_3$       + H    & $  2.7\!\times\! 10^{-14} \left(T/298\right)^{ 2.83}e^{ -3640/T}$ & Co97\\
 \refstepcounter{reaction}R\arabic{reaction}   & OH    + NH$_2$      & $\!\!\!\rightarrow$ &  NH$_3$       + O      & $  3.3\!\times\! 10^{-13} \left(T/298\right)^{ 0.41}e^{  -250/T}$ & Ba92\\
 \refstepcounter{reaction}R\arabic{reaction}   & OH   + NH$_3$      & $\!\!\!\rightarrow$ &  NH$_2$    + H$_2$O   & $  7.6\!\times\! 10^{-12} \left(T/298\right)^{ 1.60}e^{  -480/T}$ & Co91\\
 \refstepcounter{reaction}R\arabic{reaction}   & NH$_2$  + HCO  & $\!\!\!\rightarrow$ &  NH$_3$     + CO & $  5.0\!\times\! 10^{-11}$  & assumed\\
 \refstepcounter{reaction}R\arabic{reaction}   & NH$_2$  + CH$_4$  & $\!\!\!\rightarrow$ &  NH$_3$   + CH$_3$ & $  6.8\!\times\! 10^{-14} \left(T/298\right)^{ 3.01}e^{ -5000/T}$ & So03\\
  \refstepcounter{reaction}\label{RNH3+C2H}R\arabic{reaction}   & NH$_3$     + C$_2$H    & $\!\!\!\rightarrow$ &  NH$_2$   + C$_2$H$_2$   & $  4.0\!\times\! 10^{-11} \left(T/298 \right)^{-0.80}$ & Car04\\
\refstepcounter{reaction}\label{R221}R\arabic{reaction}  & NH$_2$       + C$_2$H$_3$  &$\!\!\!\rightarrow$ &  NH$_3$       + C$_2$H$_2$     & $  4.0\!\times\! 10^{-11}$ & Mo96\\
 \refstepcounter{reaction}\label{RNH2+C2H4}R\arabic{reaction}   & NH$_2$       + C$_2$H$_4$  &$\!\!\!\rightarrow$ &  NH$_3$       + C$_2$H$_3$  & $  8.8\!\times\! 10^{-12} e^{ -5170/T}$ & He95\\
 \refstepcounter{reaction}\label{R223}R\arabic{reaction}  & NH$_2$       + C$_2$H$_5$  &$\!\!\!\rightarrow$ &  NH$_3$       + C$_2$H$_4$     & $  4.1\!\times\! 10^{-11}$ & De82\\
 \refstepcounter{reaction}\label{R224}R\arabic{reaction}   & NH$_2$       + C$_2$H$_6$  &$\!\!\!\rightarrow$ &  NH$_3$       + C$_2$H$_5$    & $  1.6\!\times\! 10^{-11} e^{ -5560/T}$ & He95\\


\multicolumn{6}{l}{\bf CN}\\
 \refstepcounter{reaction}R\arabic{reaction}  & N            + CH          &$\!\!\!\rightarrow$ &  CN           + H                                       & $  2.1\!\times\! 10^{-11}$ & Me81\\
 \refstepcounter{reaction}R\arabic{reaction} & C            + NH          &$\!\!\!\rightarrow$ &  CN           + H                                       & $  7.0\!\times\! 10^{-11}$ & Mi97\\
 \refstepcounter{reaction}R\arabic{reaction}  & CN           + NO          &$\!\!\!\rightarrow$ &  N$_2$        + CO     & $  1.8\!\times\! 10^{-10}e^{ -4040/T}$ & Mu75\\
 \refstepcounter{reaction}R\arabic{reaction}   & CN     + O$_2$       &$\!\!\!\rightarrow$ &  CO           + NO       & $  3.6\!\times\! 10^{-12} e^{   210/T}$ & Ba94,Ri99\\
 \refstepcounter{reaction}R\arabic{reaction}  & N            + CN          &$\!\!\!\rightarrow$ &  N$_2$        + C                                       & $  1.0\!\times\! 10^{-10}$ & Wh83\\
 \refstepcounter{reaction}R\arabic{reaction}  & C            + NO          &$\!\!\!\rightarrow$ &  CN           + O                                       & $  3.3\!\times\! 10^{-11}$ & De91\\
  \refstepcounter{reaction}R\arabic{reaction}  & O            + CN          &$\!\!\!\rightarrow$ &  CO           + N                                       & $  1.7\!\times\! 10^{-11}$ & Ba92\\

\multicolumn{6}{l}{\bf HCN}\\
% \refstepcounter{reaction}R\arabic{reaction}   & H    + CN +M   &$\!\!\!\rightarrow$&  HCN   + M &$  3.2\!\times\! 10^{-30}\left(T/298\right)^{ -1.79}$ & \\
%             & H      + CN   &$\!\!\!\rightarrow$&  HCN   &$  1.0\!\times\! 10^{-10}$ & \\
 \refstepcounter{reaction}R\arabic{reaction}   & H    + CN +M   &$\!\!\!\rightarrow$&  HCN   + M &$  8.8\!\times\! 10^{-30}\left(T/298\right)^{ -2.2} e^{ -567/T} $ & Ts92 UPDATE \\
             & H      + CN   &$\!\!\!\rightarrow$&  HCN   & $  1.7\!\times\! 10^{-10} \left(T/298\right)^{ -0.5}$ & Ts92 UPDATE \\
 \refstepcounter{reaction}R\arabic{reaction}   & H$_2$    + CN   & $\!\!\!\rightarrow$ &  HCN       + H        & $  5.7\!\times\! 10^{-13} \left(T/298\right)^{ 2.45}e^{ -1130/T}$ & Wo96b\\
 
  \refstepcounter{reaction}R\arabic{reaction}   & CH + N$_2$  & $\!\!\!\rightarrow$ &  N    + HCN  & $  5.6\!\times\! 10^{-13} \left(T/298\right)^{ 0.88}e^{-10100/T}$ & Ro96\\
 
 \refstepcounter{reaction}R\arabic{reaction}  & N     + CH$_2$      &$\!\!\!\rightarrow$ &  HCN        + H      & $  1.7\!\times\! 10^{-11}$ & Ts90\\
 \refstepcounter{reaction}R\arabic{reaction}   & N    + CH$_3$      &$\!\!\!\rightarrow$ &  HCN        + H$_2$       & $  4.3\!\times\! 10^{-11} e^{  -420/T}$ & St88, Ma89\\
  \refstepcounter{reaction}R\arabic{reaction} & NO  +  HCCO  &$\!\!\!\rightarrow$ &  HCN  +  CO$_2$     & $ 8.6\!\times\! 10^{-12} \left(T/298 \right)^{-0.75}e^{90.2/T}$  & Mi03 \\ 
\refstepcounter{reaction}R\arabic{reaction}   & O    + HCN         & $\!\!\!\rightarrow$ &  NH           + CO    & $  8.8\!\times\! 10^{-13} \left(T/298\right)^{ 1.21}e^{ -3850/T}$ & Ts91\\

% R196   & OH    + CN      & $\!\!\!\rightarrow$ &  HCN          + O    & $  4.2\!\times\! 10^{-13} \left(T/298\right)^{ 2.57}e^{ -2000/T}$ & revPe85\\
 \refstepcounter{reaction}R\arabic{reaction}   & O + HCN  & $\!\!\!\rightarrow$ &  OH    + CN  & $  3.6\!\times\! 10^{-11} \left(T/298\right)^{ 1.58}e^{ -13400/T}$ & Pe85\\
 % \refstepcounter{reaction}R\arabic{reaction}   & O + HCN  & $\!\!\!\rightarrow$ &  OH    + CN  & $  8.2\!\times\! 10^{-12} \left(T/298\right)^{ 1.82}e^{ -14000/T}$ & Li20\\


% CN + OH > NCO


% R197   & H$_2$O  + CN  &$\!\!\!\rightarrow$ &  HCN  + OH & $  5.1\!\times\! 10^{-14}\left(T/298\right)^{ 2.54} e^{ -2660/T}$ & revWo95\\
 \refstepcounter{reaction}R\arabic{reaction}   & HCN  + OH &$\!\!\!\rightarrow$ &  H$_2$O  + CN   & $  2.2\!\times\! 10^{-13}\left(T/298\right)^{ 1.83} e^{ -5180/T}$ & Wo95\\

\refstepcounter{reaction}R\arabic{reaction}   & HCN      + OH       &$\!\!\!\rightarrow$ &  CO   + NH$_2$      & $  1.1\!\times\! 10^{-13} e^{ -5890/T}$ & Mi88\\
  \refstepcounter{reaction}R\arabic{reaction}   & CN    + CH$_2$      &$\!\!\!\rightarrow$ &  CH  + HCN     & $  1.0\!\times\! 10^{-11} e^{ -1000/T}$ & assumed\\
 \refstepcounter{reaction}R\arabic{reaction}   & CN  + CH$_3$      &$\!\!\!\rightarrow$ &  HCN + CH$_2$   & $  1.0\!\times\! 10^{-11} e^{ -1000/T}$ & assumed\\
 \refstepcounter{reaction}R\arabic{reaction}   & CN   + CH$_4$      & $\!\!\!\rightarrow$ &  HCN + CH$_3$  & $  3.4\!\times\! 10^{-13} \left(T/298\right)^{ 2.64}e^{ +220/T}$ & Ba91\\

 \refstepcounter{reaction}\label{RCH+NO}R\arabic{reaction} & CH  +   NO & $\!\!\!\rightarrow$ & O    +  HCN   &$ 1.0\!\times\! 10^{-10} $ & Be98, note\\
 \refstepcounter{reaction}R\arabic{reaction}   & N  + C$_2$H$_4$  &$\!\!\!\rightarrow$ &  HCN  + CH$_3$      & $  3.3\!\times\! 10^{-14} e^{  -352/T}$ & Ke72\\
  \refstepcounter{reaction}\label{RCN+C2H4}R\arabic{reaction}  & CN   + C$_2$H$_4$  &$\!\!\!\rightarrow$ &  HCN  + C$_2$H$_3$     & $  2.1\!\times\! 10^{-10}$ & Ga07\\
 \refstepcounter{reaction}R\arabic{reaction}   & CN    + C$_2$H$_6$  & $\!\!\!\rightarrow$ &  HCN  + C$_2$H$_5$ & $  1.4\!\times\! 10^{-12} \left(T/298\right)^{ 2.77}e^{ 900/T}$ & Ba91\\
\refstepcounter{reaction}R\arabic{reaction}  & NH$_3$   + CN   &$\!\!\!\rightarrow$ &  NH$_2$    + HCN   & $  2.9\!\times\! 10^{-11}$ & Me93 \\

\multicolumn{6}{l}{{\bf NCHOH} and {\bf HOCN}}\\
%\multicolumn{6}{l}{\bf HOCN}\\
 \refstepcounter{reaction}\label{RNCHOH}R\arabic{reaction}   & HCN          + OH  + M & $\!\!\!\rightarrow$ &  NCHOH        + M &$  5.0\!\times\! 10^{-31} \left(T/298 \right)^{-3.0}e^{  -1700/T}$ & Bu14\\
             & HCN          + OH       & $\!\!\!\rightarrow$ &  NCHOH      &$  2.5\!\times\! 10^{-12}\left(T/298 \right)^{-3.0} e^{  -1700/T}$ & Bu14\\
 \refstepcounter{reaction}R\arabic{reaction}  & NCHOH   + H      &$\!\!\!\rightarrow$ &  HOCN   +  H$_2$      & $  1.0\!\times\! 10^{-10}e^{ -500/T}$ & assume\\ 
 \refstepcounter{reaction}R\arabic{reaction}  & NCHOH  + OH      &$\!\!\!\rightarrow$ &  HOCN  +  H$_2$O      & $  2.0\!\times\! 10^{-11}e^{ -200/T}$ & assume \\ 
 \refstepcounter{reaction}R\arabic{reaction}  & NCHOH   + O         &$\!\!\!\rightarrow$ &  HOCN  +  OH          & $  5.0\!\times\! 10^{-11}e^{ -500/T}$ &assume \\ 
 \refstepcounter{reaction}\label{RHOCN+H}R\arabic{reaction}  & HOCN    + H   &$\!\!\!\rightarrow$ &  CN  +  H$_2$O    & $  5.0\!\times\! 10^{-12}e^{ -2500/T}$ &Sz84, note\\ 

\multicolumn{6}{l}{\bf NCO}\\
 \refstepcounter{reaction}\label{RHOCN}R\arabic{reaction}  & HOCN  + H   &$\!\!\!\rightarrow$ &  NCO +  H$_2$   & $  5.0\!\times\! 10^{-12}e^{ -2500/T}$ & Sz84, note\\ 
 \refstepcounter{reaction}\label{RHOCN+O}R\arabic{reaction}  &HOCN  + O  &$\!\!\!\rightarrow$ &    NCO  +  OH   & $  1.5\!\times\! 10^{-11}\left(T/298 \right)^{2.0}e^{ -2500/T}$ & Ca01, note \\ 
 \refstepcounter{reaction}\label{RHOCN+OH}R\arabic{reaction}  &HOCN   + OH   &$\!\!\!\rightarrow$ &  NCO   +  H$_2$O   & $  1.0\!\times\! 10^{-13}\left(T/298 \right)^{2.0}e^{ -500/T}$ & note \\ 
 
\refstepcounter{reaction}R\arabic{reaction} & O  +  HCN  &$\!\!\!\rightarrow$ &   H  +   NCO  & $ 1.43\!\times\! 10^{-12} \left(T/298 \right)^{1.47}e^{-3780/T}$  & Lo84\\   
 \refstepcounter{reaction}R\arabic{reaction} & O   +  NCO  &$\!\!\!\rightarrow$ &  CO  +    NO     & $ 7.5\!\times\! 10^{-11} $  & Ts92\\   
 \refstepcounter{reaction}R\arabic{reaction} & CN  +   OH  &$\!\!\!\rightarrow$ &   H    +   NCO    & $ 7.0\!\times\! 10^{-11} $  & Ts92\\   
 \refstepcounter{reaction}R\arabic{reaction} & CN   +   O$_2$  &$\!\!\!\rightarrow$ &   O    +   NCO   & $ 1.2\!\times\! 10^{-11}  e^{210/T}$  & Ba94\\  
 \refstepcounter{reaction}R\arabic{reaction} & H   +    NCO  &$\!\!\!\rightarrow$ &  CO  +    NH  & $ 2.2\!\times\! 10^{-11} \left(T/298 \right)^{0.9}$  & Be00, Ts92\\  % fit to measured rates at 300, 550, and 1500 K.
 \refstepcounter{reaction}R\arabic{reaction} & N   +    NCO  &$\!\!\!\rightarrow$ &  N$_2$   +   CO    & $ 3.3\!\times\! 10^{-11} $  & Ba92 \\    
 \refstepcounter{reaction}R\arabic{reaction} & OH  +    NCO  &$\!\!\!\rightarrow$ &  NO   +   HCO   & $ 1.8\!\times\! 10^{-11}  e^{-5700/T}$  & Ca01\\  
 \refstepcounter{reaction}R\arabic{reaction} & CH   +   NO  &$\!\!\!\rightarrow$ &   H    +   NCO        & $ 4.4\!\times\! 10^{-11} $  & Ge99\\  

\multicolumn{6}{l}{\bf HNCO}\\
 \refstepcounter{reaction}\label{RHNCO+O}R\arabic{reaction} & OH +  NCO  &$\!\!\!\rightarrow$ &  O + HNCO  & $ 6.22\!\times\! 10^{-14} \left(T/298 \right)^{2.51}e^{-2975/T}$  & Ts92\\
  \refstepcounter{reaction}R\arabic{reaction} & OH   +  HNCO  &$\!\!\!\rightarrow$ &   H$_2$O  +   NCO   & $ 9.43\!\times\! 10^{-14} \left(T/298 \right)^{2.0}e^{-1290/T}$    & Ts92\\  
 \refstepcounter{reaction}R\arabic{reaction} & H   +   HNCO  &$\!\!\!\rightarrow$ &  H$_2$    +   NCO   & $ 1.9\!\times\! 10^{-12} \left(T/298 \right)^{1.66}e^{-7000/T}$  & Mer96\\  
   \refstepcounter{reaction}R\arabic{reaction} & H   +    HNCO &$\!\!\!\rightarrow$ &    NH$_2$  +   CO     & $ 6.0\!\times\! 10^{-13} \left(T/298 \right)^{1.7}e^{-1950/T}$    & Mi92\\  
\refstepcounter{reaction}R\arabic{reaction}  & HOCN        + M         &$\!\!\!\rightarrow$ &  HNCO       +  M          & $  1.0\!\times\! 10^{-11}e^{ -14400/T}$ & De00\\ 
 \refstepcounter{reaction}\label{RHNCO}R\arabic{reaction} &  NH   +   CO  + M &$\!\!\!\rightarrow$ &   HNCO  + M  & $ 3.0\!\times\! 10^{-33}  e^{-3500/T}$  & rev-Ts92\\ 
         &  NH   +   CO  + M &$\!\!\!\rightarrow$ &   HNCO  + M  & $ 3.0\!\times\! 10^{-14}  e^{-3500/T}$  & rev-Ts92\\   
 \refstepcounter{reaction}R\arabic{reaction} & OH   +   HCN  &$\!\!\!\rightarrow$ &  H    +   HNCO   & $ 4.18\!\times\! 10^{-18} \left(T/298 \right)^{4.71}e^{248/T}$  & Ts91\\   

\multicolumn{6}{l}{\bf NH$_2$CO}\\
   \refstepcounter{reaction}R\arabic{reaction} & H   +    HNCO  + M &$\!\!\!\rightarrow$ &   NH$_2$CO  + M   & $ 1.4\!\times\! 10^{-27} \left(T/298 \right)^{-1.9} e^{-1400/T}$    & note\\     
              & H   +    HNCO  &$\!\!\!\rightarrow$ &   NH$_2$CO    & $ 1.4\!\times\! 10^{-08} \left(T/298 \right)^{-1.9} e^{-1400/T}$    & Ng96\\   
 \refstepcounter{reaction}\label{RNH2+CO}R\arabic{reaction} &  NH$_2$  +   CO  + M &$\!\!\!\rightarrow$ &   NH$_2$CO  + M  & $ 7.8\!\times\! 10^{-30}  \left(T/298 \right)^{-7.6} e^{-5530/T}$  & note\\ 
         &  NH$_2$   +   CO  + M &$\!\!\!\rightarrow$ &   NH$_2$CO  + M    & $ 8.4\!\times\! 10^{-13}  e^{-3460/T}$  & note\\   
 \refstepcounter{reaction}\label{RNH2CO+M}R\arabic{reaction} & NCHOH + M &$\!\!\!\rightarrow$ &   NH$_2$CO + M  & $ 1.0\!\times\! 10^{-11}  e^{-14400/T}$    & De00\\  
  \refstepcounter{reaction}\label{RNH2CO}R\arabic{reaction} & H  +     NH$_2$CO &$\!\!\!\rightarrow$ &   H$_2$   +   HNCO   & $ 1.4\!\times\! 10^{-11} $    & note\\  
  \refstepcounter{reaction}R\arabic{reaction} & H  +     NH$_2$CO &$\!\!\!\rightarrow$ &   NH$_3$   +   CO   & $ 3.6\!\times\! 10^{-11}  $    & note\\  
  \refstepcounter{reaction}R\arabic{reaction} & O  +     NH$_2$CO &$\!\!\!\rightarrow$ &  OH   +   HNCO   & $ 1.4\!\times\! 10^{-11} $    & note\\  
  \refstepcounter{reaction}R\arabic{reaction} & O  +     NH$_2$CO &$\!\!\!\rightarrow$ &   NH$_2$   +   CO$_2$   & $ 3.6\!\times\! 10^{-11}  $    & note\\  
  \refstepcounter{reaction}R\arabic{reaction} & OH  +     NH$_2$CO &$\!\!\!\rightarrow$ &  H$_2$O   +   HNCO   & $ 2.0\!\times\! 10^{-11} $    &note \\  
  \refstepcounter{reaction}R\arabic{reaction} & OH  +     NH$_2$CO &$\!\!\!\rightarrow$ &   NH$_3$   +   CO$_2$   & $ 8.0\!\times\! 10^{-11}  $    &note\\  

\multicolumn{6}{l}{\bf NH$_2$CHO}\\   % formamide
\refstepcounter{reaction}\label{RNH2CHO}R\arabic{reaction} & NH$_2$CO  + H  + M &$\!\!\!\rightarrow$ &  NH$_2$CHO   + M  &  $2.0\!\times\! 10^{-29}$ & note\\
     & NH$_2$CO  + H  &$\!\!\!\rightarrow$ &  NH$_2$CHO    &  $1.5\!\times\! 10^{-10} \left(T/298 \right)^{0.16}$ & note\\ 
\refstepcounter{reaction}R\arabic{reaction} & NH$_2$CHO  + H  &$\!\!\!\rightarrow$ &  NH$_2$CO + H$_2$ &  $2.1\!\times\! 10^{-11}e^{ -3500/T}$ & Sy01\\
\refstepcounter{reaction}R\arabic{reaction} & NH$_2$CHO  + H  &$\!\!\!\rightarrow$ &  NH$_3$ + HCO &  $2.0\!\times\! 10^{-11}e^{ -9620/T}$ & Sy01\\
\refstepcounter{reaction}R\arabic{reaction} & NH$_2$CHO  + CH$_3$  &$\!\!\!\rightarrow$ &  NH$_2$CO   + CH$_4$  &  $6.6\!\times\! 10^{-14}e^{ -3520/T}$ & Bo70\\
\refstepcounter{reaction}R\arabic{reaction} & NH$_2$CHO  + OH  &$\!\!\!\rightarrow$ &  NH$_2$CO   + H$_2$O &  $1.3\!\times\! 10^{-12}e^{ +365/T}$ & Bu16\\
\refstepcounter{reaction}\label{RNH2CHO+O}R\arabic{reaction} & NH$_2$CHO  + O  &$\!\!\!\rightarrow$ &  NH$_2$CO   + OH  &  $5.0\!\times\! 10^{-11}e^{ -1100/T}$ & note\\


\multicolumn{6}{l}{\bf H$_2$CN}\\
 \refstepcounter{reaction}\label{RH2CN}R\arabic{reaction}   & HCN          + H    + M & $\!\!\!\rightarrow$ &  H$_2$CN        + M &$  7.8\!\times\! 10^{-31} \left(T/298 \right)^{-2.73}e^{ -3860/T}$ &Ts91\\
             & HCN          + H    & $\!\!\!\rightarrow$ &  H$_2$CN     &$  5.5\!\times\! 10^{-11} e^{ -2440/T}$ & Ts91\\
 \refstepcounter{reaction}R\arabic{reaction}   & N   + CH$_3$      &$\!\!\!\rightarrow$ &  H$_2$CN   + H       & $  3.9\!\times\! 10^{-10} e^{  -420/T}$ &  Ma89\\
 \refstepcounter{reaction}\label{RN+C2H5}R\arabic{reaction}  & N            + C$_2$H$_5$  &$\!\!\!\rightarrow$ &  H$_2$CN    + CH$_3$       & $  5.5\!\times\! 10^{-11}$ & St95,Ya05\\
\refstepcounter{reaction}R\arabic{reaction}  & H  + H$_2$CN    &$\!\!\!\rightarrow$ &  HCN    +  H$_2$  & $ 8.3\!\times\! 10^{-11} $ & To03 \\
 \refstepcounter{reaction}R\arabic{reaction}  & O + H$_2$CN    &$\!\!\!\rightarrow$ &  HCN    +  OH   & $ 8.3\!\times\! 10^{-11} $ & To03 \\
 \refstepcounter{reaction}\label{ROH+H2CN}R\arabic{reaction}  & OH + H$_2$CN    &$\!\!\!\rightarrow$ &  HCN  +  H$_2$O & $ 7.7\!\times\! 10^{-12} $ & Ni03\\
 \refstepcounter{reaction}\label{RN+H2CN}R\arabic{reaction}  & N   + H$_2$CN    &$\!\!\!\rightarrow$ &  NH    +  HCN   & $ 1\!\times\! 10^{-10} e^{  -200/T}$ & Ne90\\
  \refstepcounter{reaction}\label{RCN+H2CN}R\arabic{reaction}  & CN + H$_2$CN  &$\!\!\!\rightarrow$ &  HCN +  HCN  & $ 1.0\!\times\! 10^{-10} $ & assumed \\

 \refstepcounter{reaction}R\arabic{reaction}  & NH$_2$    + H$_2$CN    &$\!\!\!\rightarrow$ &  HCN    +  NH$_3$         & $ 7.7\!\times\! 10^{-12} $ &  assumed \\
 \refstepcounter{reaction}R\arabic{reaction}  & CH$_3$     + H$_2$CN    &$\!\!\!\rightarrow$ &  HCN    +  CH$_4$            & $ 7.7\!\times\! 10^{-12} $ & assumed \\
 \refstepcounter{reaction}\label{RC2H+H2CN}R\arabic{reaction}  & C$_2$H + H$_2$CN  &$\!\!\!\rightarrow$ &  HCN  +  C$_2$H$_2$   & $ 1.0\!\times\! 10^{-11} $ & assumed \\


 \multicolumn{6}{l}{\bf NNH}\\
 \refstepcounter{reaction}\label{RNNH}R\arabic{reaction} & H  +     N$_2$ + M &$\!\!\!\rightarrow$ &      NNH + M & $ 3.0\!\times\! 10^{-30} \left(T/298 \right)^{-0.6} e^{-7820/T}$    & note \\     
          & H  +     N$_2$ &$\!\!\!\rightarrow$ &      NNH  & $ 3.0\!\times\! 10^{-10} \left(T/298 \right)^{-0.6}  e^{-7820/T}$    & Ca05b\\  
  \refstepcounter{reaction}R\arabic{reaction} & NNH  +   H    &$\!\!\!\rightarrow$ &    N$_2$   +   H$_2$     & $ 1.67\!\times\! 10^{-10} $  & Gl08 \\  
 \refstepcounter{reaction}R\arabic{reaction} & NH$_2$  +   N   &$\!\!\!\rightarrow$ &     NNH  +   H      & $ 1.2\!\times\! 10^{-10} $  & Wh84\\  
% R119  & NH$_2$   + N   &$\!\!\!\rightarrow$ &  N$_2$   + H  + H    & $  1.2\!\times\! 10^{-10}$ & Wh84\\
 \refstepcounter{reaction}R\arabic{reaction} & NNH  +   O   &$\!\!\!\rightarrow$ &     NH   +   NO      & $ 2.5\!\times\! 10^{-11} $  & Ha03\\   
 \refstepcounter{reaction}R\arabic{reaction} & NNH  +   O   &$\!\!\!\rightarrow$ &     N$_2$   +   OH    & $ 1.7\!\times\! 10^{-11} $  & Ha03\\   
  \refstepcounter{reaction}R\arabic{reaction} & NNH  +   O   &$\!\!\!\rightarrow$ &     HNO   +   N    & $ 3.0\!\times\! 10^{-13} e^{-7000/T}$  & Ha03\\   

 \multicolumn{6}{l}{\bf N$_2$H$_2$}\\
 \refstepcounter{reaction}\label{RN2H2}R\arabic{reaction} & NNH  +   H  + M  &$\!\!\!\rightarrow$ &    N$_2$H$_2$  + M   & $ 3.0\!\times\! 10^{-31}  \left(T/298 \right)^{-2.0} e^{-500/T}$  & note \\  
    & NNH  +   H  + M  &$\!\!\!\rightarrow$ &    N$_2$H$_2$  + M   & $ 3.0\!\times\! 10^{-11}  e^{-500/T}$  & note \\  
  \refstepcounter{reaction}R\arabic{reaction} & H   +    N$_2$H$_2$ &$\!\!\!\rightarrow$ &    NNH  +   H$_2$  & $ 4.53\!\times\! 10^{-13} \left(T/298 \right)^{2.63} e^{+115/T}$    & Li96b\\  
 \refstepcounter{reaction}R\arabic{reaction} & O +  N$_2$H$_2$ &$\!\!\!\rightarrow$ &  NNH  +   H$_2$O  & $ 1.67\!\times\! 10^{-13}   \left(T/298 \right)^{0.5} $    & Gl08  \\  
 \refstepcounter{reaction}R\arabic{reaction} & OH   +    N$_2$H$_2$ &$\!\!\!\rightarrow$ &    NNH  +   H$_2$O  & $ 1.67\!\times\! 10^{-11}  e^{-1000/T}$   & Gl08  \\  
 \refstepcounter{reaction}R\arabic{reaction} & NH  +     NH$_2$   &$\!\!\!\rightarrow$ &   N$_2$H$_2$ +  H    & $ 1.44\!\times\! 10^{-10}  \left(T/298 \right)^{-0.5}$    & Da90\\    
 \refstepcounter{reaction}R\arabic{reaction} & NH$_2$  +  NH$_2$   &$\!\!\!\rightarrow$ &   N$_2$H$_2$ +  H$_2$    & $ 1.3\!\times\! 10^{-12}$    & Sto95 \\    % at 300 K
 \refstepcounter{reaction}R\arabic{reaction} & N  +     NH$_3$   &$\!\!\!\rightarrow$ &   N$_2$H$_2$ +  H    & $ 7.42\!\times\! 10^{-12}  e^{-4830/T}$    & Ba88\\    %products assumed

 \multicolumn{6}{l}{{\bf N$_2$H$_3$} and {\bf N$_2$H$_4$}}\\
  \refstepcounter{reaction}\label{RN2H3}R\arabic{reaction} & N$_2$H$_2$  +   H +M  &$\!\!\!\rightarrow$ &   N$_2$H$_3$   +M   & $ 3.0\!\times\! 10^{-30}  e^{-740/T} $ & note\\   
          & N$_2$H$_2$  +   H   &$\!\!\!\rightarrow$ &     N$_2$H$_3$      & $ 9.0\!\times\! 10^{-12} e^{-1220/T} $  & note \\   
 % \refstepcounter{reaction}R\arabic{reaction} & N$_2$H$_3$  +   H    &$\!\!\!\rightarrow$ &    N$_2$H$_2$   +   H$_2$     & $ 1.6\!\times\! 10^{-10} $  & \\  
 \refstepcounter{reaction}R\arabic{reaction} & N$_2$H$_3$  +   H    &$\!\!\!\rightarrow$ &    NH$_2$   +   NH$_2$     & $ 2.5\!\times\! 10^{-10}  e^{-820/T} $  & Li14 \\  
  \refstepcounter{reaction}R\arabic{reaction} & N$_2$H$_3$  +   O    &$\!\!\!\rightarrow$ &    N$_2$H$_2$   +   OH     & $ 5.0\!\times\! 10^{-11} $  & assumed\\  
  \refstepcounter{reaction}R\arabic{reaction} & N$_2$H$_3$  +   OH    &$\!\!\!\rightarrow$ &    N$_2$H$_2$   +   H$_2$O    & $ 5.0\!\times\! 10^{-11} $  & assumed\\  
 % \refstepcounter{reaction}R\arabic{reaction} & N$_2$H$_3$  +   CH$_3$    &$\!\!\!\rightarrow$ &    N$_2$H$_2$   +   CH$_4$   & $ 3.0\!\times\! 10^{-11} $  & \\  
  
  
% \multicolumn{6}{l}{\bf N$_2$H$_4$}\\
\refstepcounter{reaction}\label{RN2H4}R\arabic{reaction} & NH$_2$  +     NH$_2$ + M &$\!\!\!\rightarrow$ &      N$_2$H$_4$ + M & $ 2.0\!\times\! 10^{-29} \left(T/298 \right)^{-3.9}  $   &  Fa95 \\     
          & NH$_2$  +     NH$_2$ &$\!\!\!\rightarrow$ &   N$_2$H$_4$  & $ 1.2\!\times\! 10^{-10} \left(T/298 \right)^{0.3}  $    & Fa95 \\  
 \refstepcounter{reaction}R\arabic{reaction} & N$_2$H$_3$  +   H +M  &$\!\!\!\rightarrow$ &   N$_2$H$_4$   +M   & $ 2.0\!\times\! 10^{-29} \left(T/298 \right)^{-3.9} $ &note\\   
          & N$_2$H$_3$  +   H   &$\!\!\!\rightarrow$ &     N$_2$H$_4$      & $ 1.2\!\times\! 10^{-10} \left(T/298 \right)^{0.3} $  &note \\   
   \refstepcounter{reaction}R\arabic{reaction} & N$_2$H$_4$  +   H    &$\!\!\!\rightarrow$ &    N$_2$H$_3$   +   H$_2$     & $ 1.2\!\times\! 10^{-11} e^{-1261/T}$  & Va95 \\  
   \refstepcounter{reaction}R\arabic{reaction} & N$_2$H$_4$  +   O    &$\!\!\!\rightarrow$ &    N$_2$H$_2$   +   H$_2$O    & $ 9.9\!\times\! 10^{-12} $  & Va01a, La92 \\  
  \refstepcounter{reaction}R\arabic{reaction} & N$_2$H$_4$  +   OH    &$\!\!\!\rightarrow$ &    N$_2$H$_3$   +   H$_2$O    & $ 2.2\!\times\! 10^{-11} \left(T/298 \right)^{-1.33} $  & Va01b \\  
  \refstepcounter{reaction}R\arabic{reaction} & N$_2$H$_4$  +   CH$_3$    &$\!\!\!\rightarrow$ &    N$_2$H$_3$   +   CH$_4$   & $ 1.0\!\times\! 10^{-14}  \left(T/298 \right)^{4.0}e^{-2038/T}$  & Li06 \\  


 \multicolumn{6}{l}{\bf HNO}\\
  \refstepcounter{reaction}R\arabic{reaction} &  H  +     NO + M &$\!\!\!\rightarrow$ &   HNO + M & $ 1.2\!\times\! 10^{-31} \left(T/298 \right)^{-1.17} e^{-210/T} $   &  Ts91 \\     
          & H  +     NO  &$\!\!\!\rightarrow$ &   HNO  & $ 2.4\!\times\! 10^{-10}  \left(T/298 \right)^{-0.4}$    & Ts91  \\  
 \refstepcounter{reaction}R\arabic{reaction} & HCO  +  NO  &$\!\!\!\rightarrow$ &  HNO   +  CO   & $ 1.3\!\times\! 10^{-11}  $  & Ts91 \\  
 \refstepcounter{reaction}R\arabic{reaction} & NH  +  OH  &$\!\!\!\rightarrow$ &  HNO   +  H   & $ 3.3\!\times\! 10^{-11}  $  & Co91 \\  
 \refstepcounter{reaction}R\arabic{reaction} & NH  +  O$_2$  &$\!\!\!\rightarrow$ &  HNO   +  O   & $ 6.8\!\times\! 10^{-14} \left(T/298 \right)^{2.0}e^{-3270/T} $  & Mi92 \\  
 \refstepcounter{reaction}R\arabic{reaction} & NH$_2$ +  O  &$\!\!\!\rightarrow$ &  HNO   +  H   & $ 7.5\!\times\! 10^{-11}  $  & Co91 \\  
 \refstepcounter{reaction}R\arabic{reaction} & NH$_2$  +  O$_2$  &$\!\!\!\rightarrow$ &  HNO   +  OH   & $ 1.7\!\times\! 10^{-11} e^{-13200/T} $  & He95 \\  
 \refstepcounter{reaction}R\arabic{reaction} & CH$_3$O +  NO  &$\!\!\!\rightarrow$ &  HNO   +  H$_2$CO   & $ 4.0\!\times\! 10^{-12}  \left(T/298 \right)^{-0.7}$  & At97 \\  
 \refstepcounter{reaction}R\arabic{reaction} & H + HNO  &$\!\!\!\rightarrow$ &  H$_2$   +  NO   & $ 2.4\!\times\! 10^{-11}  \left(T/298 \right)^{0.94}  e^{-250/T} $ & Ng04\\  
 \refstepcounter{reaction}R\arabic{reaction} & O + HNO  &$\!\!\!\rightarrow$ &  OH   +  NO   & $ 3.8\!\times\! 10^{-11} $ & In99 \\  
 \refstepcounter{reaction}R\arabic{reaction} & OH + HNO  &$\!\!\!\rightarrow$ &  H$_2$O   +  NO   & $ 1.8\!\times\! 10^{-12}  \left(T/298 \right)^{1.2}  e^{-168/T} $ & Ng04 \\  
 \refstepcounter{reaction}R\arabic{reaction} & H$_2$ + HNO  &$\!\!\!\rightarrow$ &  H$_2$O   +  NH   & $ 1.66\!\times\! 10^{-10}  e^{-8060/T} $ & Ro94 \\  
 \refstepcounter{reaction}R\arabic{reaction} & CH$_3$ + HNO  &$\!\!\!\rightarrow$ &  CH$_4$  +  NO   & $ 1.85\!\times\! 10^{-11} \left(T/298 \right)^{0.76}  e^{-176/T} $ & Ch05 \\  
 \refstepcounter{reaction}R\arabic{reaction} & CN  + HNO  &$\!\!\!\rightarrow$ &  HCN  +  NO   & $ 3.0\!\times\! 10^{-11} $ & Ts91 \\  
 \refstepcounter{reaction}R\arabic{reaction} & CO  + HNO  &$\!\!\!\rightarrow$ &  NH  +  CO$_2$   & $ 3.3\!\times\! 10^{-12} e^{-6190/T} $ & Ro94 \\  
 \refstepcounter{reaction}R\arabic{reaction} & NH$_2$ + HNO  &$\!\!\!\rightarrow$ &  NH$_3$  +  NO   & $ 6.5\!\times\! 10^{-13} \left(T/298 \right)^{1.63}  e^{+630/T} $ & Meb96\\  

 \refstepcounter{reaction}R\arabic{reaction} & CH$_3$O + HNO  &$\!\!\!\rightarrow$ &  CH$_3$OH  +  NO   & $ 5.25\!\times\! 10^{-11} $ & He88\\  


 \multicolumn{6}{l}{\bf N$_2$O}\\
 \refstepcounter{reaction}R\arabic{reaction} & NO + HNO  &$\!\!\!\rightarrow$ & N$_2$O  +  OH   & $ 1.0\!\times\! 10^{-17} \left(T/298 \right)^{4.33}  e^{-12600/T} $ & Me96, Di95 \\  % Diau does the reverse
 \refstepcounter{reaction}R\arabic{reaction} & HNO + HNO  &$\!\!\!\rightarrow$ & N$_2$O  +  H$_2$O   & $ 4.2\!\times\! 10^{-17} \left(T/298 \right)^{4.0}  e^{-600/T} $ & He92 \\  
 \refstepcounter{reaction}R\arabic{reaction} & NH + NO  &$\!\!\!\rightarrow$ & N$_2$O  +  H   & $ 4.0\!\times\! 10^{-11} $ & Ba94 \\  
 \refstepcounter{reaction}R\arabic{reaction} & NO + NCO  &$\!\!\!\rightarrow$ & N$_2$O  +  CO   & $ 5.2\!\times\! 10^{-11} \left(T/298 \right)^{-1.03}  e^{-420/T} $ & Lin93\\  
  \refstepcounter{reaction}R\arabic{reaction} & H + N$_2$O  &$\!\!\!\rightarrow$ & N$_2$  +  OH   & $ 1.6\!\times\! 10^{-10} e^{-7600/T} $ & Ts91 \\  
 \refstepcounter{reaction}R\arabic{reaction} & O + N$_2$O  &$\!\!\!\rightarrow$ & NO  +  NO   & $ 1.5\!\times\! 10^{-10} e^{-14000/T} $ & Me00\\  
 \refstepcounter{reaction}R\arabic{reaction} & O + N$_2$O  &$\!\!\!\rightarrow$ & N$_2$  +  O$_2$   & $ 6.0\!\times\! 10^{-12} e^{-8000/T} $ & Me00\\  
 \refstepcounter{reaction}R\arabic{reaction} & OH + N$_2$O  &$\!\!\!\rightarrow$ & HO$_2$  +  N$_2$   & $ 1.0\!\times\! 10^{-14} \left(T/298 \right)^{4.72} e^{-18400/T} $ & Me96 \\  
  \refstepcounter{reaction}R\arabic{reaction} & NNH  +   O   &$\!\!\!\rightarrow$ &     N$_2$O   +   H    & $ 1.0\!\times\! 10^{-10} e^{-700/T}$  & Ha03\\   
% \refstepcounter{reaction}R\arabic{reaction} & O + NNH  &$\!\!\!\rightarrow$ & N$_2$O  +  H$_2$O   & $ 5.1\!\times\! 10^{-10} \left(T/298 \right)^{-0.76}  e^{-775/T} $ & Ha03 \\  % error

 \multicolumn{6}{l}{\bf NO$_2$}\\
  \refstepcounter{reaction}R\arabic{reaction} &  NO  +    O + M &$\!\!\!\rightarrow$ &   NO$_2$ + M & $ 9.0\!\times\! 10^{-32}  \left(T/298 \right)^{-1.5} $   & De97 \\     
          & NO  +    O   &$\!\!\!\rightarrow$ &   NO$_2$  & $ 3.0\!\times\! 10^{-11} $    &  De97\\  
 \refstepcounter{reaction}R\arabic{reaction} & NO + O$_3$  &$\!\!\!\rightarrow$ & NO$_2$  +  O$_2$   & $ 1.4\!\times\! 10^{-12} e^{-1310/T} $ & At04 \\  
 \refstepcounter{reaction}R\arabic{reaction} & NO + HO$_2$  &$\!\!\!\rightarrow$ & NO$_2$  +  OH   & $ 3.6\!\times\! 10^{-12} e^{+270/T} $ & At04  \\  
 \refstepcounter{reaction}R\arabic{reaction} & NO$_2$  + O &$\!\!\!\rightarrow$ & NO + O$_2$  & $ 5.5\!\times\! 10^{-12} e^{+188/T} $ & At04 \\  
 \refstepcounter{reaction}R\arabic{reaction} & NO$_2$  + H &$\!\!\!\rightarrow$ & NO + OH  & $ 1.5\!\times\! 10^{-10}  $ &  Su02\\  
% \refstepcounter{reaction}R\arabic{reaction} & NO$_2$  + OH &$\!\!\!\rightarrow$ & NO + HO$_2$  & $ 3.0\!\times\! 10^{-11} e^{-3360/T} $ &  Ts91 \\  
\refstepcounter{reaction}R\arabic{reaction} & NO$_2$  + HCO &$\!\!\!\rightarrow$ & HNO + CO$_2$  & $ 5.5\!\times\! 10^{-11}  $ & Dam07 \\  
\refstepcounter{reaction}R\arabic{reaction} & NO$_2$  + N &$\!\!\!\rightarrow$ & N$_2$O  + O  & $ 5.8\!\times\! 10^{-12}   e^{+221/T} $ & De97 \\  
\refstepcounter{reaction}R\arabic{reaction} & NO$_2$  + N &$\!\!\!\rightarrow$ & NO  + NO  & $ 4.5\!\times\! 10^{-12}   e^{+221/T} $ &  De97 \\  
\refstepcounter{reaction}R\arabic{reaction} & NO$_2$  + N &$\!\!\!\rightarrow$ & N$_2$ + O$_2$  & $ 1.5\!\times\! 10^{-12}   e^{+221/T} $ &  De97 \\  
\refstepcounter{reaction}R\arabic{reaction} & NO$_2$  + HS  &$\!\!\!\rightarrow$ & HSO + NO  & $ 2.9\!\times\! 10^{-11} e^{+240/T} $ & At97 \\  
\refstepcounter{reaction}R\arabic{reaction} & NO$_2$  + SO  &$\!\!\!\rightarrow$ & SO$_2$ + NO  & $ 1.4\!\times\! 10^{-11}  $ &  At04 \\  
\refstepcounter{reaction}R\arabic{reaction} & NO$_2$  + CH$_3$  &$\!\!\!\rightarrow$ & CH$_3$O + NO  & $ 2.66\!\times\! 10^{-11}  $ & Sri05 \\  
\refstepcounter{reaction}R\arabic{reaction} & NO$_2$  + CN &$\!\!\!\rightarrow$ & NO + NCO   & $ 8.0\!\times\! 10^{-11}   e^{+186/T} $ & Pa93  \\  
\refstepcounter{reaction}R\arabic{reaction} & NO$_2$  + CN &$\!\!\!\rightarrow$ & CO + N$_2$O   & $ 7.1\!\times\! 10^{-12}    $ &  Pa93 \\  
\refstepcounter{reaction}R\arabic{reaction} & NO$_2$  + CN &$\!\!\!\rightarrow$ & CO$_2$ + N$_2$   & $ 5.0\!\times\! 10^{-12}  $ & Pa93  \\  
\refstepcounter{reaction}R\arabic{reaction} & NO$_2$  + NCO &$\!\!\!\rightarrow$ & CO$_2$ + N$_2$O   & $ 1.6\!\times\! 10^{-11}  $ &  Pa93 \\  
\refstepcounter{reaction}R\arabic{reaction} & NO$_2$  + NH$_2$ &$\!\!\!\rightarrow$ & N$_2$O + H$_2$O  & $ 7.0\!\times\! 10^{-12} \left(T/298 \right)^{-1.44}  e^{-135/T} $ &  Pa93 \\  
 \refstepcounter{reaction}R\arabic{reaction} & NO$_2$ + CO &$\!\!\!\rightarrow$ & NO + CO$_2$ & $1.15\!\times\! 10^{-14}\left(T/298 \right)^{3.44} e^{-11425/T} $ &  Kr18\\  

 \multicolumn{6}{l}{\bf HNO$_2$}\\
  \refstepcounter{reaction}R\arabic{reaction} &  NO  +    OH + M &$\!\!\!\rightarrow$ &   HNO$_2$ + M & $ 7.0\!\times\! 10^{-31}  \left(T/298 \right)^{-2.6} $   & De97 \\     
          & NO  +    OH    &$\!\!\!\rightarrow$ &   HNO$_2$  & $ 3.6\!\times\! 10^{-11} \left(T/298 \right)^{-0.1} $    &  De97\\  
 \refstepcounter{reaction}R\arabic{reaction} & NO$_2$  + HNO &$\!\!\!\rightarrow$ & HNO$_2$ + NO  & $ 1.0\!\times\! 10^{-12}  e^{-1000/T} $ &  Ts91\\  
\refstepcounter{reaction}R\arabic{reaction} & NO$_2$  + H$_2$CO  &$\!\!\!\rightarrow$ & HNO$_2$ + HCO  & $ 9.5\!\times\! 10^{-15} \left(T/298 \right)^{2.77}  e^{-6910/T} $ &  Ts91\\  
 \refstepcounter{reaction}R\arabic{reaction} & HNO$_2$  + H &$\!\!\!\rightarrow$ & NO$_2$ + H$_2$  & $ 2.0\!\times\! 10^{-11}  e^{-3700/T} $ &  Ts91\\  
 \refstepcounter{reaction}R\arabic{reaction} & HNO$_2$  + O &$\!\!\!\rightarrow$ & NO$_2$ + OH  & $ 2.0\!\times\! 10^{-11}  e^{-3000/T} $ & Ts91 \\  
 \refstepcounter{reaction}R\arabic{reaction} & HNO$_2$  + OH &$\!\!\!\rightarrow$ & NO$_2$ + H$_2$O  & $ 6.0\!\times\! 10^{-12}  $ &  At97 \\  

 \multicolumn{6}{l}{\bf NO$_3$, HNO$_3$}\\
  \refstepcounter{reaction}R\arabic{reaction} &  NO$_2$  +    OH + M &$\!\!\!\rightarrow$ &   HNO$_3$ + M & $ 2.6\!\times\! 10^{-30}  \left(T/298 \right)^{-2.9} $   &  At97 \\     
          & NO$_2$  +    OH    &$\!\!\!\rightarrow$ &   HNO$_3$  & $ 7.5\!\times\! 10^{-11} \left(T/298 \right)^{-0.6} $    & At97 \\  
\refstepcounter{reaction}R\arabic{reaction} & HNO$_3$  + H   &$\!\!\!\rightarrow$ & NO$_2$ + H$_2$O  & $ 1.4\!\times\! 10^{-14} \left(T/298 \right)^{3.29}  e^{-3160/T} $ &  Bo97 \\  
\refstepcounter{reaction}R\arabic{reaction} & HNO$_3$  + OH   &$\!\!\!\rightarrow$ & NO$_3$ + H$_2$O  & $ 1.5\!\times\! 10^{-13}  $ & At97 \\  
\refstepcounter{reaction}R\arabic{reaction} & HNO$_3$  + H   &$\!\!\!\rightarrow$ & NO$_3$ + H$_2$  & $ 5.6\!\times\! 10^{-12} \left(T/298 \right)^{1.53}  e^{-8250/T} $ &  Bo97\\  
  \refstepcounter{reaction}R\arabic{reaction} &  NO$_2$  +    O + M &$\!\!\!\rightarrow$ &   NO$_3$ + M & $ 1.3\!\times\! 10^{-30}  \left(T/298 \right)^{-1.5} $   &  At04 \\     
          & NO$_2$  +    O    &$\!\!\!\rightarrow$ &   NO$_3$  & $ 2.3\!\times\! 10^{-11} $    &  At04 \\  
\refstepcounter{reaction}R\arabic{reaction} & NO$_2$  + O$_3$   &$\!\!\!\rightarrow$ & NO$_3$ + O$_2$  & $ 1.4\!\times\! 10^{-13}   e^{-2470/T} $ &  At04 \\  
\refstepcounter{reaction}R\arabic{reaction} & NO$_3$  + H   &$\!\!\!\rightarrow$ & NO$_2$ + OH  & $ 9.4\!\times\! 10^{-11}   $ & Be92 \\  
\refstepcounter{reaction}R\arabic{reaction} & NO$_3$  + O   &$\!\!\!\rightarrow$ & NO$_2$ + O$_2$  & $ 1.7\!\times\! 10^{-11}   $ &   At04 \\  
\refstepcounter{reaction}R\arabic{reaction} & NO$_3$  + OH   &$\!\!\!\rightarrow$ & NO$_2$ + HO$_2$  & $ 2.0\!\times\! 10^{-11}   $ &  At04 \\  
\refstepcounter{reaction}R\arabic{reaction} & NO$_3$  + HO$_2$   &$\!\!\!\rightarrow$ & HNO$_3$ + O$_2$  & $ 1.9\!\times\! 10^{-12}   $ & Be92 \\  
\refstepcounter{reaction}R\arabic{reaction} & NO$_3$  + NO   &$\!\!\!\rightarrow$ & NO$_2$ + NO$_2$  & $ 1.8\!\times\! 10^{-11}   e^{+110/T} $ & At04 \\  


\multicolumn{6}{l}{\bf $^1$CH$_2$}\\
% \refstepcounter{reaction}R\arabic{reaction}  &   CH     + H$_2$      &$\!\!\!\rightarrow$ &  $^1$CH$_2$   + H   & $  1.9\!\times\! 10^{-11}e^{-4400/T}$ & xx\\
 \refstepcounter{reaction}R\arabic{reaction}  &   CH$_3$     + OH      &$\!\!\!\rightarrow$ &  $^1$CH$_2$   + H$_2$O   & $  1.2\!\times\! 10^{-10}e^{-1400/T}$ & Ba92\\
 \refstepcounter{reaction}\label{1CH2+H2}R\arabic{reaction}  & $^1$CH$_2$   + H$_2$    &$\!\!\!\rightarrow$ &  CH$_3$    + H        & $  6.0\!\times\! 10^{-10}$ & Ga08\\
 \refstepcounter{reaction}R\arabic{reaction}  & $^1$CH$_2$   + H$_2$    &$\!\!\!\rightarrow$ &  CH$_2$    + H$_2$            & $  4.0\!\times\! 10^{-11}$ & Do18\\
 \refstepcounter{reaction}R\arabic{reaction}  & $^1$CH$_2$   + CH$_4$      &$\!\!\!\rightarrow$ &  CH$_3$       + CH$_3$     & $  4.1\!\times\! 10^{-11}$ & Do18 \\
 \refstepcounter{reaction}R\arabic{reaction}  & $^1$CH$_2$   + CH$_4$      &$\!\!\!\rightarrow$ &  CH$_4$       + CH$_2$     & $  4.1\!\times\! 10^{-11}$ & Do18\\
% \refstepcounter{reaction}R\arabic{reaction}  & $^1$CH$_2$   + C$_2$H$_2$      &$\!\!\!\rightarrow$ &  C$_3$H$_4$       + H                                  & $  3.3\!\times\! 10^{+10}$ & high pressure limit \\
% \refstepcounter{reaction}R\arabic{reaction}  & $^1$CH$_2$   + C$_2$H$_2$      &$\!\!\!\rightarrow$ &  C$_3$H$_4$       + H                                  & $  1.8\!\times\! 10^{-12} \left(T/298 \right)^{-0.9}$ & products  \\
% \refstepcounter{reaction}R\arabic{reaction}  & $^1$CH$_2$   + NH$_3$  &$\!\!\!\rightarrow$ &  CH$_3$       + NH$_2$     & $  3.6\!\times\! 10^{-11}$ & \\
 \refstepcounter{reaction}R\arabic{reaction}  & $^1$CH$_2$   + H   &$\!\!\!\rightarrow$ &  CH       + H$_2$            & $  5.0\!\times\! 10^{-11}$ & Ba92\\
 \refstepcounter{reaction}R\arabic{reaction}  & $^1$CH$_2$   + H   &$\!\!\!\rightarrow$ &  CH$_2$       + H                 & $  8.0\!\times\! 10^{-12}$ & \\
  \refstepcounter{reaction}\label{CH2+CO}R\arabic{reaction}  & $^1$CH$_2$   + CO   + M    &$\!\!\!\rightarrow$ &  CH$_2$CO     + M    & $  1.7\!\times\! 10^{-29} \left(T/298 \right)^{-7.2}e^{-1550/T}$ & note \\
  & $^1$CH$_2$   + CO    &$\!\!\!\rightarrow$ &  CH$_2$CO       & $  1.7\!\times\! 10^{-09} \left(T/298 \right)^{-7.2}e^{-1550/T}$ & note \\
 \refstepcounter{reaction}\label{1CH2+CO}R\arabic{reaction}  & $^1$CH$_2$ + CO&$\!\!\!\rightarrow$ &  CH$_2$    + CO & $ 4.7\!\times\! 10^{-11} e^{+60/T}$ & note \\
 \refstepcounter{reaction}\label{1CH2+CO2}R\arabic{reaction}  & $^1$CH$_2$  + CO$_2$  &$\!\!\!\rightarrow$ &  CH$_2$  + CO$_2$  & $ 7.4\!\times\! 10^{-11} e^{+120/T}$ & note \\
% \refstepcounter{reaction}R\arabic{reaction}  & $^1$CH$_2$   + CO$_2$   &$\!\!\!\rightarrow$ &  H$_2$CO       + CO     & $ 3.9\!\times\! 10^{-14} $ & ?? \\
\refstepcounter{reaction}R\arabic{reaction}   & CH$_2$CO   + $^1$CH$_2$   & $\!\!\!\rightarrow$ &  C$_2$H$_4$ + CO &$  2.0\!\times\! 10^{-10}\left(T/298 \right)^{-0.6} $ & Sa19 \\ 
\refstepcounter{reaction}R\arabic{reaction} & C$_2$H$_2$   +  $^1$CH$_2$  + M &$\!\!\!\rightarrow$ &  C$_3$H$_4$   + M   &   $1.0\!\times\! 10^{-29} \left(T/298 \right)^{1.2} e^{+500/T}$ & Po13 \\   
      & C$_2$H$_2$   +  $^1$CH$_2$  &$\!\!\!\rightarrow$ &  C$_3$H$_4$  &  $1.0\!\times\! 10^{-12} \left(T/298 \right)^{1.2} e^{+500/T}$ & Po13\\  
 \refstepcounter{reaction}\label{1CH2+N2}R\arabic{reaction}  & $^1$CH$_2$ + N$_2$ &$\!\!\!\rightarrow$ &  CH$_2$ + N$_2$ & $ 1.8\!\times\! 10^{-11} e^{+110/T}$ & note \\

\multicolumn{6}{l}{\bf CH$_2$N$_2$}\\
\refstepcounter{reaction}\label{CH2N2}R\arabic{reaction} & $^1$CH$_2$    +    N$_2$ + M  &$\!\!\!\rightarrow$ &   CH$_2$N$_2$   + M   &   $1.7\!\times\! 10^{-29} \left(T/298 \right)^{-7.2} e^{-1550/T}$ & Xu10\\  
           & $^1$CH$_2$    +    N$_2$   &$\!\!\!\rightarrow$ &   CH$_2$N$_2$   &   $1.7\!\times\! 10^{-09} \left(T/298 \right)^{-7.2} e^{-1550/T}$ & Xu10\\  
\refstepcounter{reaction}R\arabic{reaction} & CH$_2$N$_2$  +   H  &$\!\!\!\rightarrow$ &   CH$_3$  + N$_2$   &   $1.6\!\times\! 10^{-11}$ & Mo73 \\  
\refstepcounter{reaction}\label{CH2N2+OH}R\arabic{reaction} & CH$_2$N$_2$  +  OH &$\!\!\!\rightarrow$ &  N$_2$ +  H$_2$COH   &   $1.6\!\times\! 10^{-10}$ & An20 \\   
\refstepcounter{reaction}\label{CH2N2+1CH2}R\arabic{reaction} & CH$_2$N$_2$  +  $^1$CH$_2$  &$\!\!\!\rightarrow$ &  N$_2$   +  C$_2$H$_4$   &   $2.0\!\times\! 10^{-10}\left(T/298 \right)^{-0.6}$ & note \\  
\refstepcounter{reaction}\label{CH2N2+CH2}R\arabic{reaction} & CH$_2$N$_2$   +    CH$_2$   &$\!\!\!\rightarrow$ &  N$_2$    +      C$_2$H$_4$   &   $2.2\!\times\! 10^{-13} \left(T/298 \right)^{2.8} e^{-1200/T}$ & note \\   
\refstepcounter{reaction}\label{CH2N2+CH2}R\arabic{reaction} & CH$_2$N$_2$   +  CH$_3$  &$\!\!\!\rightarrow$ &   N$_2$   +   C$_2$H$_5$   &   $4.0\!\times\! 10^{-11} e^{-1200/T}$ & note \\   
\refstepcounter{reaction}R\arabic{reaction} & CH$_2$N$_2$  +  C$_2$H$_2$  &$\!\!\!\rightarrow$ &  C$_3$H$_4$  +  N$_2$   &   $2.0\!\times\! 10^{-11} e^{-4500/T}$ &note \\ 
\refstepcounter{reaction}R\arabic{reaction} & CH$_2$N$_2$  +  C$_2$H$_4$   &$\!\!\!\rightarrow$ &  C$_3$H$_6$  + N$ _2$ &   $2.0\!\times\! 10^{-11}  e^{-4500/T}$ &note \\    

\multicolumn{6}{l}{\bf CH$_2$CN}\\
\refstepcounter{reaction}R\arabic{reaction} & C$_2$H$_3$   +  N    &$\!\!\!\rightarrow$ &   CH$_2$CN  +  H   &   $6.1\!\times\! 10^{-11}$ & Pa96\\  
\refstepcounter{reaction}\label{CH2CN+O}R\arabic{reaction} & CH$_2$CN  +  O  &$\!\!\!\rightarrow$ &   H$_2$CN  + CO  & $4.2\!\times\! 10^{-11} \left(T/298 \right)^{0.64} $ & Ho95\\ 
\refstepcounter{reaction}R\arabic{reaction} & CH$_2$CN  +  O  &$\!\!\!\rightarrow$ &   HCN   +  HCO   & $4.2\!\times\! 10^{-11} \left(T/298 \right)^{0.64} $ & Ho95 \\

\multicolumn{6}{l}{\bf CH$_3$CN}\\
\refstepcounter{reaction}\label{C2H3+N}R\arabic{reaction} & C$_2$H$_3$  +   N   + M  &$\!\!\!\rightarrow$ &     CH$_3$CN + M  &   $9\!\times\! 10^{-29}$ & Pa96\\ 
     & C$_2$H$_3$  +   N     &$\!\!\!\rightarrow$ &     CH$_3$CN    &   $6.0\!\times\! 10^{-11}$ & Pa96\\  
\refstepcounter{reaction}\label{CH2CN+H}R\arabic{reaction} & CH$_2$CN  +  H  + M  &$\!\!\!\rightarrow$ &      CH$_3$CN  + M      &   $2.9\!\times\! 10^{-29} \left(T/298 \right)^{0.04} $ & note \\  
         & CH$_2$CN  +  H    &$\!\!\!\rightarrow$ &      CH$_3$CN    &   $2.9\!\times\! 10^{-10} \left(T/298 \right)^{0.04} $ & note \\  
\refstepcounter{reaction}\label{1CH2+HCN}R\arabic{reaction} & HCN   +   $^1$CH$_2$ + M  &$\!\!\!\rightarrow$ &    CH$_3$CN   + M  &   $1.0\!\times\! 10^{-29} \left(T/298 \right)^{1.2} e^{+500/T}$ & note \\  
    & HCN   +   $^1$CH$_2$   &$\!\!\!\rightarrow$ &    CH$_3$CN    &   $1.0\!\times\! 10^{-12} \left(T/298 \right)^{1.2} e^{+500/T}$ & \\   
\refstepcounter{reaction}R\arabic{reaction} & CH$_2$N$ _2$   + HCN   &$\!\!\!\rightarrow$ &     CH$_3$CN   + N$ _2$    &   $2.0\!\times\! 10^{-11}  e^{-4500/T}$ & assumed \\  
\refstepcounter{reaction}R\arabic{reaction} & CH$_3$CN  +  O     &$\!\!\!\rightarrow$ &     CH$_3$  +    NCO   &   $5.7\!\times\! 10^{-11} e^{-4560/T}$ & Su10\\   
%\refstepcounter{reaction}R\arabic{reaction} & CH$_3$CN  +  OH    &$\!\!\!\rightarrow$ &     CH$_4$    +  NCO   &   $4.2\!\times\! 10^{-13} \left(T/298 \right)^{0.7} e^{-900/T}$ & \\   
\refstepcounter{reaction}R\arabic{reaction} & CH$_3$CN  +  OH    &$\!\!\!\rightarrow$ &     CH$_2$CN    +  H$_2$O   &   $8.1\!\times\! 10^{-13}  e^{-1080/T}$ & At01 \\   % products
\refstepcounter{reaction}\label{CH3CN+H}R\arabic{reaction} & CH$_3$CN  +  H    &$\!\!\!\rightarrow$ &      HCN   +   CH$_3$   &   $1.4\!\times\! 10^{-11} \left(T/298 \right)^{2.0} e^{-4300/T}$ & note \\  
\refstepcounter{reaction}R\arabic{reaction} & CH$_3$CN  +  H    &$\!\!\!\rightarrow$ &      CH$_2$CN  +  H$_2$   &   $4.0\!\times\! 10^{-12} \left(T/298 \right)^{2.0} e^{-4300/T}$ &note \\  

\refstepcounter{reaction}R\arabic{reaction} & CH$_3$CN  +  CN    &$\!\!\!\rightarrow$ &    CH$_2$CN  +  HCN   &   $6.5\!\times\! 10^{-11} e^{-1190/T}$ & Za89\\  % products - rate looks like H-abstraction

\multicolumn{6}{l}{\bf HCCCN}\\
\refstepcounter{reaction}R\arabic{reaction} & C$_2$H$_2$   +  CN  &$\!\!\!\rightarrow$ &    HCCCN   +    H  &   $2.3\!\times\! 10^{-10} $ & Ga07\\
\refstepcounter{reaction}R\arabic{reaction} & C$_2$H   +  HCN   &$\!\!\!\rightarrow$ &  HCCCN   +   H  &   $5.3\!\times\! 10^{-12}e^{-770/T}$ & Ho97\\    
\refstepcounter{reaction}R\arabic{reaction} & C$_2$H  + CN + M  &$\!\!\!\rightarrow$ &    HCCCN   +   M    &   $1.0\!\times\! 10^{-30} $ & Mo96\\
          & C$_2$H  + CN   &$\!\!\!\rightarrow$ &    HCCCN       &   $1.3\!\times\! 10^{-11} $ & Mo965\\

\multicolumn{6}{l}{\bf O($^1$D)}\\
 \refstepcounter{reaction}R\arabic{reaction}  & O($^1$D) + H$_2$       &$\!\!\!\rightarrow$ &  OH      + H     & $  1.5\!\times\! 10^{-10}$ & Ba92\\
 \refstepcounter{reaction}R\arabic{reaction}  & O($^1$D) + CH$_4$      &$\!\!\!\rightarrow$ &  CH$_3$   + OH   & $  1.2\!\times\! 10^{-10}$ & Sa03\\
 \refstepcounter{reaction}R\arabic{reaction}  & O($^1$D) + CH$_4$      &$\!\!\!\rightarrow$ &  H$_2$COH   + H        & $  3.0\!\times\! 10^{-11}$ & Sa03\\
 \refstepcounter{reaction}R\arabic{reaction}  & O($^1$D)  + CO          &$\!\!\!\rightarrow$ &  O    + CO    & $  4.7\!\times\! 10^{-11}e^{+60/T}$ & note \\
 \refstepcounter{reaction}R\arabic{reaction}  & O($^1$D)  + H$_2$O      &$\!\!\!\rightarrow$ &  OH    + OH     & $  2.2\!\times\! 10^{-10}$ & Sa03\\
 \refstepcounter{reaction}R\arabic{reaction}  & O($^1$D)  + O$_2$       &$\!\!\!\rightarrow$ &  O    + O$_2$  & $  3.2\!\times\! 10^{-11}e^{+70/T}$ & Sa03\\
 \refstepcounter{reaction}R\arabic{reaction} & O($^1$D) + O$_3$   &$\!\!\!\rightarrow$ &  O$_2$   +  O$_2$   & $ 1.2\!\times\! 10^{-10} $   & De97 \\  
 \refstepcounter{reaction}R\arabic{reaction}  & O($^1$D)   + N$_2$       &$\!\!\!\rightarrow$ &  O      + N$_2$    & $  1.8\!\times\! 10^{-11}e^{+110/T}$ & Sa03\\
  \refstepcounter{reaction}R\arabic{reaction} &  O($^1$D)  +     N$_2$ + M &$\!\!\!\rightarrow$ &   N$_2$O + M & $ 2.8\!\times\! 10^{-36}  $   &  At04 \\     
          & O($^1$D)  +     N$_2$   &$\!\!\!\rightarrow$ &   N$_2$O  & $ 3.4\!\times\! 10^{-16} $    & At04 \\  
 \refstepcounter{reaction}R\arabic{reaction}  & O($^1$D)    + CO$_2$       &$\!\!\!\rightarrow$ &  O     + CO$_2$    & $  7.4\!\times\! 10^{-11}e^{+120/T}$ & Sa03\\
\refstepcounter{reaction}\label{O1D+CO}R\arabic{reaction}  & O($^1$D)  + CO  + M &$\!\!\!\rightarrow$ &  CO$_2$ + M  & $  1.0\!\times\! 10^{-35}$ & note \\
            & O($^1$D)       + CO          &$\!\!\!\rightarrow$ &  CO$_2$      & $  1.0\!\times\! 10^{-15}$ & note \\
 \refstepcounter{reaction}R\arabic{reaction} & O($^1$D) + N$_2$O   &$\!\!\!\rightarrow$ & N$_2$  +  O$_2$   & $ 4.9\!\times\! 10^{-11} $ & De97 \\  
 \refstepcounter{reaction}R\arabic{reaction} & O($^1$D) + N$_2$O   &$\!\!\!\rightarrow$ & NO  +  NO   & $ 6.7\!\times\! 10^{-11} $ & De97 \\  

\multicolumn{6}{l}{\bf N($^2$D)}\\
\refstepcounter{reaction}R\arabic{reaction} & N($^2$D)    +     H$_2$   &$\!\!\!\rightarrow$ &   NH     +     H    &   $2.2\!\times\! 10^{-12} $ & He99 \\    
\refstepcounter{reaction}R\arabic{reaction} & N($^2$D)    +    CH$_4$  &$\!\!\!\rightarrow$ &   H$_2$CN   +    H$_2$   &   $2.0\!\times\! 10^{-12} $ &  He99 \\   
\refstepcounter{reaction}R\arabic{reaction} & N($^2$D)    +     CH$_4$  &$\!\!\!\rightarrow$ &   NH     +     CH$_3$   &   $1.0\!\times\! 10^{-12} $ & He99  \\  
\refstepcounter{reaction}R\arabic{reaction} & N($^2$D)    +     H$_2$O  &$\!\!\!\rightarrow$ &   HNO   +  H    &   $5.0\!\times\! 10^{-11} $ &  He99 \\  
\refstepcounter{reaction}R\arabic{reaction} & N($^2$D)   +      CO$_2$  &$\!\!\!\rightarrow$ &   NO     +     CO   &   $3.6\!\times\! 10^{-13} $ &  He99 \\   
\refstepcounter{reaction}R\arabic{reaction} & N($^2$D)    +     CO  &$\!\!\!\rightarrow$ &    N     +    CO   &   $1.9\!\times\! 10^{-12} $ &  He99 \\   
\refstepcounter{reaction}R\arabic{reaction} & N($^2$D)    +     N$_2$   &$\!\!\!\rightarrow$ &   N    +     N$_2$ &   $1.7\!\times\! 10^{-14} $ &  He99 \\   
\refstepcounter{reaction}\label{N2D+HCN}R\arabic{reaction} & N($^2$D)    +     HCN   &$\!\!\!\rightarrow$ &   CH    +     N$_2$ &   $3.0\!\times\! 10^{-12} $ &  assumed \\   
\refstepcounter{reaction}\label{N2D+C2H2}R\arabic{reaction} & N($^2$D)  +  C$_2$H$_2$  &$\!\!\!\rightarrow$ &   HCN + CH  &   $1.0\!\times\! 10^{-10}e^{-300/T} $ & He99, Nu19 \\   % products
\refstepcounter{reaction}\label{N2D+C2H4}R\arabic{reaction} & N($^2$D)  +  C$_2$H$_4$  &$\!\!\!\rightarrow$ &   CH$_2$CN + H$_2$ &   $1.0\!\times\! 10^{-10}$ & He99, Sa99 \\   



 \multicolumn{6}{l}{\bf Sulfur}\\
 \refstepcounter{reaction}\label{RS2}R\arabic{reaction}   & S            + S            + M & $\!\!\!\rightarrow$ &  S$_2$        + M &$  2.0\!\times\! 10^{-33} e^{ 206/T}$ & Du08\\
             & S            + S           &$\!\!\!\rightarrow$&  S$_2$         &$  2.3\!\times\! 10^{-14} e^{ 415/T}$ &  Du08\\
 \refstepcounter{reaction}\label{RS3}R\arabic{reaction}   & S            + S$_2$        + M & $\!\!\!\rightarrow$ &  S$_3$        + M &$  2.0\!\times\! 10^{-32} \left(T/298 \right)^{-2.0}$ & note \\
             & S            + S$_2$       &$\!\!\!\rightarrow$&  S$_3$      &$  1.0\!\times\! 10^{-11}$ & note \\
\refstepcounter{reaction}\label{RS+S3}R\arabic{reaction}  & S            + S$_3$       &$\!\!\!\rightarrow$ &  S$_2$        + S$_2$      & $  1.0\!\times\! 10^{-11} e^{ -1200/T}$ & note \\
 \refstepcounter{reaction}R\arabic{reaction}   & S     + S$_3$  + M & $\!\!\!\rightarrow$ &  S$_4$  + M &$  2.0\!\times\! 10^{-31} \left(T/298 \right)^{-2.0}$ & assumed \\
           & S    + S$_3$       &$\!\!\!\rightarrow$&  S$_4$   &$  1.0\!\times\! 10^{-11}$ & assumed \\
\refstepcounter{reaction}\label{RS2+S2}R\arabic{reaction}  & S$_2$    + S$_2$    + M & $\!\!\!\rightarrow$ &  S$_4$        + M &$  2.0\!\times\! 10^{-31} \left(T/298 \right)^{-2.0}$ & Ni79 \\
            & S$_2$        + S$_2$       &$\!\!\!\rightarrow$&  S$_4$     &$  1.0\!\times\! 10^{-11}$ & assumed \\
\refstepcounter{reaction}\label{RS+S4}R\arabic{reaction}  & S    + S$_4$     &$\!\!\!\rightarrow$ &  S$_3$   + S$_2$      & $  1.0\!\times\! 10^{-11} e^{ -1000/T}$ & note \\
\refstepcounter{reaction}\label{RS8}R\arabic{reaction}  & S$_4$        + S$_4$        + M & $\!\!\!\rightarrow$ &  S$_8$        + M &$  1.0\!\times\! 10^{-29} \left(T/298 \right)^{-2.0}$ & note \\
            & S$_4$        + S$_4$       &$\!\!\!\rightarrow$&  S$_8$   &$  1.0\!\times\! 10^{-11}$ & note \\

\refstepcounter{reaction}\label{RS8star}R\arabic{reaction}  & S$_4$   + S$_4$   + M & $\!\!\!\rightarrow$ &  S$_8^{\ast}$        + M &$  1.0\!\times\! 10^{-28} \left(T/298 \right)^{-2.0} $ &note  \\
            & S$_4$        + S$_4$       &$\!\!\!\rightarrow$&  S$_8^{\ast}$   &$  1.0\!\times\! 10^{-11}$ & note \\
\refstepcounter{reaction}R\arabic{reaction}  & S$_8^{\ast}$        + M & $\!\!\!\rightarrow$ &  S$_8$        + M &$  1.0\!\times\! 10^{-11}  e^{ -1000/T}$ & note \\
%            & S$_8^{\ast}$        + M       &$\!\!\!\rightarrow$&  S$_8$  + M &$  0.0\!\times\! 10^{08}e^{  -1000/T}$ & \\  % this looks pretty fucked

\multicolumn{6}{l}{\bf HS}\\
 \refstepcounter{reaction}R\arabic{reaction}   & S   + H      + M & $\!\!\!\rightarrow$ &  HS  + M & $  1.0\!\times\! 10^{-32} $  & assumed \\
        & S      + H   &$\!\!\!\rightarrow$&  HS         &$  1.0\!\times\! 10^{-12}$   &  assumed  \\
 \refstepcounter{reaction}\label{RS+HS}R\arabic{reaction}  & S  + HS    &$\!\!\!\rightarrow$ &  S$_2$    + H  & $  2.0\!\times\! 10^{-11}\left(T/298\right)^{0.7} e^{-300/T}$ & note  \\

% \refstepcounter{reaction}\label{RS+H2}R\arabic{reaction}  & S            + H$_2$      &$\!\!\!\rightarrow$ &  H        + HS      & $  5.3\!\times\! 10^{-10}\left(T/298 \right)^{0.95} e^{-9920/T}$ & Woi95, Sh98\\

\refstepcounter{reaction}\label{RH+HS}R\arabic{reaction}  & H            +  HS      &$\!\!\!\rightarrow$ &  S        + H$_2$      & $  2.0\!\times\! 10^{-11}e^{-2400/T} $ & note\\

% \refstepcounter{reaction}\label{RHS+HS}R\arabic{reaction}   & HS  + HS    &$\!\!\!\rightarrow$ &  S$_2$  + H$_2$    & $  1.3\!\times\! 10^{-11} e^{-20000/T}$ & note\\

 \refstepcounter{reaction}\label{RH+S3}R\arabic{reaction}   & H      + S$_3$       &$\!\!\!\rightarrow$ &  HS    + S$_2$   & $  5.0\!\times\! 10^{-11} e^{  -1200/T}$ & note\\
 \refstepcounter{reaction}\label{RH+S4}R\arabic{reaction}    & H  + S$_4$   &$\!\!\!\rightarrow$ &  HS   + S$_3$  & $  5.0\!\times\! 10^{-11} e^{  -1000/T}$ & note\\

 \refstepcounter{reaction}R\arabic{reaction}   & O + HS     & $\!\!\!\rightarrow$ &  OH   + S     & $  1.7\!\times\! 10^{-11} \left(T/298\right)^{ 0.67}e^{  -956/T}$ & Sc73\\
 \refstepcounter{reaction}\label{ROH+HS}R\arabic{reaction}   & HS    + OH    &$\!\!\!\rightarrow$ &  H$_2$O    + S  & $  6.0\!\times\! 10^{-12} e^{  -81/T}$ & note \\
 \refstepcounter{reaction}R\arabic{reaction}   & S   + CH  & $\!\!\!\rightarrow$ &  HS  + C   & $  1.7\!\times\! 10^{-11} \left(T/298\right)^{ 0.50}e^{ -4000/T}$ & Mi97\\
 \refstepcounter{reaction}R\arabic{reaction}   & S    + NH    & $\!\!\!\rightarrow$ &  HS    + N    & $  1.7\!\times\! 10^{-11} \left(T/298\right)^{ 0.50}e^{ -4000/T}$ & Mi97\\
 \refstepcounter{reaction}R\arabic{reaction}  & NH$_2$       + HS          &$\!\!\!\rightarrow$ &  NH$_3$   + S   & $  5.0\!\times\! 10^{-12}e^{  -500/T}$ & Mo96\\
 \refstepcounter{reaction}R\arabic{reaction}  & HS     + CH$_2$      &$\!\!\!\rightarrow$ &  S   + CH$_3$   & $  4.0\!\times\! 10^{-12}e^{  -500/T}$ & Mo96\\ % divided by five
 \refstepcounter{reaction}\label{RCH3SH}R\arabic{reaction}  & HS   + CH$_3$    &$\!\!\!\rightarrow$ &  S   + CH$_4$  & $  4.0\!\times\! 10^{-11}e^{  -500/T}$ & Sh85\\ % divided by five
 \refstepcounter{reaction}R\arabic{reaction}  & S  + HCO    &$\!\!\!\rightarrow$ &  HS  + CO   & $  6.0\!\times\! 10^{-11}$ & Mo96\\
% \refstepcounter{reaction}R\arabic{reaction} & S + HO$_2$   &$\!\!\!\rightarrow$ &  HS  +   O$_2$   & $ 5.0\!\times\! 10^{-12} $  & ?? \\  



\multicolumn{6}{l}{\bf H$_2$S}\\
\refstepcounter{reaction}\label{RH2S}R\arabic{reaction}   & H    + HS   + M & $\!\!\!\rightarrow$ &  H$_2$S    + M &$  1.4\!\times\! 10^{-31} \left(T/298 \right)^{-2.52}e^{ +488/T}$ & note  \\
             & H    + HS   &$\!\!\!\rightarrow$&  H$_2$S   &$  1.0\!\times\! 10^{-10}$ & note \\
 \refstepcounter{reaction}\label{RHSH}R\arabic{reaction}   & S            + H$_2$    + M & $\!\!\!\rightarrow$ &  H$_2$S       + M &$  1.4\!\times\! 10^{-31} \left(T/298 \right)^{-1.9}e^{ -8140/T}$ & note \\
             & S            + H$_2$          &$\!\!\!\rightarrow$&  H$_2$S       &$  1.0\!\times\! 10^{-11}$ & note \\

 \refstepcounter{reaction}\label{RHS+HS}R\arabic{reaction}    & HS   + HS   & $\!\!\!\rightarrow$ &  H$_2$S  + S  & $  8.5\!\times\! 10^{-12} \left(T/298\right)^{ 0.20}$ & rev Sh96\\
% \refstepcounter{reaction}\label{R289}R\arabic{reaction}   & H$_2$S       + S & $\!\!\!\rightarrow$ &  HS + HS  & $  1.4\!\times\! 10^{-10}e^{ -3720/T}$ & Sh96\\
 \refstepcounter{reaction}\label{RH+H2S}R\arabic{reaction}   & H     + H$_2$S      & $\!\!\!\rightarrow$ &  HS           + H$_2$        & $  3.7\!\times\! 10^{-12} \left(T/298\right)^{ 1.94}e^{  -455/T}$ & Pe99\\
 \refstepcounter{reaction}R\arabic{reaction}   & O            + H$_2$S      &$\!\!\!\rightarrow$ &  HS           + OH       & $  9.2\!\times\! 10^{-12} e^{ -1800/T}$ & De97\\
 \refstepcounter{reaction}R\arabic{reaction}   & OH     + H$_2$S      &$\!\!\!\rightarrow$ &  H$_2$O       + HS      & $  6.1\!\times\! 10^{-12} e^{  -81/T}$ & At04 \\
 \refstepcounter{reaction}R\arabic{reaction}  & HS           + HCO         &$\!\!\!\rightarrow$ &  H$_2$S       + CO       & $  5.0\!\times\! 10^{-11}$ & assumed \\
 \refstepcounter{reaction}\label{RCH2+H2S}R\arabic{reaction}   & CH$_2$      + H$_2$S      &$\!\!\!\rightarrow$ &  CH$_3$       + HS      & $  2.5\!\times\! 10^{-11} e^{  -746/T}$ & Da95\\

% \refstepcounter{reaction}R\arabic{reaction}    & HS           + CH$_4$      &$\!\!\!\rightarrow$ &  CH$_3$       + H$_2$S   & $  3.1\!\times\! 10^{-12} e^{ -8260/T}$ & rev Pe88\\
 \refstepcounter{reaction}R\arabic{reaction}   & H$_2$S   + CH$_3$      &$\!\!\!\rightarrow$ &  HS + CH$_4$   & $  2.1\!\times\! 10^{-13} e^{ -1160/T}$ & Pe88\\
 \refstepcounter{reaction}R\arabic{reaction} & HS + HO$_2$   &$\!\!\!\rightarrow$ &  H$_2$S  +   O$_2$   & $ 1.0\!\times\! 10^{-11} $  & assumed \\  

\multicolumn{6}{l}{\bf HS$_4$}\\
 \refstepcounter{reaction}\label{RHS4}R\arabic{reaction}& H  + S$_4$ + M  & $\!\!\!\rightarrow$ &  HS$_4$ + M & $  1.0\!\times\! 10^{-31} \left(T/298\right)^{-1.0}$ & note\\
            & H     + S$_4$     &$\!\!\!\rightarrow$&  HS$_4$      &$  5.0\!\times\! 10^{-11}$   &  note \\

 \refstepcounter{reaction}\label{RH+S8}R\arabic{reaction} & H  + S$_8$ & $\!\!\!\rightarrow$ &  HS$_4$  + S$_4$  & $  1.0\!\times\! 10^{-11} e^{  -2200/T}$ & note\\
 \refstepcounter{reaction}\label{RH+S8L}R\arabic{reaction}   & H  + S$_8^{\ast}$  & $\!\!\!\rightarrow$ &  HS$_4$  + S$_4$ & $  1.0\!\times\! 10^{-10} e^{ -1000/T}$ & note\\
 \refstepcounter{reaction}\label{RH+HS4}R\arabic{reaction}   & H    + HS$_4$    & $\!\!\!\rightarrow$ &  H$_2$  + S$_4$ & $  1.0\!\times\! 10^{-10} e^{  -1250/T}$ & note\\
 \refstepcounter{reaction}\label{RHS4+H}R\arabic{reaction}   & H    + HS$_4$    & $\!\!\!\rightarrow$ &  H$_2$S  + S$_3$    & $  7.3\!\times\! 10^{-11} e^{  -3200/T}$ & note\\

\refstepcounter{reaction}\label{ROH+HS4}R\arabic{reaction}   & OH      + HS$_4$    & $\!\!\!\rightarrow$ &  H$_2$O  + S$_4$   & $  5.0\!\times\! 10^{-11} $ & note\\
 \refstepcounter{reaction}\label{RHS+HS4}R\arabic{reaction}   & HS     + HS$_4$   & $\!\!\!\rightarrow$ &  H$_2$S  + S$_4$     & $ 9.0\!\times\! 10^{-11}\left(T/298\right)^{0.7} e^{ -2100/T}$ & note\\
 
 \refstepcounter{reaction}\label{RCH3+HS4}R\arabic{reaction}   & CH$_3$   + HS$_4$   & $\!\!\!\rightarrow$ &  CH$_4$  + S$_4$  & $ 9.0\!\times\! 10^{-11}\left(T/298\right)^{0.7} e^{ -2100/T}$ & note\\
 \refstepcounter{reaction}\label{RNH2+HS4}R\arabic{reaction}   & NH$_2$    + HS$_4$   & $\!\!\!\rightarrow$ &  NH$_3$  + S$_4$  & $ 9.0\!\times\! 10^{-11}\left(T/298\right)^{0.7} e^{ -2100/T}$ & note\\
\refstepcounter{reaction}R\arabic{reaction}   & O   + HS$_4$   & $\!\!\!\rightarrow$ &  OH  + S$_4$     & $ 9.0\!\times\! 10^{-11}\left(T/298\right)^{0.7} e^{ -2100/T}$ & note\\
\refstepcounter{reaction}\label{RS+HS4}R\arabic{reaction}   & S   + HS$_4$   & $\!\!\!\rightarrow$ &  HS  + S$_4$     & $ 9.0\!\times\! 10^{-11}\left(T/298\right)^{0.7} e^{ -2100/T}$ & note\\
\refstepcounter{reaction}\label{RHS4+HS4}R\arabic{reaction}   & HS$_4$  + HS$_4$   & $\!\!\!\rightarrow$ &  H$_2$  + S$_8$   & $ 1.0\!\times\! 10^{-11} e^{  -1250/T}$ & note\\

\multicolumn{6}{l}{\bf SO, SO$_2$}\\
 \refstepcounter{reaction}\label{RO+HS}R\arabic{reaction}  & O            + HS          &$\!\!\!\rightarrow$ &  SO           + H             & $  7.0\!\times\! 10^{-11}$ & Sa03\\
 \refstepcounter{reaction}R\arabic{reaction}  & S            + OH          &$\!\!\!\rightarrow$ &  H            + SO            & $  6.6\!\times\! 10^{-11}$ & JPL19\\
 
  \refstepcounter{reaction}R\arabic{reaction}  &  S$_2$ + O   &$\!\!\!\rightarrow$ &  SO           + S          & $  2.0\!\times\! 10^{-11}e^{  -84/T}$ & Cra87\\

 \refstepcounter{reaction}R\arabic{reaction}   & S            + O$_2$       & $\!\!\!\rightarrow$ &  SO           + O         & $  1.5\!\times\! 10^{-13} \left(T/298\right)^{ 2.11}e^{  +730/T}$ & Lu04\\
 \refstepcounter{reaction}R\arabic{reaction}  & S$_3$        + O           &$\!\!\!\rightarrow$ &  S$_2$        + SO       & $  5.0\!\times\! 10^{-11}e^{  -1000/T}$ & Mos02\\
 \refstepcounter{reaction}R\arabic{reaction}   & S$_4$        + O           &$\!\!\!\rightarrow$ &  S$_3$        + SO      & $ 5.0\!\times\! 10^{-11} e^{  -1000/T}$ & Mos02\\
  \refstepcounter{reaction}\label{RN+SO}R\arabic{reaction}   & N  + SO   & $\!\!\!\rightarrow$ &  NO   + S    & $  4.5\!\times\! 10^{-12} \left(T/298\right)^{ 1.0}e^{ -3270/T}$ & note \\
 \refstepcounter{reaction}\label{RCO+SO}R\arabic{reaction} & SO    +   CO    &$\!\!\!\rightarrow$ &     CO$_2$  +   S    &   $3.65\!\times\! 10^{-13}  e^{-10000/T}$ & note \\  

\refstepcounter{reaction}R\arabic{reaction}  & SO + O + M & $\!\!\!\rightarrow$ &  SO$_2$ + M  & $  4.8\!\times\! 10^{-31} \left(T/298\right)^{-2.17}$ & Lu03\\
          & SO + O   & $\!\!\!\rightarrow$ &  SO$_2$    & $  5.3\!\times\! 10^{-11}$ & Si88\\
 \refstepcounter{reaction}\label{RSO+O2}R\arabic{reaction}   & O$_2$  + SO  &$\!\!\!\rightarrow$ &  SO$_2$   + O  & $  9.4\!\times\! 10^{-14}\left(T/298\right)^{1.3} e^{ -2210/T}$ & note\\

 \refstepcounter{reaction}R\arabic{reaction}   & SO  + SO    &$\!\!\!\rightarrow$ &  S     + SO$_2$       &  $  1.0\!\times\! 10^{-12}e^{ -1700/T}$ & Ma83\\ 
 
 \refstepcounter{reaction}R\arabic{reaction}   & OH           + SO          & $\!\!\!\rightarrow$ &  SO$_2$       + H      & $  8.6\!\times\! 10^{-11} \left(T/298 \right)^{-1.35}$ & Bl00,Bl06,Ba07\\
%  \refstepcounter{reaction}R\arabic{reaction} & S + HO$_2$   &$\!\!\!\rightarrow$ &  SO  +   OH   & $ 5.0\!\times\! 10^{-12} $  & ?? \\  
 \refstepcounter{reaction}R\arabic{reaction} & S  +  O$_3$  &$\!\!\!\rightarrow$ &  SO   +  O$_2$   & $ 1.2\!\times\! 10^{-11}  $  & At04\\  
 \refstepcounter{reaction}R\arabic{reaction} & SO  +  O$_3$  &$\!\!\!\rightarrow$ &  SO$_2$   +  O$_2$   & $ 4.5\!\times\! 10^{-12} e^{ -1170/T} $  & At04 \\  
\refstepcounter{reaction}R\arabic{reaction} & CO    +   SO$_2$   &$\!\!\!\rightarrow$ &     CO$_2$   +   SO    &   $4.5\!\times\! 10^{-12}  e^{-24200/T}$ & Ba67\\   

\multicolumn{6}{l}{\bf HSO}\\
 \refstepcounter{reaction}\label{RHSO}R\arabic{reaction}   & H    + SO +M        &$\!\!\!\rightarrow$&  HSO    + M &$  5.5\!\times\! 10^{-32}\left(T/298 \right)^{-1.8}$ & note\\
           & H            + SO          &$\!\!\!\rightarrow$&  HSO        &$  7.5\!\times\! 10^{-11}$ & note \\
 \refstepcounter{reaction}\label{RHSO+H}R\arabic{reaction}  & HSO          + H           &$\!\!\!\rightarrow$ &  HS           + OH               & $  7.6\!\times\! 10^{-11}\left(T/298 \right)^{0.2}$ & note\\% assumed same as H + HO2 Ba92
 \refstepcounter{reaction}R\arabic{reaction}   & HSO          + H           &$\!\!\!\rightarrow$ &  H$_2$        + SO             & $  7.1\!\times\! 10^{-11}e^{ -710/T} $ & note\\ % assumed same as H + HO2 Ba92
 \refstepcounter{reaction}R\arabic{reaction}   & HSO          + H           &$\!\!\!\rightarrow$ &  H$_2$O      + S            & $  5.0\!\times\! 10^{-11} e^{  -870/T}$ & note\\% assumed same as H + HO2 Ba92
 \refstepcounter{reaction}R\arabic{reaction}  & H$_2$S      + O  &$\!\!\!\rightarrow$ &  H + HSO      & $  5.0\!\times\! 10^{-11}e^{  -3850/T}$ & Tsu94 \\

 \refstepcounter{reaction}R\arabic{reaction}  & HSO    + OH          &$\!\!\!\rightarrow$ &  H$_2$O       + SO          & $  3.0\!\times\! 10^{-11}e^{  -300/T}$ & assumed \\
 \refstepcounter{reaction}R\arabic{reaction}  & HSO     + HS          &$\!\!\!\rightarrow$ &  H$_2$S       + SO           & $  1.0\!\times\! 10^{-11}e^{  -500/T}$ & assumed\\
 \refstepcounter{reaction}R\arabic{reaction}  & HSO     + O           &$\!\!\!\rightarrow$ &  OH      + SO                & $  1.0\!\times\! 10^{-11}e^{  -500/T}$ & assumed\\
 
% \refstepcounter{reaction}R\arabic{reaction}  & HSO    + O           &$\!\!\!\rightarrow$ &  O$_2$    + HS              & $  1.0\!\times\! 10^{-11}e^{  -500/T}$ & \\
  \refstepcounter{reaction}R\arabic{reaction}  & HSO    + O           &$\!\!\!\rightarrow$ &  SO$_2$    + H              & $  1.0\!\times\! 10^{-11}e^{  -500/T}$ & assumed\\
 
 \refstepcounter{reaction}R\arabic{reaction}  & HSO     + S           &$\!\!\!\rightarrow$ &  HS     + SO      & $  1.0\!\times\! 10^{-11}e^{  -500/T}$ & assumed\\
 \refstepcounter{reaction}R\arabic{reaction}  & SO        + HCO         &$\!\!\!\rightarrow$ &  HSO          + CO          & $  3.0\!\times\! 10^{-11}$ & assumed \\
% \refstepcounter{reaction}R\arabic{reaction} & SO + HO$_2$   &$\!\!\!\rightarrow$ &  HSO +   O$_2$   & $ 0.0 $  & best \\  
%  \refstepcounter{reaction}R\arabic{reaction} & H$_2$S + HO$_2$   &$\!\!\!\rightarrow$ &  H$_2$O +   HSO   & $ 0.0 $  & \\  
 \refstepcounter{reaction}R\arabic{reaction} & HS  +  O$_3$  &$\!\!\!\rightarrow$ &  HSO   +  O$_2$   & $ 9.5\!\times\! 10^{-12}  e^{-280/T} $  & Lee1994\\  

\multicolumn{6}{l}{\bf OCS}\\
% \refstepcounter{reaction}\label{OCS}R\arabic{reaction}   & S     + CO +M   &$\!\!\!\rightarrow$&  OCS   + M &$  1.76\!\times\! 10^{-32}\left(T/298 \right)^{-2.42}$ & \\ % reverse of our own fit to Woi95 and Oy94
%           & S     + CO     &$\!\!\!\rightarrow$&  OCS        &$  1.5\!\times\! 10^{-10}\left(T/298 \right)^{-4.4} e^{-1200/T}$& ?? \\ 
 \refstepcounter{reaction}\label{RS+CO}R\arabic{reaction}   & S     + CO +M   &$\!\!\!\rightarrow$&  OCS   + M &$  1.0\!\times\! 10^{-32}\left(T/298\right)^{-2.0}e^{-1000/T}$ & note \\ 
           & S     + CO     &$\!\!\!\rightarrow$&  OCS        &$  2.0\!\times\! 10^{-12}\left(T/298\right)^{-2.0}e^{-1000/T}$& note \\ 
  \refstepcounter{reaction}R\arabic{reaction}   & O   + OCS         &$\!\!\!\rightarrow$ &  CO           + SO         & $  6.37\!\times\! 10^{-11} e^{ -2465/T}$ & Si88\\
 \refstepcounter{reaction}R\arabic{reaction}   & HS           + CO          &$\!\!\!\rightarrow$ &  OCS          + H            & $  4.15\!\times\! 10^{-14} e^{ -7660/T}$ & Ku95\\
 \refstepcounter{reaction}R\arabic{reaction}   & OCS  + S   & $\!\!\!\rightarrow$ &  CO    + S$_2$    & $  1.5\!\times\! 10^{-13} \left(T/298\right)^{ 2.57}e^{ -1180/T}$ & Lu06\\
 \refstepcounter{reaction}R\arabic{reaction}   & O            + OCS         &$\!\!\!\rightarrow$ &  S            + CO$_2$     & $  8.3\!\times\! 10^{-11} e^{ -5530/T}$ & Si88\\
  \refstepcounter{reaction}R\arabic{reaction}   & OCS          + OH          &$\!\!\!\rightarrow$ &  CO$_2$       + HS          & $  1.1\!\times\! 10^{-13} e^{ -1200/T}$ & At04\\
 \refstepcounter{reaction}R\arabic{reaction}  & S            + HCO         &$\!\!\!\rightarrow$ &  OCS          + H            & $  6.0\!\times\! 10^{-11}$ & Mos96\\

 \refstepcounter{reaction}\label{RCO+S3}R\arabic{reaction}  & CO  + S$_3$  &$\!\!\!\rightarrow$ &  S$_2$   + OCS  & $  1.0\!\times\! 10^{-12} e^{-2900/T}$ & note \\
\refstepcounter{reaction}\label{RCO+S4}R\arabic{reaction} & CO    +   S$_4$   &$\!\!\!\rightarrow$ &   S$_3$  +  OCS  &  $1.0\!\times\! 10^{-12}  e^{-2700/T}$ & note \\  

\multicolumn{6}{l}{\bf CS}\\
 % R396   & CS           + M           & $\!\!\!\rightarrow$ &  C            + S +M        & $  2.7\!\times\! 10^{-03} \left(T/298\right)^{-3.52}e^{-85700/T}$ & \\
 \refstepcounter{reaction}R\arabic{reaction}  & S            + CH          &$\!\!\!\rightarrow$ &  CS           + H                                       & $  2.0\!\times\! 10^{-11}$ & assume \\
\refstepcounter{reaction}R\arabic{reaction}  & C            + HS          &$\!\!\!\rightarrow$ &  CS           + H                                       & $  2.0\!\times\! 10^{-11}$ & assume \\
 \refstepcounter{reaction}R\arabic{reaction}   & N            + CS          & $\!\!\!\rightarrow$ &  CN           + S      & $  3.8\!\times\! 10^{-11} \left(T/298\right)^{ 0.50}e^{ -1160/T}$ & Mi97\\
 \refstepcounter{reaction}R\arabic{reaction}   & O      + CS     &$\!\!\!\rightarrow$ &  CO           + S     & $  2.7\!\times\! 10^{-10} e^{  -760/T}$ & At04\\
 \refstepcounter{reaction}R\arabic{reaction}  & C   + SO      &$\!\!\!\rightarrow$ &  CS           + O      & $  5.0\!\times\! 10^{-11}$ & assumed\\
\refstepcounter{reaction}\label{ROH+CS}R\arabic{reaction}  & OH  + CS     &$\!\!\!\rightarrow$ &  OCS    + H   & $  1.0\!\times\! 10^{-12} \left(T/298\right)^{ 1.50}e^{ +250/T}$ & note \\
\refstepcounter{reaction}R\arabic{reaction} & CS  +  O$_3$  &$\!\!\!\rightarrow$ &  OCS   +  O$_2$   & $ 3.0\!\times\! 10^{-16}  $  & At04\\  

%BREAK

\multicolumn{6}{l}{\bf CS$_2$}\\
%\refstepcounter{reaction}R\arabic{reaction}   & CS      + S +M        &$\!\!\!\rightarrow$&  CS$_2$   + M &$  5.2\!\times\! 10^{-29}\left(T/298 \right)^{-4.5}$ & rev-Sa80 \\
%           & CS      + S          &$\!\!\!\rightarrow$&  CS$_2$        &$  5.0\!\times\! 10^{-11}$ & \\
\refstepcounter{reaction}\label{RCS2}R\arabic{reaction}   & CS + S +M  &$\!\!\!\rightarrow$&  CS$_2$   + M &$  1.76\!\times\! 10^{-30}\left(T/298 \right)^{-2.42}$ & Gl15 \\
           & CS      + S          &$\!\!\!\rightarrow$&  CS$_2$        &$  2.0\!\times\! 10^{-08}\left(T/298 \right)^{-4.5}e^{ -1500/T}$ & fit to Gl15 \\
 % R282   & CS$_2$       + M           &$\!\!\!\rightarrow$ &  CS           + S          +M           & $  4.2\!\times\! 10^{-10} e^{-37400/T}$ & Sa80\\
 \refstepcounter{reaction}R\arabic{reaction}   & CS$_2$       + S           &$\!\!\!\rightarrow$ &  CS           + S$_2$     & $  2.8\!\times\! 10^{-10} e^{ -5920/T}$ & Woi95a\\
 \refstepcounter{reaction}\label{RCS+HS}R\arabic{reaction}   & CS           + HS          & $\!\!\!\rightarrow$ &  CS$_2$       + H         & $  1.0\!\times\! 10^{-12} \left(T/298\right)^{ 1.50}e^{+250/T}$ & note \\
 \refstepcounter{reaction}\label{RCS2+O}R\arabic{reaction}   & CS$_2$       + O           &$\!\!\!\rightarrow$ &  CS           + SO        & $  5.3\!\times\! 10^{-11} e^{  -822/T}$ & Si88, Co92\\
 \refstepcounter{reaction}R\arabic{reaction}   & CS$_2$       + O           &$\!\!\!\rightarrow$ &  OCS          + S               & $  4.7\!\times\! 10^{-12} e^{  -822/T}$ & Si88, Co92\\
 \refstepcounter{reaction}\label{R345}R\arabic{reaction}   & CS$_2$       + O           &$\!\!\!\rightarrow$ &  CO           + S$_2$          & $  1.8\!\times\! 10^{-12} e^{  -822/T}$ & Si88, Co92\\
% R288   & CS$_2$       + OH   &$\!\!\!\rightarrow$ &  OCS + HS& $  0.02\!\times\!1.1\!\times\! 10^{-13} e^{ -1200/T}$ & At04, Cox\\
 \refstepcounter{reaction}\label{R346}R\arabic{reaction}   & CS$_2$       + OH    &$\!\!\!\rightarrow$ &  OCS   + HS     & $  0.0 $ & At04\\


\multicolumn{6}{l}{\bf HCS}\\
 \refstepcounter{reaction}\label{RHCS}R\arabic{reaction}   & CS      + H +M        &$\!\!\!\rightarrow$&  HCS   + M &$  1.0\!\times\! 10^{-31}\left(T/298 \right)^{-1.0}$ & assumed \\  % assumed 10X faster than H+CO
           & CS      +  H          &$\!\!\!\rightarrow$&  HCS        &$  1.0\!\times\! 10^{-11} \left(T/298 \right)^{-1.0}$ & assumed \\
 \refstepcounter{reaction}R\arabic{reaction}  & S            + CH$_2$      &$\!\!\!\rightarrow$ &  HCS          + H                       & $  5.0\!\times\! 10^{-11}$ & Mo96\\
%\refstepcounter{reaction}R\arabic{reaction} & S + CH$_2$  &$\!\!\!\rightarrow$& HCS   +  H     & $ 2.0\!\times\! 10^{-11}$ & assumed \\
 \refstepcounter{reaction}\label{RS+CH3}R\arabic{reaction} & S + CH$_3$  &$\!\!\!\rightarrow$& HCS   +  H$_2$    &$ 2.0\!\times\! 10^{-11}$ & note \\
  \refstepcounter{reaction}R\arabic{reaction}  & S       + HCS         &$\!\!\!\rightarrow$ &  CS$_2$       + H             & $  2.0\!\times\! 10^{-11}$ & Mo96\\
 \refstepcounter{reaction}R\arabic{reaction}  & S        + HCS         &$\!\!\!\rightarrow$ &  CS           + HS              & $  2.0\!\times\! 10^{-11}$ &Mo96 \\
 \refstepcounter{reaction}R\arabic{reaction}  & S$_2$      + CH          &$\!\!\!\rightarrow$ &  HCS          + S              & $  4.0\!\times\! 10^{-12}$ & Mo96\\
 \refstepcounter{reaction}R\arabic{reaction}  & HCS         + H           &$\!\!\!\rightarrow$ &  CS           + H$_2$          & $  5.0\!\times\! 10^{-11}$ & assumed \\
 \refstepcounter{reaction}R\arabic{reaction}  & HCS         + CH$_3$      &$\!\!\!\rightarrow$ &  CS      + CH$_4$      & $  2.0\!\times\! 10^{-11}$ & Mo96\\ % divided by five
 \refstepcounter{reaction}R\arabic{reaction}  & HCS    + C$_2$H      &$\!\!\!\rightarrow$ &  CS    + C$_2$H$_2$   & $  2.0\!\times\! 10^{-11}$ & Mo96\\ % divided by five
 \refstepcounter{reaction}R\arabic{reaction}  & HCS     + OH      &$\!\!\!\rightarrow$ &  CS       + H$_2$O    & $  2.0\!\times\! 10^{-11}$ & Mo96\\ % divided by five
 
\multicolumn{6}{l}{\bf SO$_3$}\\
  \refstepcounter{reaction}R\arabic{reaction} &  SO$_2$  +    O + M &$\!\!\!\rightarrow$ &   SO$_3$ + M & $ 4.0\!\times\! 10^{-32}  e^{-1000/T} $   &  At97 \\     
          & SO$_2$  +    O  &$\!\!\!\rightarrow$ &   SO$_3$  & $ 6.1\!\times\! 10^{-13} e^{-850/T} $    & Na05 \\  
\refstepcounter{reaction}R\arabic{reaction} & SO$_2$  + O$_3$   &$\!\!\!\rightarrow$ & SO$_3$ + O$_2$  & $ 3.0\!\times\! 10^{-12}   e^{-7000/T} $ &  De97\\  
\refstepcounter{reaction}R\arabic{reaction} & SO$_3$  + H   &$\!\!\!\rightarrow$ & SO$_2$ + OH  & $ 1.46\!\times\! 10^{-11} \left(T/298 \right)^{1.22}  e^{-1670/T} $ & Hi07 \\  
\refstepcounter{reaction}R\arabic{reaction} & SO$_3$  + O   &$\!\!\!\rightarrow$ & SO$_2$ + O$_2$  & $ 1.06\!\times\! 10^{-13} \left(T/298 \right)^{2.57}  e^{-14700/T} $ &  Hi07 \\  
\refstepcounter{reaction}\label{RSO+SO3}R\arabic{reaction} & SO$_3$  + SO   &$\!\!\!\rightarrow$ & SO$_2$ + SO$_2$  & $ 2.0\!\times\! 10^{-15} $ &  Ch75 \\  

 \multicolumn{6}{l}{\bf HSO$_3$, H$_2$SO$_4$}\\
  \refstepcounter{reaction}\label{RHSO3}R\arabic{reaction} &  SO$_2$  +    OH  + M &$\!\!\!\rightarrow$ &   HSO$_3$ + M & $ 4.0\!\times\! 10^{-31}  \left(T/298 \right)^{-3.3}  $   & At97 \\     
          & SO$_2$  +    OH  &$\!\!\!\rightarrow$ &   HSO$_3$  & $ 1.3\!\times\! 10^{-12} \left(T/298 \right)^{-0.7}  $    & At04 \\  
\refstepcounter{reaction}R\arabic{reaction} & HSO$_3$  + H   &$\!\!\!\rightarrow$ & SO$_3$ + H$_2$  & $ 1.0\!\times\! 10^{-11}   e^{-500/T} $ & assumed \\  
\refstepcounter{reaction}R\arabic{reaction} & HSO$_3$  + O   &$\!\!\!\rightarrow$ & SO$_3$ + OH  & $ 1.0\!\times\! 10^{-11}   e^{-500/T} $ &  assumed \\  
% \refstepcounter{reaction}R\arabic{reaction} & HSO$_3$  + OH   &$\!\!\!\rightarrow$ & SO$_3$ + H$_2$O  & $ 0.0 $ &  \\  
\refstepcounter{reaction}R\arabic{reaction} & HSO$_3$  + O$_2$   &$\!\!\!\rightarrow$ & SO$_3$ + HO$_2$  & $ 1.3\!\times\! 10^{-12}  e^{-330/T} $ &  At04 \\  
  \refstepcounter{reaction}R\arabic{reaction} & HSO$_3$  +  OH  + M &$\!\!\!\rightarrow$ &   H$_2$SO$_4$ + M & $ 4.0\!\times\! 10^{-31}  \left(T/298 \right)^{-3.3}  $   & At97 \\     
          & HSO$_3$  +    OH  &$\!\!\!\rightarrow$ &   H$_2$SO$_4$  & $ 1.3\!\times\! 10^{-12} \left(T/298 \right)^{-0.7}  $    & At04 \\  
  \refstepcounter{reaction}\label{RH2SO4}R\arabic{reaction} &  SO$_3$  +    H$_2$O  + M &$\!\!\!\rightarrow$ &   H$_2$SO$_4$ + M & $ 1.5\!\times\! 10^{-33} \left(T/298 \right)^{1.0} e^{+1500/T}   $   & note \\     
          & SO$_3$  +    H$_2$O   &$\!\!\!\rightarrow$ &    H$_2$SO$_4$   & $ 1.2\!\times\! 10^{-15}  $    &  Re94 \\  



 \multicolumn{6}{l}{\bf Chlorine (incomplete)}\\
 \refstepcounter{reaction}\label{Chlorine}R\arabic{reaction} & Cl  +    Cl + M &$\!\!\!\rightarrow$ &      Cl$_2$ + M & $ 6.0\!\times\! 10^{-34} e^{-910/T}  $   & Ba81  \\     
          & Cl  +    Cl  &$\!\!\!\rightarrow$ &   Cl$_2$  & $ 1.0\!\times\! 10^{-12} $    &  assumed \\  
% \refstepcounter{reaction}R\arabic{reaction} & HCl  +   H    &$\!\!\!\rightarrow$ &    H$_2$   +   Cl   & $ 1.3\!\times\! 10^{-11} e^{-1710/T}$  & Ba81  REVERSE \\  
 \refstepcounter{reaction}R\arabic{reaction} & Cl$_2$  +   H    &$\!\!\!\rightarrow$ &    Cl   +   HCl   & $ 1.43\!\times\! 10^{-10} e^{-590/T}$  & Ba81 \\  
 \refstepcounter{reaction}R\arabic{reaction} & Cl$_2$  +   O    &$\!\!\!\rightarrow$ &    ClO  +   Cl   & $ 4.2\!\times\! 10^{-12} e^{-1370/T}$  & Ba81 \\  
 \refstepcounter{reaction}R\arabic{reaction} & Cl + HO$_2$    &$\!\!\!\rightarrow$ &    ClO  +  OH   & $ 6.3\!\times\! 10^{-11} e^{-570/T}$  & At07 \\  
\refstepcounter{reaction}R\arabic{reaction} & ClO + O    &$\!\!\!\rightarrow$ &    Cl  +  O$_2$   & $ 2.5\!\times\! 10^{-11} e^{+110/T}$  & At07 \\   
% \refstepcounter{reaction}R\arabic{reaction} & ClO + OH    &$\!\!\!\rightarrow$ &    Cl  +  HO$_2$   & $ 1.9\!\times\! 10^{-11} $  & At07   UPDATE\\   
\refstepcounter{reaction}R\arabic{reaction} & ClO + OH    &$\!\!\!\rightarrow$ &    HCl  +  O$_2$   & $ 1.2\!\times\! 10^{-12} $  & At07 \\   
 \refstepcounter{reaction}R\arabic{reaction} & Cl + H$_2$    &$\!\!\!\rightarrow$ &    HCl  +  H   & $ 3.9\!\times\! 10^{-11} e^{-2310/T}$  & At07  \\  
 \refstepcounter{reaction}R\arabic{reaction} & HCl  +   OH    &$\!\!\!\rightarrow$ &    H$_2$O   +   Cl   & $ 1.7\!\times\! 10^{-12} e^{-230/T}$  & At07 \\  
 \refstepcounter{reaction}R\arabic{reaction} & Cl + CH$_4$    &$\!\!\!\rightarrow$ &    HCl + CH$_3$   & $ 2.5\!\times\! 10^{-12} \left(T/298 \right)^{1.27} e^{-940/T}$  & Mi01\\  
 \refstepcounter{reaction}R\arabic{reaction} & Cl + NH$_3$    &$\!\!\!\rightarrow$ &    HCl + NH$_2$   & $ 1.1\!\times\! 10^{-11} e^{-1380/T}$  & Ga06 \\  
 \refstepcounter{reaction}R\arabic{reaction} & Cl + H$_2$S    &$\!\!\!\rightarrow$ &    HCl + HS   & $ 3.7\!\times\! 10^{-11} e^{+210/T}$  & At04 \\  
 \refstepcounter{reaction}R\arabic{reaction} & Cl + HS    &$\!\!\!\rightarrow$ &    HCl +  S   & $ 1.1\!\times\! 10^{-10} $  & Cl84 \\  
 \refstepcounter{reaction}R\arabic{reaction} & Cl$_2$  + OH    &$\!\!\!\rightarrow$ &    HOCl   +   Cl   & $ 7.9\!\times\! 10^{-13} \left(T/298 \right)^{1.35} e^{-745/T}$  & Br04 \\  
 \refstepcounter{reaction}R\arabic{reaction} & HOCl  + H    &$\!\!\!\rightarrow$ &    HCl   + OH  & $ 4.9\!\times\! 10^{-12} \left(T/298 \right)^{1.2} e^{-187/T}$  & Wa03  \\  
% \refstepcounter{reaction}R\arabic{reaction} & Cl  +    CO + M &$\!\!\!\rightarrow$ &      ClCO + M & $ 1.3\!\times\! 10^{-33} \left(T/298 \right)^{-3.8}  $   &  At07 \\     
%          & Cl  +    CO  &$\!\!\!\rightarrow$ &   ClCO  & $ 1.0\!\times\! 10^{-12} $    &  \\  
% \refstepcounter{reaction}R\arabic{reaction} & ClCO  + Cl    &$\!\!\!\rightarrow$ &    CO  + Cl$_2$  & $ 2.1\!\times\! 10^{-09}  e^{-1670/T}$  & Ba81 \\  
% \refstepcounter{reaction}R\arabic{reaction} & ClCO  + H    &$\!\!\!\rightarrow$ &    CO  + HCl  & $ 2.1\!\times\! 10^{-09}  e^{-1670/T}$  & \\  
 \refstepcounter{reaction}R\arabic{reaction} & OClO   + O &$\!\!\!\rightarrow$ &   ClO  +   O$_2$    & $ 2.4\!\times\! 10^{-12} e^{-960/T}$  &  Sa03\\    
 \refstepcounter{reaction}R\arabic{reaction} & OClO   + H   &$\!\!\!\rightarrow$ &  ClO   +  OH      & $ 7.8\!\times\! 10^{-11} $  & Wa89 \\  
 \refstepcounter{reaction}R\arabic{reaction} & OClO  +  OH  &$\!\!\!\rightarrow$ &  HOCl +  O$_2$   & $ 1.0\!\times\! 10^{-11}$  &  At07\\   
 
 \refstepcounter{reaction}R\arabic{reaction} & ClO  +   O  + M  &$\!\!\!\rightarrow$ & OClO   + M      & $ 6.0\!\times\! 10^{-31} \left(T/298 \right)^{-4.1} e^{-420/T}$  &  Zh03\\
              & ClO  +   O   &$\!\!\!\rightarrow$ &  OClO    & $ 3.6\!\times\! 10^{-11} e^{+44/T}$  & Zh03 \\  
  \refstepcounter{reaction}R\arabic{reaction} & OClO +   Cl  &$\!\!\!\rightarrow$ &  ClO  +   ClO   & $ 3.2\!\times\! 10^{-11} e^{+170/T}$  & At07 \\ 
 \refstepcounter{reaction}R\arabic{reaction} & Cl   +   HO$_2$  &$\!\!\!\rightarrow$ &  HCl  +   O$_2$   & $ 3.4\!\times\! 10^{-11} $  & At07 \\   



\hline
\hline
\multicolumn{6}{l}{ }\\
\multicolumn{6}{l}{ $a$ --- M refers to the background atmosphere, principally H$_2$ and He; units of density [cm$^{-3}$].}\\
\multicolumn{6}{l}{$b$ --- 2-body reaction rates are in cm$^{3}$s$^{-1}$;  3-body rates are in cm$^{6}$s$^{-1}$.}\\
\end{longtable}  

\newpage

\noindent {\bf Notes on Reactions}

 Reaction rates are selected from the publicly (http://kinetics.nist.gov/kinetics) available NIST database.
In order of priority, we choose between reported reaction rates according to relevant temperature range, newest review, newest experiment, and newest theory.  
Three body reactions are from experiments in H$_2$ when possible.  
Rates are not available for all reactions, especially for reverse (endothermic) reactions and reactions involving elemental sulfur.
Rates $k_r=K_{\rm eq}k_f$ of reverse two-body reactions are determined from the forward rate $k_f$ and the equilibrium $K_{\rm eq}=\exp{\left\{\left(-\Delta H + T\Delta S\right)/RT\right\}}$ using $H^{\circ}(T)$ and $S^{\circ}(T)$ from Chase (1998) as available.  In general we use the form $k_r \approx k_f A \left(T/300\right)^n e^{-B/T}$.  To first approximation the fits are tuned to 1400 K, so that $A=e^{\Delta S(1400)/R}$, $B=\Delta H(1400)/R$, and $n=0$.  The fits are extended to wider temperature range by adjusting $A$, $B$, and $n$.
Where we have had to assume a rate we have left the reference blank.
For assumed rates we also supply the thermodynamically consistent inverse.

Thermodynamic data for C$_2$H are very uncertain.  We use reported rates for both instead of using the thermodynamic data to derive a reverse rate. 



\noindent {\bf R\ref{R1}, R\ref{R2}} and many others.  The 2-body limit is included, but it is effectively irrelevant to our calculations.

\noindent {\bf R\ref{RCO}.} The 3-body recombination rate is estimated from the reported thermal decomposition rate at high temperatures and available thermodynamic data. 

\noindent {\bf R\ref{RHCO}.} The 3-body recombination rate is consistent with the reported thermal decomposition rate and available thermodynamic data. 

\noindent {\bf R\ref{RH2CO}.} The high pressure limit is constructed from the reported thermal decomposition rate and available thermodynamic data. 

\noindent {\bf R\ref{RHCHO}.} The recombination rate is constructed from the reported thermal decomposition rate and available thermodynamic data. 

\noindent {\bf R\ref{RCH}.} The 3-body recombination rate is constructed from the reported thermal decomposition rate at high temperatures and available thermodynamic data. 

\noindent {\bf R\ref{RCH2}.} The recombination rate is constructed from the reported thermal decomposition rate and available thermodynamic data. 

\noindent {\bf R\ref{RCH2+O2}.} Products assumed.

\noindent {\bf R\ref{RCH3OH}{\it ff.}}  Low pressure rates assume a critical density of $10^{19}$ cm$^{-3}$. 

\noindent {\bf R\ref{RCH3O+H2COH}.}  Assumed by analogy to CH$_3$+CH$_3$.

\noindent {\bf R\ref{RH2COH+H2COH}.}  Assumed by analogy to CH$_3$+CH$_3$.

\noindent {\bf R\ref{RC2H2}.}  The rate is assumed the same as for CH$_3$+H.

\noindent {\bf R\ref{RC2H3}.} The 3-body recombination rate is constructed from the reported thermal decomposition rate at high temperatures and available thermodynamic data.  In general, measured rates of thermal decomposition when expressed in Arrhenius form $Ae^{-B/RT}$ suggest lower $B$ than is suggested by the heats of formation of the shards, which would imply extrapolated low temperature rates for 3-body recombination that are much too high.  The high temperature rate is adjusted to the form $A\left(T/298\right)^c e^{-B'/RT}$ with $B'>B$ to give plausible extrapolations at low temperature. 

\noindent {\bf R\ref{RCH+CH4}.}  Products assumed.

\noindent {\bf R\ref{RC2H4}.} The rate is assumed the same as for CH$_3$+H.

\noindent {\bf R\ref{RC2H2+H2}.} This rather unlikely looking reaction is listed by Ts86 in the high pressure limit.  Ba94 list thermal breakup of C$_2$H$_4$ into C$_2$H$_2$ + H$_2$ in the low pressure limit; the reverse of that rate is listed here.  

\noindent {\bf R\ref{RC2H5}.} Two recent studies (Cu06, Mi05) differ by a factor of two at relevant temperatures; the older recommendation (Ba94) is close to Cu06.  

\noindent {\bf R\ref{RC2H2OH}.}  The measured reaction lacks a substantial activation  barrier, which implies that the primary product is CHCHOH (a.k.a.\ C$_2$H$_2$OH). 
According to Argonne 2025,  CHCHOH has enthalpy of formation $H^{\circ}_{298}=132$ kJ/mol.
Other isomers include CH$_3$CO (``acetyl,'' $H^{\circ}_{298} = -10.17$ kJ/mol), CH$_2$CHO  ($H^{\circ}_{298} = 16.1$ kJ/mol),
 CH$_2$COH  ($H^{\circ}_{298} = 115$ kJ/mol), and C$_2$H$_3$O ($H^{\circ}_{298} = 164.5$ kJ/mol, oxyranyl, which is a triangular radical related to oxyrane). 
 Although any of these would be an exothermic product of C$_2$H$_2$ + OH, only CHCHOH can form directly from C$_2$H$_2$ without climbing over substantial energy barriers.
 We omit CH$_2$COH and C$_2$H$_3$O for simplicity.

\noindent {\bf R\ref{RC2H2OH}.}  There is very little kinetics information on CHCHOH, which suggests that it is short lived and highly reactive.
We look to the analogous reactions of C$_2$H$_4$OH and its isomers for guidance.
Rearranging CHCHOH into the more energetically favored and better studied CH$_2$CHO likely requires climbing over a $\sim\! 10^4$ K.
 activation barrier ({\bf R\ref{RC2H2OH+M}}); rearranging CH$_2$CHO to CH$_3$CO likely requires climbing over a second $\sim\! 10^4$ K barrier.   

\noindent {\bf R\ref{RO+CH2CHO}.}  Assumed same as CH$_3$CO + O.

\noindent {\bf R\ref{RH+C2H2OH}.} Assumed same as C$_2$H$_4$OH+H $\rightarrow$ C$_2$H$_4$ + H$_2$O.

\noindent {\bf R198.}  The total rate is $7.7\times 10^{-11}$ at 298 K and 133 Pa.  Branching to CH$_2$CN (80\%) and CH$_3$CN (4\%) leaves 16\% for another path.  We assume that NH is the third path. 

% The apparent second order rate listed in the NIST database is at a low pressure and is not the actual asymptotic rate.  

\noindent {\bf R\ref{RC2H2OH+H}-R\ref{RC2H2OH+OH}.}  These reactions, which combine abstraction with rearrangement, are invented. 
Modest activation energies are assumed.

\noindent {\bf R\ref{RCH2CHO+H}-R\ref{RCH2CHO+OH}.}  Assumed same as for CH$_3$CO.

\noindent {\bf R\ref{RO+CH2CHO}.}  Assumed same as for C$_2$H$_4$OH.

\noindent {\bf R\ref{RC2H3OH}.}  Although vinyl alcohol is a molecule, there is relatively little kinetics information. 
We assume that C$_2$H$_2$OH+H $\rightarrow$ C$_2$H$_3$OH proceeds at a similar rate to C$_2$H$_4$OH+H $\rightarrow$ C$_2$H$_5$OH.

\noindent {\bf R\ref{RC2H3OH+O}.}  Assume a branch that cracks the C-C bond. 

\noindent {\bf R\ref{RC2H3OH+OH}.}  The actual product is probably an adduct.   

\noindent {\bf R\ref{RCH3CO+O}.} Ts86 lists a total rate of $7.0\!\times\! 10^{-11}$; we choose CO$_2$ as a likely product. 

\noindent {\bf R\ref{RC2H3OH+M}.} Liquid vinyl alcohol spontaneously converts to acetaldehyde with a 30 minute half-life at room temperature. 
This rate was chosen to reproduce that behavior.  

\noindent {\bf R\ref{RC2H4OH}.} The reaction of OH and C$_2$H$_4$ is fast, has a low barrier, and generates an adduct.
The adduct has several isomers.  Argonne 2025 characterizes 3 isomers in 9 configurations or electronic excitations.
The likely primary adduct is CH$_2$CH$_2$OH (``2-hydroxyethyl'').   Argonne 2025 estimates $H^{\circ}_{298} = -26$ kJ/mol.
The two other base isomers are CH$_3$CHOH (``1-hydroxyethyl,'' $H^{\circ}_{298} = -56$ kJ/mol) and CH$_3$CH$_2$O  (``ethoxy,'' $H^{\circ}_{298}=-12.56$ kJ/mol). 
There is some ambiguity in older literature regarding the two hydroxyethyls.
 There is very little information re the reactions of C$_2$H$_4$OH.
There is more re CH$_3$CH$_2$O and CH$_3$CHOH:  a couple of experiments and some theory. 

\noindent {\bf R\ref{RH+C2H4OH}}{\it ff.}  
The reported total rate of CH$_2$CH$_2$OH with H is fast (Bar82).
NIST lists the same reaction rate for CH$_3$CHOH.
The confusion results from both CH$_2$CH$_2$OH and CH$_3$CHOH being called ``hydroxyethyl.''  
The main product was once thought to be CH$_3$CHO, but this now seems doubtful (Xu11). 
  Xu11 compute rates and provide complicated fits for twenty reactions of H with the isomers CH$_3$CH$_2$O and CH$_3$CHOH.
 We construct rough fits to the more important reactions characterized by Xu11, and apply these to C$_2$H$_4$OH.
 In effect we treat C$_2$H$_4$OH, CH$_3$CH$_2$O, and CH$_3$CHOH as equivalent.  
 
\noindent {\bf R\ref{RO+C2H4OH}.} 
The total reaction rate of hydroxyethyl radicals with O at 298 K is very fast (Her88, Gro89, Ed92).
In these studies, the reactant is listed as CH$_3$CHOH, but here again there is ambiguity. 
The main product is listed as CH$_3$CHO, but given that abstraction by H atoms is not fast, we suspect that the fast reaction with O has other products.
Sk18 propose that the fastest path at low temperatures cleaves CH$_2$CH$_2$OH into two pieces, each with a C=O bond.
We adopt this, but reduce the reaction rate to something more comparable to what was measured by Gro89. 
For the hypothetical abstraction paths we use rates inspired by {\bf R\ref{RH+C2H4OH}.}

\noindent {\bf R\ref{ROH+C2H4OH}.} There is no information.  One expects an adduct, which goes beyond our truncation.  We invent two abstraction paths.


\noindent {\bf R\ref{RC2H4OH+H}.}  Xu11 compute rates for the isomers CH$_3$CH$_2$O and CH$_3$CHOH.  If anything, it should be more straightforward to add
 an H to CH$_2$CH$_2$OH to make ethanol.  The quoted rate is our own approximation to Xu11's rates.

 \noindent {\bf R\ref{RC2H5+OH}.}  The low pressure rate is quoted as a lower limit.

 \noindent {\bf R\ref{R163}.} Ground state C$_2$ reacts with almost any hydrocarbon to make bigger hydrocarbons.  Because C$_2$ is not very abundant in these atmospheres, we have chosen this single alternative to R62 (hydrogenation) as representative. Pa08 lists the products as ``products,'' we assume C$_4$H as the most extreme outcome.

\noindent {\bf R\ref{RC4H2},R\ref{R161}-R\ref{R162}.}  Placeholder rates for truncating chemistry assumed by analogy to H+C$_2$H$_2$. 
In practice C$_4$H$_2$ readily adds H atoms and grows (Kl05).

\noindent {\bf R\ref{R191}.} Assumes that the product HNO reacts quickly with H to give H$_2$+NO. {\bf Fix}

\noindent {\bf R\ref{RNO+CH}.}  There is strong disagreement re the major products: Ge99 favor CO, Be98 and Mar98 favor HCN.  We split the difference.

 \noindent {\bf R\ref{R196}-R\ref{R197}, R\ref{R200}-R\ref{R201}, R\ref{R209}-R\ref{R211}, R\ref{R221}.}  By analogy to R\ref{R223}, Moses (Mo96) assigned rates to these strongly exothermic reactions. However, high activation energies for other similar reactions such as R\ref{R212}, R\ref{RNH2+C2H4}, and R\ref{R224} suggest that the reactions could be slower at low $T$ than given here.
 
 \noindent {\bf R\ref{R199}.}  The rate is for products St95. Ya05 give NH+C$_2$H$_4$ and H$_2$CN + CH$_3$ as dominant channels.  We presume the channels break 50:50.

\noindent {\bf R\ref{RNH2}.}  There are no data.  We assume a slow rate comparable to similar reactions.  

\noindent {\bf R\ref{R212}.}  A reverse fit to Xu99 for $500<T<1300$ K.  

 \noindent {\bf R\ref{RNH3+C2H}.} The measured rate for NH$_2$+C$_2$H$_2$ of $k=6.1\!\times\! 10^{-12}e^{-5700/T}$ (He95) has a much lower activation energy than expected if the products were NH$_3$+C$_2$H.  The product is therefore likely to be an adduct
 leading perhaps to the formation of stable molecules such as CH$_3$CN or HC$_3$N, species that we do not yet include.
 HCN and CH$_3$ are also possible exothermic products, but these are not likely, and thus we have omitted the reaction.
 
 \noindent {\bf R\ref{RNH2+C2H4}.}  The measured rate is for all products (probably an adduct).  We have presumed an outcome of two stable molecules.
 
 \noindent {\bf R\ref{RCN+C2H4}.}  Products assumed; the actual products may be C$_2$H$_3$CN + H.

\noindent {\bf R\ref{RCH+NO}.}  There is strong disagreement re the major products: Ge99 favor CO, Be98 and Mar98 favor HCN.  We split the difference.

\noindent {\bf R\ref{RNCHOH}.} HCN oxidation.  The reaction to form an adduct is known to be rather fast with only a modest activation energy.
The rate given here is a rough fit to data plotted in Figure 5.17 in Bu14.
The modest activation energy implies that the OH attaches directly to HCN without additional rearrangements.
The simple addition of OH to N to make HCNOH ($H^{\circ}_{298}=244$ kJ/mol, Argonne 2025) is very endothermic and hence, despite our having once imagined otherwise (Za86),
 HCNOH is not a plausible primary adduct.
De00 and Bu14 agree that the primary adduct has the form NCHOH, which is noteworthy for having no NH bond. 
This reaction path is illustrated by De00 in their Figure 1.
However, Argonne 2025 does not list NCHOH as a characterized species.
Instead, Argonne 2025 lists four other isomers, all of which could form exothermically from HCN+OH:
HNCOH ($H^{\circ}_{298}=68$ kJ/mol), HNCHO (71 kJ/mol), NH$_2$CO (-14 kJ/mol), and CH$_2$NO (154 kJ/mol).
In their Figure 1, De00 estimate that $H^{\circ}_{298}$ of NCHOH would be $\sim\!60$ kJ/mol (we use 70 kJ/mol --- the precise value does not matter much). 
De00 predict that each substantial rearrangement of NCHOH requires overcoming an energy barrier of the order of 100 kJ/mol.
It takes one step to go from NCHOH to HNCHO or to H+HOCN.  It takes a second step to take HNCHO to NH$_2$CO or to HNCO+H.
Two more steps can take HNCHO to CH$_2$NO.
De00 omit HNCOH from their network.
  
There is no information re what NCHOH might do other than isomerize at high temperatures. 
One expects that NCHOH would grow by addition --- in giant planet atmospheres, it should hydrogenate --- 
but there is no information re the imaginary molecule HNCHOH (e.g.).
 As we have no alternative fates for NCHOH, we invent three fast abstraction reactions that yield HOCN (cyanic acid).
In an H$_2$-rich atmosphere we can expect HOCN to convert to isocyanic acid (HNCO) by reactions of the form X+HOCN $\rightarrow$ XH + NCO followed
by NCO+H$_2$ $\rightarrow$ HNCO + H. 
Hence we do not need to include HNCHO, HNCOH, or CH$_2$NO in our truncated network.
We keep the energetically favored NH$_2$CO, but we omit the isomerization pathways,
leaving direct addition of H to HNCO as the main path to NH$_2$CO.
 In network, adding H to NH$_2$CO gives a direct path to formamide, a molecule that interests origin-of-life chemists.
 We have therefore added NH$_2$CHO to our network.
 
\noindent {\bf R\ref{RHOCN+H}}. Sz84 measured $k=1.66\times 10^{-12}$ cm$^3$/s for $1800<T<2600$ K.  As HOCN is an actual molecule, we expect a significant activation energy.  The invented reaction rate is one possibility. 

\noindent {\bf R\ref{RHOCN}}.  All particulars are the same as for R\ref{RHOCN+H}.

\noindent {\bf R\ref{RHOCN+O}}. Cam01 estimate rates at 1000 and 1500 K from theory.  The adopted rate is patterned after the reaction O + HNCO $\rightarrow$ OH + NCO
(the reverse of R\ref{RHNCO+O}).

\noindent {\bf R\ref{RHOCN+OH}}. The assumed rate is patterned after  OH + HNCO $\rightarrow$ H$_2$O + NCO.

\noindent {\bf R\ref{RHNCO+O}.}  The reverse has also been studied.  Ts92 recommend $k_r = 6.2\times 10^{-13} \left(T/298)\right)^{2.11} e^{-5750/T}$.
Cam01 compute rates at 1000 and 1500 K that are consistent with Ts92.  

\noindent {\bf R\ref{RHNCO}.}  The 3-body recombination rate is consistent with the reported thermal decomposition rate and available thermodynamic data.

\noindent {\bf R\ref{RNH2+CO}}.  Assumed same as CH$_3$+CO.

\noindent {\bf R\ref{RNH2CO+M}.}  The activation barrier for rearranging is from De00.

\noindent {\bf R\ref{RNH2CO}}{\it ff.}  Reactions of NH$_2$CO are patterned after the reactions of CH$_3$CO.  The low pressure rate for R\ref{RNH2CO}
assumes a critical density of $10^{19}$ cm$^{-3}$.

\noindent {\bf R\ref{RNH2CHO}}.  Formamide has at times been a popular origin-of-life molecule.  Here we truncate the system at formamide; in reality, formamide is a stepping stone to make bigger molecules.
We assume that addition of H to NH$_2$CO resembles addition of H to CH$_3$CO
and that simple attacks on NH$_2$CHO resemble those on CH$_3$CHO.

\noindent {\bf R\ref{RNH2CHO+O}.}  Scaled from O + CH$_3$CHO.

\noindent{\bf R\ref{RH2CN}.} The published rate for R\ref{RH2CN} leads to H$_2$CN being rather abundant.  Reported reactions of H$_2$CN are abstractions by H (To03) and OH (Ni03).  Upper limits $<10^{-15}$ cm$^3$/s have been reported on reactions with H$_2$, CO, CH$_4$, and C$_2$H$_4$ (Ni03).  Because H$_2$CN can be abundant, we have created other abstraction channels analogous to the reaction with OH.  Unfortunately, truncation precludes hydrogenation of HCN through NH$_2$CH$_3$ and eventually to NH$_3$ and CH$_4$.  In the bigger picture, hydrogenation of HCN is unimportant compared to the hydrogenation of N$_2$ and CO, because HCN is never a major species.  
 
 \noindent {\bf R\ref{RN+C2H5}.}  The rate is for products St95. Ya05 give NH+C$_2$H$_4$ and H$_2$CN + CH$_3$ as dominant channels.  We presume the channels break 50:50.
 
\noindent {\bf R\ref{ROH+H2CN}.} Although this probably forms an adduct, our truncated system assumes abstraction.

\noindent {\bf R\ref{RN+H2CN}.} Rate from Ne90.  Products assumed by To03.

\noindent {\bf R\ref{RCN+H2CN}.} Mo96 use an even higher rate.

\noindent {\bf R\ref{RC2H+H2CN}.} Geometric mean between C$_2$H + C$_2$H$_3$ and CN + H$_2$CN.
 
\noindent {\bf R\ref{RNNH}.}  The low pressure rate for N$_2$+H+M $\rightarrow$ NNH + M assumes a critical density of $10^{20}$ cm$^{-3}$.
NNH is the only adduct of which I am aware that is unstable with respect to its reactants.  It resembles a resonance more than it resembles a molecule.
But it is essential to the hydrogenation of N$_2$ in giant planets, and it is an essential intermediate in NO production in flames.
The reaction rates listed here are rebuilt from Ha03's high temperature fits to behave well at low temperatures.

\noindent {\bf R\ref{RN2H2}.} One can find estimated rates for thermal decomposition of N$_2$H$_2$, but inverting these is problematic.
Calculations by Bi06 are consistent with little or no barrier between NNH and N$_2$H$_2$.  We therefore assume rather fast rates inspired by the somewhat analogous addition of H to HCO.

\noindent {\bf R\ref{RN2H3}.} There is little information about the N$_2$H$_3$ radical apart from its reactions with N-containing species in a combustion setting.
 Here we assume that the reaction goes at the same rate as C$_2$H$_2$ + H $\rightarrow$ C$_2$H$_3$.

\noindent {\bf R\ref{RN2H4}.} We assume the same rate for N$_2$H$_3$ + H as for NH$_2$+NH$_2$ $\rightarrow$ N$_2$H$_4$. 
 This rate is also very similar to C$_2$H$_3$+H $\rightarrow$ C$_2$H$_4$.
 
\noindent{\bf R\ref{1CH2+H2}.} Total rate from Ga08, product distribution form Do18.

\noindent{\bf R\ref{CH2+CO}.} Assumed same as $^1$CH$_2$+N$_2$.

\noindent{\bf R\ref{1CH2+CO}.} Assumed same as de-excitation of O($^1$D).

\noindent{\bf R\ref{1CH2+N2}.} Assumed same as de-excitation of O($^1$D).

\noindent{\bf R\ref{1CH2+CO2}.} Assumed same as de-excitation of O($^1$D).

\noindent{\bf R\ref{CH2N2}.} Diazomethane.  Low pressure rate assumed.  The fascinating diazirine (a triangular molecule whose uncoiling could provide a route to cyanamide) forms about 1\% as often (Xu10).  It costs roughly 50 kJ/mol to convert diazomethane into diazirine.

\noindent{\bf R\ref{CH2N2+OH}.} Products assumed (probably an adduct in real life).

\noindent{\bf R\ref{CH2N2+1CH2}.} Assumed same as $^1$CH$_2$ + CH$_2$CO.

\noindent{\bf R\ref{CH2N2+CH2}}{\it ff.} Assumed same as X + CH$_2$CO but with activation energy rescaled in proportion to the bond strength. 

\noindent{\bf R\ref{C2H3+N}.} Pa96 list the addition as 4\% of products at 298 K and 133 Pa.  We construct a suitable low pressure rate
from the total rate at 133 Pa. 

\noindent{\bf R\ref{CH2CN+H}.}  This is the rate we are using for H + H$_2$COH $\rightarrow$ CH$_3$OH.

\noindent{\bf R\ref{1CH2+HCN}.}  The rate is the same as what we use for CH$_2$ + C$_2$H$_2$ $\rightarrow$ C$_3$H$_4$.

\noindent{\bf R\ref{CH2CN+O}.} Products assumed.

\noindent{\bf R\ref{CH3CN+H}.}  These reaction rates are constructed from rates measured for H+CH$_3$CN by Ja70 and rates computed
for the analogous reaction H + CH$_3$Cl by Lo04. We reject CH$_4$ as an endothermic product.  We assume products
analogous to H$_2$ + CH$_2$Cl and HCl + CH$_3$ of the CH$_3$Cl reaction. 

\noindent{\bf R\ref{O1D+CO}.}  Hypothetical slow reaction patterned after  O($^1$D) + N$_2$ $\rightarrow$ N$_2$O.

\noindent{\bf R\ref{N2D+HCN}.} This reaction may be an important sink of HCN on Titan.  
Based on R\ref{N2D+C2H2}, the likely products of a fast reaction would be NCN + H, which is much more interesting (and less destructive of nitriles) than N$_2$.
As NCN lies outside our system, we are forced to use CH and N$_2$ as products, but we have opted against assuming a fast rate for this channel.    

\noindent{\bf R\ref{N2D+C2H2}.}  He99 reports that H+HCCN, which retains a C=C double bond, is the most likely channel. This seems right, but
HCCN lies outside the boundaries of our system. Under conditions where this reaction might be important (an N$_2$ atmospheres subject to cosmic rays or stellar storms), HCCN could be a source of cyanogen.  The only possible products in our system are HCN + CH, which at least have the virtue of creating a C-N bond.  The rate is an average.  

\noindent{\bf R\ref{N2D+C2H4}.}  This reaction may or may not be an important source of CH$_3$CN on Titan (Lo15).
He99 and Lo15 favor cyclic-CH(N)CH$_2$+H as the product.  This species is outside our scope, nor is it in the scope of Argonne 2025.
Within our system, possible products include CH$_3$CN + H, CH$_2$CN + H$_2$,  H$_2$CN + CH$_2$, HCN + CH$_3$, and CN + CH$_4$.
We have opted to assume the CH$_2$CN channel as easy to picture geometrically.
The rate is an average of several measurements.

\noindent A version of sulfur chemistry is describe in Za16.  Za16 used an incomplete set of S species: S$_2$, S$_3$, S$_4$, and chain and ring forms of S$_8$.
We will update that here.
NIST's tables of thermodynamic properties of S$_n$ are problematic for $n>2$.  
We have followed Rao73 and Stu03 in developing our own estimates for standard enthalpies and entropies of formation for these species.
The table:
\begin{table}[h]
\caption{Thermodynamic parameters for the Sulfur allotropes used here}
%\hskip-1.6cm   % this is a bruteforce command to move the box.  \flushleft did not work for this table.  
\begin{tabular}{l r r r r r r r}
\hline
Species  & $H^a_{298}$ & $S^b_{298}$ & A & B & E & F & G \\
S      & 276.98 & 167.83 & 27.46 & -13.33 & -0.056 & 269.1 & 204.3  \\
S$_2$  & 130 & 228 & 35.73 & 1.17 & -0.33 & 118.2 & 269  \\
S$_3$  & 141 & 270 & 53.8 & 4.35 & -0.65 & 122.6 & 330.1  \\
S$_4$  & 146 & 310 & 79.9 & 3.276  &  -1.18 & 118.1 & 399  \\
S$_8$  & 101 & 423 & 178.53 & 3.60 & -2.138 & 40.45  & 626  \\
S$_8^{\ast}$  & 134 & 448 & 203 & 3.60   & -2.14 & 66  & 681\\
HS$_4$  & 130 & 330 & 79.9 & 3.276  &  -1.18 & 102.1 & 419\\
\hline
\multicolumn{8}{l}{$a$ --- kJ/mol}\\
\multicolumn{8}{l}{$b$ --- J/mol/K}\\
\multicolumn{8}{l}{For S$_n$, $n>1$, $C_p$ are from Rau73. }\\
\multicolumn{8}{l}{$H = At + Bt^2/2 + Ct^3/3 + D t^4/4 - E/t + F$}\\
\multicolumn{8}{l}{$S = A\ln{t} + Bt + Ct^2/2 + Dt^3/2 - E/(2t^2) + G$}\\
\multicolumn{8}{l}{where $t=T/1000$ }\\
 \end{tabular}
\label{table:two}
\end{table}


In Za16, HS$_4$ was invented to create a concise catalytic cycle to reduce S$_8$ to H$_2$S in the deep atmosphere,
a process that would ordinarily also involve other sulfur allotropes S$_n$, $n\neq 8$; sulfanes, H$_2$S$_n$; and radicals HS$_n$.

For adding S atoms or S chains to other S chains, we assume there is no barrier to addition and that low pressure rates get systematically faster for longer chains, as the longer chains have more ways of distributing excess energy. For rearranging chains, we 
For breaking cyclic S$_8$ into S$_8^{\ast}$, we assume an activation barrier of 1200 K.
For breaking di-radical chains (S$_3$, S$_4$, and S$_8^{\ast}$ in the severely truncated system used here), we assume energy barriers of the order of 1000 K. 

\noindent {\bf R\ref{RS2}.}  Reported rates are highly discrepant. 
 Ni79 estimate in their review that $k=1.2\!\times\! 10^{-29}$ cm$^6$/s, which discounts the rate previously obtained by Fair and Thrush 1969, which is nearly 4 orders of magnitude slower.
 Moses et al (2002) used $k=5\!\times\! 10^{-32}e^{+900/T}$ to model S$_2$ on Io;  the temperature dependence is copied from O + O.
 Moses 2002 rate splits the difference:  it is is about 40 times faster than Fair and Thrush and about 20 times slower than Ni79 at 298 K. 
 Du et al (2008) used theory to calculate $k_0=2.0\!\times\! 10^{-33}e^{ +210/T}$, which is very close to the rate measured by Fair and Thrush in 1969.
Krasnopolsky 2013 adopts the Du et al temperature dependence but increases the overall rate by a factor of 5.  
We follow Du08 precisely.  Note that Du08's $k_{\infty}$ seems surprisingly slow.  

\noindent {\bf R\ref{RS3}{\it ff}.}  The analogous low pressure reaction to make ozone is quite slow, although the high pressure rate in the 2-body limit is rather fast.
Kra13 uses the same rate he uses for S + S.  Mos02 use a faster rate.  Our assumed 3-body rate is similar to Kra13's but uses the same $T^{-2}$ temperature dependence as Mo02.  We assume a moderately fast high pressure limit similar to ozone.

\noindent {\bf R\ref{RS+S3}.} This one is perplexing, and possibly important. 
La73 infer from a complex mechanism that this reaction is fast.  
Mo02 assume a very fast rate of $8\times 10^{-11}$ cm$^3$/s.
By contrast, Kra13 assumes a 2800 K activation barrier. 
We can estimate an activation energy using Pauling's rule of thumb: the energy barrier will be roughly 7\% of the bond strength when one bond is broken in a reaction.
The Bond strength of symmetrical S$_3$ is 345 kJ/mol, hence Pauling's 7\% solution would map to a barrier of 2900 K,  
which is almost exactly what Kra13 estimated.
On the other hand, Pauling's is just a rule of thumb. 
 For comparison, the analogous reaction O + O$_3$ $\rightarrow$ O$_2$ + O$_2$ has an energy barrier of 2060 K.
Some other less convincing analogs are S + O$_3$ $\rightarrow$ SO + O$_2$ (no activation energy) and O + S$_2$O $\rightarrow$ SO + SO (no activation energy).
There is no obvious barrier to adding S to S$_3$.  
Accessible metastable transitional states of excited S$_4$ are numerous (Ste03 illustrates a dozen), whilst O$_4$ is almost unimaginable.
One collisions can remove enough excess energy to preclude returning to S to S$_3$, but before S$_4$ is fully stabilized,
 there can still be enough energy in excited S$_4^{\ast}$ to fission into two S$_2$ molecules, which one easily imagines spinning away like a pair dumbbells.
We therefore expect R\ref{RS+S3} to compete with polymerization at low pressures.
We assume an activation energy of 1200 K, an arbitrary choice. 

\noindent {\bf R\ref{RS2+S2}.} La73 and Ni79 both infer fast 3-body rates of order $10^{-29}$ for this.
 Kra13 uses the same rate he uses for S + S and S + S$_2$.
 Mo02 use a faster rate (in deference to La73 and Ni79) and give it the same temperature dependence that they assume for S + S.
 Our assumed rate is in between Kra13 and Mos02.

\noindent {\bf R\ref{RS+S4}.} This seems similar to R\ref{RS+S3}.  
The weakest bond in S$_4$ will have a strength or order $\sim 300$, kJ/mol, for which the rule of thumb implies a barrier of 2500 K.
 By contrast, Mo02 assume no barrier and a fast rate of $8.0\!\times\! 10^{-11}$ cm$^3$/s.
The closest stable analog to S$_4$ is probably S$_2$Cl$_2$. 
There is little information despite the stability S$_2$Cl$_2$.
Reaction with Cl was too slow to measure ($<\!\times 4\times 10^{-14}$, Kra84), whilst reaction with H to make HCl is pretty fast ($\sim\! 10^{-11}$, Sun1978).
For R\ref{RS+S4}, we will assume a 1000 K reaction barrier.   

\noindent {\bf R\ref{RS8}.}  We presume that longer sulfur changes are more readily stabilized and thus that the 3-body reactions are much faster --- this expectation is based on a loose analogy between sulfur and sulfur chains and alkyl radicals and alkanes (Be78).  E.g., CH$_3$ + CH$_3$ $\rightarrow$ C$_2$H$_6$ is very fast.

\noindent {\bf R\ref{RS8star}.}  This represents the noncyclic (chain) form of S$_8$. 
According to Ste3, this is a partial helix, with the diameter of a 7-member ring in which the first and second S atoms overlie the seventh and eight.
 We presume that S$_8^{\ast}$ is both easily made and easily rearranged into a ring.
We presume a modest energy barrier of order of 1000 K that resists uncoiling the structure to the diameter of an 8-member ring. 
 
 
\noindent {\bf R\ref{RS+HS}.} Reported rates for R\ref{RH+HS} are discrepant, $4.0\times 10^{-11}$ cm$^{3}$s$^{-1}$ (Sc73) and $<\!5\times 10^{-12}$ cm$^{3}$s$^{-1}$ (Ni79).  The seemingly analogous reactions O+OH $\rightarrow$ O$_2$+H ($k=3.3\times 10^{-11}$ cm$^{3}$s$^{-1}$), O+HS $\rightarrow$ SO+H ($k=7\times 10^{-11}$ cm$^{3}$s$^{-1}$), and S+OH $\rightarrow$ SO+H ($k=6.6\times 10^{-11}$ cm$^{3}$s$^{-1}$), are all rather fast at room temperature. 
The reverse, although very endothermic, is a major sink of S$_2$ in giant planet atmospheres.
The reverse rate can be fit by $k_r=5.3\!\times\! 10^{-10}\left(T/298 \right)^{0.95} e^{-9920/T}$, which spans high temperature measurements by Sh98 and Woi95.
Reversing the reverse suggests a forward rate of the order 
 $k_f = 2.0\!\times\! 10^{-11}\left(T/298 \right)^{0.70}e^{-300/T}$, in which we have inserted a small activation energy inspired by O + OH $\rightarrow$ O$_2$ +H.
Kra13 uses $4.5\times 10^{-11}$.


\noindent {\bf R\ref{RH+HS}.}  Another case of discrepant rates.  Ni79 report $k=2.2\times 10^{-11}$ and Ti81 report  $k<1.7\times 10^{-11}$ cm$^{3}$s$^{-1}$ at 298 K.  The analogous slightly endothermic reaction H + OH $\rightarrow$ H$_2$ + O provides context.
The OH radical has a bond strength of 430 kJ.  Pauling's rule of thumb is that the activation barrier should be 7\% of 430 kJ/mol, i.e. 30.1 kJ/mol.
The measured reaction rate can be approximated by $1.4\times 10^{-11} e^{-3800/T}$;  3800 K corresponds to 31.6 kJ/mol.
The reaction is endothermic by 6.3 kJ/mol, which corresponds to an actual reaction barrier of 25.3 kJ/mol at 298 K.
The SH radical has a bond strength of $\sim\!350$ kJ/mol, so the scaled activation barrier is 20.6 kJ/mole.
The expected reaction rate is therefore $\sim\!2\times 10^{-11} e^{-2400/T}$ cm$^3$/s.
By contrast, Kra13 assumes a fast rate of $2.5\times 10^{-11}$.


\noindent {\bf R\ref{RH+S3}} Rate assumed with activation energy consistent with R\ref{RS+S3}.
For comparison, Kra13 assumes $1.2\times 10^{-10}e^{-1950}$ for H + S$_3$, a rate that seems fully consistent with his rate for R\ref{RS+S3}.
 
\noindent {\bf R\ref{RH+S4}.} Rate assumed with activation energy consistent with R\ref{RS+S4}.

\noindent {\bf R\ref{ROH+HS}.} Assume same as OH + H$_2$S.

 \noindent {\bf R\ref{RCH3SH}.}  The actual product at low temperatures is CH$_3$SH, a molecule we have not included.
 
\noindent {\bf R\ref{RH2S}, R\ref{RHSH}.}  We have constructed forward reaction rates that, when reversed, approximate the
reported rates of the two thermal decomposition pathways for H$_2$S at the temperature ranges where these
data have been obtained (Ro84, Te90, Woi94, Ol94, Sh96).  Note that S + H$_2$ $\rightarrow$ H$_2$S does not have a parallel in O + H$_2$.

\noindent {\bf R\ref{RHS+HS}.} We construct a rate from the reverse of high temperature S+H$_2$S $\!\leftrightarrow\!$ HS+HS.
We use Sh96 because the reported activation energy is more consistent than Woi94 with thermodynamic equilibria
and the generally high rates reported for the reverse reaction (Sc73, Ni79, St87) at room temperature.

\noindent {\bf R\ref{RH+H2S}.} H+H$_2$S $\!\leftrightarrow\!$ HS+H$_2$.
Pr77 reported a rate for HS+D$_2$ $\rightarrow$ HDS + D, $2.24\!\times\! 10^{-11} e^{-3530/T}$ cm$^{3}$s$^{-1}$ ($808<T<937$ K), that is two orders of magnitude faster than $k_{r}$ at 900 K.   

\noindent {\bf R\ref{RCH2+H2S}.} The measured rate at 295 K is $ 2.0\!\times\! 10^{-12}$ cm$^3$s$^{-1}$.  We have assumed the activation energy.

\noindent {\bf R\ref{RHS4}.} HS$_4$ and all its reactions are invented to provide a concise hydrogenation path for S$_8$.  The rate for H + H$S_2$ that NIST associates with
Se02 appears to suffer from a typo or garbled units.  The rate for H + O$_2$ $\rightarrow$ HO$_2$ is a possible analog, hydrocarbons (e.g., H + C$_2$H$_4$) could be another.  
We assume a relatively slow rate with no activation energy.  
 
 
\noindent {\bf R\ref{RH+S8}.}  Activation energy of 2200 K is appropriate for an S-S bond strength of 265 kJ/mol.
 
\noindent {\bf R\ref{RH+S8L}.}  Assumes same as for H + S$_4$.
 
\noindent {\bf R\ref{RH+HS4}.}  Assumes same rate as H + HS$_2$ $\rightarrow$ H$_2$ + S$_2$, Se02.  Se02's rate is for $873<T<1423$ K,
but it is a form that extrapolates poorly to low temperatures.  We have replaced the published rate with a nearly identical Arrhenius form that extrapolates safely to low temperatures.  

\noindent {\bf R\ref{RHS4+H}.} Assumes same rate as H + HS$_2$ $\rightarrow$ H$_2$S + S, Se02.  

\noindent {\bf R\ref{ROH+HS4}.}  Assumes same rate as OH + HS$_2$ $\rightarrow$ H$_2$O + S$_2$, An18.

\noindent {\bf R\ref{RHS+HS4}.}  We assume same rate as HS + HS$_2$ $\rightarrow$ H$_2$S + S$_2$, Se02.
 Se02's rate is for $873<T<1423$ K, but it is a form that extrapolates poorly to low temperatures.  We have replaced the published rate with a nearly identical Arrhenius form that extrapolates safely to low temperatures.

\noindent {\bf R\ref{RCH3+HS4}-R\ref{RS+HS4}.} We assume same rate as HS + HS$_2$ $\rightarrow$ H$_2$S + S$_2$, Se02.

\noindent {\bf R\ref{RHS4+HS4}.}  Assumes that other abstraction reactions have same barrier as H+HS$_2$ $\rightarrow$ H$_2$ + S$_2$ (Se02),
with allowance for tenfold slower thermal velocity of HS$_4$ compared to H.

\noindent {\bf R\ref{RO+HS}.}  JPL19 recommend $1.6\times 10^{-11}$ with an uncertainty of a factor 5.

\noindent {\bf R\ref{RN+SO}.}  Assumed by analogy to N+O$_2$.

 \noindent {\bf R\ref{RCO+SO}.}   Assumes an analogy to CO+O$_2$ with an activation barrier about 2/3rds as high.
 
 \noindent {\bf R\ref{RSO+O2}.}   This is NIST best fit to At04, Ga98.
  
\noindent {\bf R\ref{RHSO}.} The H-SO bond significantly strongly than the H-O$_2$ bond.
 Nonetheless, we use the same rate as for H + O$_2$.  % The HSO reactions turn out to be unimportant.

\noindent {\bf R\ref{RHSO+H}}{\em ff.} We use the same rates as for H + HO$_2$.  % The HSO reactions turn out to be unimportant.

\noindent {\bf R\ref{RS+CO}.} The rate for S + CO $\rightarrow$ OCS is effectively unknown.  The reaction is likely important on Venus.
Experiments show that OCS + M $\rightarrow$ CO + S + M is rather slow (Oy94) despite the weak S-CO bond, such 
that simple attempts to reverse the high temperature thermal destruction imply that 3-body association of S + CO $\rightarrow$ OCS is also quite slow.
Kra13 uses $k=3\times 10^{-33}e^{-1000/T}$ as the low pressure limit, which resembles the rate for O+CO $\rightarrow$ CO$_2$ but with a lower activation barrier that accounts for the weaker S-CO bond.
We adjust Kra13's rate slightly to match the reverse reaction rates at 2000 K.

\noindent {\bf R\ref{ROH+CS}.}  Patterned after OH + CO, taking into account that CS is much more reactive than CO.

\noindent {\bf R\ref{RCO+S3}.}  Energy barrier estimated by Pauling's rule of thumb.  Bond strength of symmetrical S$_3$ is 345 kJ/mol, hence the 7\% solution gives a barrier of 2900 K.  Kra13 use a barrier of 20,000 K, which is either a typo or a means of limiting the importance of what would otherwise be a dominant reaction on Venus.  We use a relatively small $A$ factor as consistent with many CO reactions.

\noindent {\bf R\ref{RCO+S4}.}   Energy barrier estimated by Pauling's rule of thumb.  Bond strength of symmetrical S$_4$ is 321 kJ/mol, hence the 7\% solution gives a barrier of 2700 K.  

\noindent {\bf R\ref{RS+CH3}.}  The actual product is likely to be H$_2$CS, which lies outside the scope of this list.

\noindent {\bf R\ref{RCS2}.}  The Gl15  high pressure rate is quoted for $1900<T<2800$.  It extrapolates to an unrealistically high rate at low temperatures. 
We have therefore constructed a form that extrapolates to much lower but still very fast rates of the order of $1\times 10^{-10}$ at room $T$.

\noindent {\bf R\ref{RCS+HS}.}  Assumed twenty times faster than CO+OH.

\noindent {\bf R\ref{RCS2+O}-R\ref{R345}.}  Overall rate from Si88, branching ratios from Co92.


\noindent {\bf R\ref{R346}.} The room temperature reaction rate is appreciable, but the product is an adduct (At04). The room temperature upper limit on the products OCS + HS is $k<2\!\times\!10^{-15}$ cm$^3$/s (At04). 

\noindent {\bf R\ref{RHCS}}{\em ff}. Tabulated thermodynamic properties for HCS appear nonexistent.
The HCS radical is by assumption an important intermediate for CS and CS$_2$ formation in our scheme.
The HCS reactions are invented but plausible. We have in general reduced the very fast rates estimated by Mo96 by a factor of five. They are important here because they provide the chief kinetic pathway to CS and CS$_2$, two molecules that were observed to be abundant in the SL9 impacts (Harrington et al 2004). 
 
\noindent {\bf R\ref{RSO+SO3}.}  Experiment at 298 K.

\noindent {\bf R\ref{RHSO3}.}  The HOSO$_2$ radical appears to be little studied because it reacts quickly with O$_2$.
We include it as a step to H$_2$SO$_4$ for conditions where O$_2$ may be less abundant.

\noindent {\bf R\ref{RH2SO4}.}  This is an awkward rate to fit at low pressures, for many reasons, as discussed by San03.
The high pressure limit for gas phase H$_2$SO$_4$ seems relatively secure.
The low pressure reaction seems to be mostly driven by H$_2$O and H$_2$SO$_4$ itself as the third body.
Sa03 recommend $2.26\times 10^{-43} \left(T/298 \right)^{1.0} e^{+6544/T} \left[\mathrm{H}_2\mathrm{O}\right]^2$.
The strong temperature dependence builds in the saturation vapor pressure of H$_2$O, and hence would appear to describe the
direct formation of condensed sulfuric acid rather than the vapor.  Here I force fit a rate to the prescribed format
$k=\left[\mathrm{H}_2\mathrm{O}\right]\left[\mathrm{SO}_3\right]\left[\mathrm{M}\right]$ calibrated to the preferred rate at 298 K
for saturated water vapor.

\noindent {\bf R\ref{Chlorine}.} Chlorine in an oxidizing atmosphere is very well studied (e.g., see Sa03).

\newpage
% \addtocounter{photo}{1}
\setlongtables % keeps the width uniform across both pages
% \footnotesize{
\begin{longtable}{l lcl l p{3.5cm} } 

 &  {\large\bf Table 3.}  &  & {\large\bf Photolysis Reactions} & \multicolumn{2}{l}{needs to delete ``rate,'' update branching} \\
\hline
 & {\large\strut Species}  &  & {\large Products} & {\large Rate$^f$} & {\large Reference} \\
\hline \hline 
\endfirsthead
\hline
 & {\large\strut Species}  &  & {\large Products} & {\large Yield$^f$} & {\large Reference} \\
\hline
\endhead
 \refstepcounter{photo}\label{PH2}P\arabic{photo}  & H$_2$      + h$\nu$             &$\!\!\!\rightarrow$ &  H + H     & 0.25  &  note \\ %
 \refstepcounter{photo}P\arabic{photo}  & H$_2$O       + h$\nu$         &$\!\!\!\rightarrow$ &  OH           + H                          & 0.86 & Sa03\\ %
 \refstepcounter{photo}P\arabic{photo}  & H$_2$O       + h$\nu$         &$\!\!\!\rightarrow$ &  H$_2$     + O($^1$D)             & 0.06 & Sa03, Hu92\\ %
 \refstepcounter{photo}P\arabic{photo}  & H$_2$O       + h$\nu$         &$\!\!\!\rightarrow$ &  H + H + O                               & 0.08 & Sa03, Hu92\\ %
\refstepcounter{photo}P\arabic{photo}  & O$_2$        + h$\nu$         &$\!\!\!\rightarrow$ &  O       + O                          & 0.03 & Sa03\\ %
\refstepcounter{photo}P\arabic{photo}  & O$_2$        + h$\nu$         &$\!\!\!\rightarrow$ &  O      + O($^1$D)              & 0.97 & Sa03\\ %
 \refstepcounter{photo}P\arabic{photo}  & HO$_2$       + h$\nu$         &$\!\!\!\rightarrow$ &  O           + OH     & 1.0 & Sa03 \\ %
 \refstepcounter{photo}\label{PO3}P\arabic{photo}  & O$_3$       + h$\nu$         &$\!\!\!\rightarrow$ &  O($^1$D)     + O$_2$     & 0.95 &  note \\ %
 \refstepcounter{photo}P\arabic{photo}  & O$_3$       + h$\nu$         &$\!\!\!\rightarrow$ &  O           + O$_2$     & 0.05 &  \\ %
 \refstepcounter{photo}P\arabic{photo}  & H$_2$O$_2$       + h$\nu$         &$\!\!\!\rightarrow$ &  OH          + OH     & 1.0 & Hu92\\ %
 \refstepcounter{photo}P\arabic{photo}  &CO      + h$\nu$         &$\!\!\!\rightarrow$ &  C + O      & 1.0 & assumed \\ %
 \refstepcounter{photo}\label{PCO2}P\arabic{photo}  & CO$_2$       + h$\nu$         &$\!\!\!\rightarrow$ &  CO        + O          & 0.25 & note\\ %
 \refstepcounter{photo}P\arabic{photo}  & CO$_2$       + h$\nu$         &$\!\!\!\rightarrow$ &  CO       + O($^1$D)                  & 0.75 & Ok78,Hu92\\ %
 \refstepcounter{photo}P\arabic{photo}  & CO$_2$       + h$\nu$         &$\!\!\!\rightarrow$ &  C          + O$_2$                         & 0.001 &  note\\ %
 \refstepcounter{photo}P\arabic{photo}  & H$_2$CO      + h$\nu$         &$\!\!\!\rightarrow$ &  CO           + H$_2$                   & 0.6 & Sa03\\ %
 \refstepcounter{photo}P\arabic{photo}  & H$_2$CO    + h$\nu$   &$\!\!\!\rightarrow$ &  HCO  + H                                           & 0.3 & Sa03\\ %
 \refstepcounter{photo}P\arabic{photo}  & H$_2$CO      + h$\nu$         &$\!\!\!\rightarrow$ &  CO           + H           +H           & 0.1 & Sa03 \\ %
\refstepcounter{photo}\label{PCH3}P\arabic{photo}  & CH$_3$   + h$\nu$         &$\!\!\!\rightarrow$ &  $^1$CH$_2$          + H       &   1.0 & note\\ %
 \refstepcounter{photo}P\arabic{photo}  & CH$_4$       + h$\nu$         &$\!\!\!\rightarrow$ &  CH$_3$    + H                                   & $  0.42$ & Hu92\\ %
 \refstepcounter{photo}P\arabic{photo}  & CH$_4$       + h$\nu$         &$\!\!\!\rightarrow$ &  $^1$CH$_2$    + H$_2$                      & $  0.48$ & Hu92\\ %
 \refstepcounter{photo}P\arabic{photo}  & CH$_4$       + h$\nu$         &$\!\!\!\rightarrow$ &  CH$_2$   + H     + H                                   & $  0.03 $ & Hu92\\ %
\refstepcounter{photo}P\arabic{photo}  & CH$_4$       + h$\nu$         &$\!\!\!\rightarrow$ &  CH       + H     + H$_2$                                   & $  0.07 $ & Hu92\\ %
 \refstepcounter{photo}P\arabic{photo}  & CH$_3$OH     + h$\nu$     &$\!\!\!\rightarrow$ &  CH$_3$O + H    & $0.86\pm 0.1$ & Hu92 \\ %
 \refstepcounter{photo}P\arabic{photo}  & CH$_3$OH     + h$\nu$         &$\!\!\!\rightarrow$ &  CH$_3$ + OH                  & 0.07 & Hu92\\ %
 \refstepcounter{photo}P\arabic{photo}  & CH$_3$OH     + h$\nu$         &$\!\!\!\rightarrow$ &  H$_2$CO + H$_2$           & 0.07 & Hu92 \\ %
\refstepcounter{photo}P\arabic{photo}  & C$_2$H$_2$   + h$\nu$         &$\!\!\!\rightarrow$ &  C$_2$H       + H          & $  0.75$ & Ok78,Hu92\\ %
\refstepcounter{photo}P\arabic{photo}  & C$_2$H$_2$   + h$\nu$         &$\!\!\!\rightarrow$ &  C$_2$       + H$_2$                         & $  0.25$ & Ok78,Hu92\\ %
\refstepcounter{photo}P\arabic{photo}  & C$_2$H$_4$   + h$\nu$         &$\!\!\!\rightarrow$ &  C$_2$H$_2$       + H   + H      & $  0.5$ & Ok78,Hu92\\ %
\refstepcounter{photo}P\arabic{photo}  & C$_2$H$_4$   + h$\nu$         &$\!\!\!\rightarrow$ &  C$_2$H$_2$       + H$_2$                 & $  0.5$ & Ok78,Hu92\\ %
\refstepcounter{photo}\label{PC2H6}P\arabic{photo}  & C$_2$H$_6$   + h$\nu$   &$\!\!\!\rightarrow$ &  C$_2$H$_5$ + H          & $  0.326$ & note\\ %
\refstepcounter{photo}P\arabic{photo}  & C$_2$H$_6$   + h$\nu$         &$\!\!\!\rightarrow$ &  $^1$CH$_2$       + CH$_4$           & $  0.22$ & Hu92\\ %
\refstepcounter{photo}P\arabic{photo}  & C$_2$H$_6$   + h$\nu$         &$\!\!\!\rightarrow$ &  C$_2$H$_4$       + H$_2$            & $  0.355$ & Hu92\\ %
\refstepcounter{photo}P\arabic{photo}  & C$_2$H$_6$   + h$\nu$         &$\!\!\!\rightarrow$ &  CH$_3$       + CH$_3$              & $  0.088$ & Hu92\\ %
 \refstepcounter{photo}P\arabic{photo}  & CH$_2$CO     + h$\nu$         &$\!\!\!\rightarrow$ &  $^1$CH$_2$  + CO       & 1.0 & St68\\ %
  \refstepcounter{photo}P\arabic{photo}  & CH$_3$CHO     + h$\nu$         &$\!\!\!\rightarrow$ &  CH$_3$ + HCO     & 0.85 & Hu92\\ %
  \refstepcounter{photo}P\arabic{photo}  & CH$_3$CHO     + h$\nu$         &$\!\!\!\rightarrow$ &  CH$_4$ + CO                            & 0.1 & Hu92\\ %
  \refstepcounter{photo}P\arabic{photo}  & CH$_3$CHO     + h$\nu$         &$\!\!\!\rightarrow$ &  CH$_3$CO + H                            & 0.05 & Hu92\\ %
\refstepcounter{photo}\label{PC2H5OH}P\arabic{photo}  & C$_2$H$_5$OH   + h$\nu$  &$\!\!\!\rightarrow$ &  CH$_3$CHO  + H$_2$       & 0.8 &\\ %
\refstepcounter{photo}P\arabic{photo}  & C$_2$H$_5$OH   + h$\nu$         &$\!\!\!\rightarrow$ &  C$_2$H$_4$OH       + H      &   0.2 & note \\ %
\refstepcounter{photo}\label{PC4H2}P\arabic{photo}  & C$_4$H$_2$   + h$\nu$         &$\!\!\!\rightarrow$ &  C$_4$H        + H                 & 1.0  & assumed \\ %
 \refstepcounter{photo}\label{PN2}P\arabic{photo}  & N$_2$      + h$\nu$         &$\!\!\!\rightarrow$ &  N($^2$D) + N   & 1.0 & note \\ %
  \refstepcounter{photo}P\arabic{photo}  & NH$_3$       + h$\nu$         &$\!\!\!\rightarrow$ &  NH$_2$       + H                  & 0.94 & Ok78,Hu92\\ %
 \refstepcounter{photo}P\arabic{photo}  & NH$_3$       + h$\nu$         &$\!\!\!\rightarrow$ &  NH           + H$_2$                & 0.02 & Ok78,Hu92\\ %
 \refstepcounter{photo}P\arabic{photo}  & NH$_3$       + h$\nu$         &$\!\!\!\rightarrow$ &  NH    + H + H                     & 0.04 & Ok78,Hu92\\ %
 \refstepcounter{photo}P\arabic{photo}  & NO           + h$\nu$         &$\!\!\!\rightarrow$ &  N            + O                          & 1.0  & Sa03 \\ %
 \refstepcounter{photo}P\arabic{photo}  & HCN          + h$\nu$         &$\!\!\!\rightarrow$ &  CN           + H                        & 1.0 & Hu92\\ %
\refstepcounter{photo}P\arabic{photo}  & HNCO   + h$\nu$         &$\!\!\!\rightarrow$ &  NH    + CO       &   0.5 & Hu92\\ %
\refstepcounter{photo}P\arabic{photo}  & HNCO   + h$\nu$         &$\!\!\!\rightarrow$ &  H  +  NCO        &   0.5 & Hu92 \\ %
\refstepcounter{photo}P\arabic{photo}  & N$_2$H$_4$      + h$\nu$         &$\!\!\!\rightarrow$ &  N$_2$H$_3$    + H      & 1.0 & assumed \\ %
 \refstepcounter{photo}\label{PHNO}P\arabic{photo}  & HNO        + h$\nu$         &$\!\!\!\rightarrow$ &  NO          + H     & 1.0  & note \\ %
 \refstepcounter{photo}\label{PN2O}P\arabic{photo}  & N$_2$O      + h$\nu$         &$\!\!\!\rightarrow$ &  N$_2$ + O($^1$D)      & 1.0 & note \\ %
 \refstepcounter{photo}P\arabic{photo}  & NO$_2$       + h$\nu$         &$\!\!\!\rightarrow$ &  NO           + O       & 1.0 & Hu92 \\ %
 \refstepcounter{photo}P\arabic{photo}  & HNO$_2$       + h$\nu$         &$\!\!\!\rightarrow$ &  NO           + OH     & 1.0 & Hu92 \\ %
 \refstepcounter{photo}P\arabic{photo}  & HNO$_3$       + h$\nu$         &$\!\!\!\rightarrow$ &  NO$_2$    + OH     & 1.0 &  Hu92 \\ %
\refstepcounter{photo}\label{PCH2N2}P\arabic{photo}  & CH$_2$N$_2$   + h$\nu$         &$\!\!\!\rightarrow$ &  $^1$CH$_2$ + N$_2$      & 1.0 & note \\ %
   \refstepcounter{photo}P\arabic{photo}  & CH$_3$CN     + h$\nu$         &$\!\!\!\rightarrow$ &  CH$_3$ + CN                    & 0.5 & \\ %
  \refstepcounter{photo}P\arabic{photo}  & CH$_3$CN     + h$\nu$         &$\!\!\!\rightarrow$ &  CH$_2$CN  + H                & 0.5 & \\ %
\refstepcounter{photo}\label{PHCCCN}P\arabic{photo}  & HCCCN   + h$\nu$         &$\!\!\!\rightarrow$ &  C$_2$H          + CN       &   0.05 & note\\ %
 \refstepcounter{photo}P\arabic{photo}  & S$_2$        + h$\nu$         &$\!\!\!\rightarrow$ &  S            + S                   & 0.5 & Za09\\ %
 \refstepcounter{photo}P\arabic{photo}  & S$_3$        + h$\nu$         &$\!\!\!\rightarrow$ &  S$_2$        + S               & 1.0 & Za09\\ %
 \refstepcounter{photo}P\arabic{photo}  & S$_4$        + h$\nu$         &$\!\!\!\rightarrow$ &  S$_3$        + S              & 1.0 & Za09\\ %
 \refstepcounter{photo}\label{PS8}P\arabic{photo}  & S$_8$       + h$\nu$         &$\!\!\!\rightarrow$ &  S$_8^{\ast}$                       & 1.0 & note \\ %
 \refstepcounter{photo}\label{PS8*}P\arabic{photo}  & S$_8^{\ast}$       + h$\nu$         &$\!\!\!\rightarrow$ &  S$_4$     + S$_4$      & 1.0 & note \\ %
 \refstepcounter{photo}P\arabic{photo}  & HS           + h$\nu$         &$\!\!\!\rightarrow$ &  H            + S                    & 1.0 & Za09\\ %
 \refstepcounter{photo}P\arabic{photo}  & H$_2$S       + h$\nu$         &$\!\!\!\rightarrow$ &  HS           + H                     & 1.0 & Ok78,Hu92\\ %
 \refstepcounter{photo}P\arabic{photo}  & SO           + h$\nu$         &$\!\!\!\rightarrow$ &  S            + O                      & 1.0  & Ok78,Hu92\\ %
 \refstepcounter{photo}P\arabic{photo}  & SO$_2$       + h$\nu$         &$\!\!\!\rightarrow$ &  SO           + O                                       & 0.99 &  Ok78,Mi98\\ %
 \refstepcounter{photo}\label{PSO2}P\arabic{photo}  & SO$_2$       + h$\nu$         &$\!\!\!\rightarrow$ &  S            + O$_2$                                   & 0.01 &  Ok78,Mi98\\ %
 \refstepcounter{photo}\label{PHSO}P\arabic{photo}  & HSO          + h$\nu$         &$\!\!\!\rightarrow$ &  HS           + O         & 1.0 & note \\ %
 \refstepcounter{photo}\label{POCS}P\arabic{photo}  & OCS          + h$\nu$         &$\!\!\!\rightarrow$ &  CO           + S                  & 0.67 & note\\ %
\refstepcounter{photo}P\arabic{photo}  & OCS          + h$\nu$         &$\!\!\!\rightarrow$ &  CS           + O($^1$D)        & 0.05 & Hu92 \\ %
 \refstepcounter{photo}\label{PCS2}P\arabic{photo}  & CS$_2$       + h$\nu$         &$\!\!\!\rightarrow$ &  CS           + S                 & 1.0 & Mo81,Ah92\\ %
 \refstepcounter{photo}P\arabic{photo}  & SO$_3$       + h$\nu$         &$\!\!\!\rightarrow$ &  SO$_2$     + O       & 1.0 & assumed \\ %
 \refstepcounter{photo}P\arabic{photo}  & HCl       + h$\nu$         &$\!\!\!\rightarrow$ &  H     + Cl     & 1.0 & assumed \\ %
 \refstepcounter{photo}P\arabic{photo}  & HOCl       + h$\nu$         &$\!\!\!\rightarrow$ &  OH     + Cl      & 1.0 & Hu92 \\ %
 \refstepcounter{photo}P\arabic{photo}  & Cl$_2$      + h$\nu$         &$\!\!\!\rightarrow$ &  Cl     + Cl      & 1.0 & assumed \\ %



\hline
\hline
\multicolumn{6}{l}{$a$ --- Photolysis rates are computed at the top of the atmosphere for $I=100$ and a $30^{\circ}$ zenith angle. }\\
\multicolumn{6}{l}{$f$ --- Relative yield at the top of the atmosphere.}\\
%\end{tabular}
%\end{center}
%\label{default}
%\end{table}%

\end{longtable}  



\newpage
\noindent {\bf P\ref{PH2}.} 75\% lead to excited H$_2$.

\noindent {\bf P\ref{PO3}.} The actual products of the dominant channel are O($^1$D) + O$_2$($a^1\Delta_g$). 
We have omitted electronically excited O$_2$ in our network.  This may not be a good choice, but O$_3$ is a specialized molecule,
of exceptional interest in the search for habitability but not abundant in the kinds of gas giants that will be accessible in the near future.
In any event, we implicitly assume that O$_2$($a^1\Delta_g$) de-excites radiatively or collisionally.

\noindent {\bf P\ref{PCO2}.} The direct channel to CO + O($^3$P) is forbidden but does take place for $167\!<\!220$ nm, with a yield of order 1\%.
The permitted channel CO + O($^1$D) dominates the far UV, but at shorter wavelengths
 CO($a^3\Pi$) + O($^3$P) becomes important.  We have not included electronically excited CO($a^3\Pi$) because we do not know
 what it can do apart from de-excite, either collisionally or radiatively.  Hence for accounting purposes we have treated this path as a source of CO + O.
A peculiarity of CO$_2$ is that it has a very small photolysis cross section to Ly$\alpha$; moreover, a small fraction of
these Ly$\alpha$ photolyses emerge as C + O$_2$, a curiosity because it is a source of atomic carbon in an oxidized atmosphere.

\noindent {\bf P\ref{PCH3}.} Narrow band at $216\pm 1$ nm, said to be important on Titan.

\noindent {\bf P\ref{PC2H6}.} Entirely dominated by Ly$\alpha$.

\noindent {\bf P\ref{PC2H5OH}.} Minor channel assumed.

\noindent {\bf P\ref{PC4H2}.}  Assumes twice the cross section of C$_2$H$_2$.

\noindent {\bf P\ref{PN2}.} I can no longer find the source for this.

\noindent {\bf P\ref{PN2O}.} In reality this is 85\% O($^1$S) and only 15\% O($^1$D), but here we treat both as O($^1$D).

\noindent {\bf P\ref{PHNO}.} Assumes cross section of HO$_2$ (Sa03).

\noindent {\bf P\ref{PCH2N2}.} See MPI-Mainz UV/VIS Spectral Atlas www.uv-vis-spectral-atlas-mainz.org (Ke13)  for the unique source of the photolysis cross sections.

\noindent {\bf P\ref{PHCCCN}.} Photolysis dominated by Ly$\alpha$ (Hu92).  At longer wavelengths HCCCN is excited by not photolyzed.

\noindent {\bf P\ref{PS8}.} Mechanism assumed following Ka1989

\noindent {\bf P\ref{PS8*}.} Products assumed to keep this simple.

\noindent {\bf P\ref{PSO2}.} Hu92 gives 25\% for the S + O$_2$ channel; Mi98 has this close to zero.

\noindent {\bf P\ref{PHSO}.} Assumes cross section of HO$_2$ (Sa03).

\noindent {\bf P\ref{POCS}.} OCS photolysis is well studied (Sa03).
There are three CO + S channels, corresponding S($^3$P), S($^1$D), and S($^1$S). 
Doubtless the excited S atoms are more reactive than the already rather reactive S($^3$P).
In the immortal words of Dave Emory, ``Food for thought, and grounds for further research.''
In practice the CO + S($^3$P) will dominated because it goes in the near UV, at wavelengths where most
atmospheres are much less opaque than in the far UV.

\noindent {\bf P\ref{PCS2}.} Major products include four permutations of excited CS and S.
Unlike the superficially analogous CO$_2$, the most important products are the ground states CS($X^1\Sigma^+$) + S($^3$P),
which are favored by relatively low energy near UV ($\lambda \!>\!200$ nm) photons.


%\newpage
\medskip
\medskip

\noindent {\bf References}
%\medskip

\noindent {\bf Ad05.}
Adam, L., Hack, W., Zhu, H., Qu, Z.W., Schinke, R. (2005). Experimental and theoretical investigation of the reaction NH +H $\rightarrow$ N($^4$S) +H2. {\em J. Chem. Phys. 122,} 114301.

\noindent {\bf Ah92.}
Ahmed S.M., Kumar, V. (1992). Measurement of photoabsorption and fluorescence cross-sections for CS2 at 188.2213 and 287.5339.5 nm. {\em Pramana 39,} 367-380.

\noindent {\bf Alt15.}
Altinay, G.; Macdonald, R. G.
Determination of the Rate Constants for the NH2 + NH2 and NH2 + H Recombination Reactions in N2 as a Function of Temperature and Pressure
{\em J. Phys. Chem. A 119,} 7593 - 7610.

\noindent {\bf An18.}
Anglada, J.M.; Crehuet, R.; Adhikari, S.; Francisco, J.S.; Xia, Y. (2018).
 Reactivity of hydropersulfides toward the hydroxyl radical unraveled: disulfide bond cleavage, hydrogen atom transfer, and proton-coupled electron transfer.
 {\em Proc. Combust. Inst. 29,}  2439 - 2446.

\noindent {\bf An20.}
Antonsen, S.G.; Bunkan, A.J.C.; Mikoviny, T.; Nielsen, C.J.; Stenstrom, Y.; Wisthaler, A.; Zardin, E. (2020).
Atmospheric chemistry of diazomethane - an experimental and theoretical study
{\em Mol. Phys. 118}

\noindent {\bf Ar81.} Arai, H., Nagai, S., Hatada, M. (1981) Radiolysis of methane containing small amounts of carbon monoxide-formation of organic acids.  {\em Radiat. Phys. Chem.  17}

\noindent {\bf At89.}
Atkinson, R., Baulch, D.L., Cox, R.A., Hampson Jr., R.F., Kerr, J.A., Troe, J. (1989). Evaluated Kinetic and Photochemical Data for Atmospheric Chemistry: Supplement III.  {\em J. Phys. Chem. Ref. Data 18,}  881-1097

\noindent {\bf At04.}
Atkinson, R., Baulch, D.L., Cox, R.A., Crowley, J.N., Hampson, R.F., Hynes, R.G., Jenkin, M.E., Rossi, M.J., Troe, J., (2004). Evaluated kinetic and photochemical data for atmospheric chemistry: Volume I - gas phase reactions of Ox, HOx, NOx and SOx species.  {\em Atmos. Chem. Phys.  4,} 1461-1738.

\noindent {\bf At07.}
Atkinson, R.;Baulch, D.L.;Cox, R.A.;Crowley, J.N.;Hampson, R.F.;Hynes, R.G.;Jenkin, M.E.;Rossi, M.J.;Troe, J. (2007).  Evaluated kinetic and photochemical data for atmospheric chemistry: Volume III - gas phase reactions of inorganic halogens.
{\em Atmos. Chem. Phys. 7,} 981 - 1191.

\noindent {\bf Ba67.}
Basco N and Pearson AE (1967). Reactions of sulphur atoms in presence of carbon disulphide, carbonyl sulphide and nitric oxide. {\em Trans.\ Faraday Soc.\ 63,} 2684-2689.

\noindent {\bf Ba71.} 
Bauer, S.H.; Jeffers, P.; Lifshitz, A.; Yadava, B.P. (1971).
Reaction between CO and SO2 at Elevated Temperatures: A Shock-Tube Investigation
{\em  Symp. Int. Combust. Proc. 13.}

\noindent {\bf Ba81.}
Baulch, D.L.; Duxbury, J.; Grant, S.J.; Montague, D.C. (1981).  Evaluated kinetic data for high temperature reactions. Volume 4 Homogeneous gas phase reactions of halogen- and cyanide- containing species. {\em J. Phys. Chem. Ref. Data 10,} 

\noindent {\bf Ba82.}
Bartels, M., Hoyermann, K., Sievert, R. (1982).  Elementary Reactions in the Oxidation of Ethylene: The Reaction of OH Radicals with Ethylene and the Reaction of C2H4OH Radicals with H Atoms.  {\em Symp. Int. Combust. Proc. 19,} 61-72.

\noindent {\bf Ba88.}
Basevich, V.Ya., Vedeneev, V.I. (1988). Reaction of nitrogen atoms with ammonia and hydrogen. {\em Khim. Fiz. 7,} 1552 - 1558.

\noindent {\bf Ba91.}
Balla, R.J., Casleton, K.H., Adams, J.S., Pasternack, L. (1991). Absolute rate constants for the reaction of CN with CH4, C2H6, and C3H8 from 292 to 1500 K using high-temperature photochemistry and diode laser absorption.  {\em J. Phys. Chem. 95,} 8694-8701.

\noindent {\bf Bar91.}
Bartels, M.; Edelbuttel-Einhaus, J.; Hoyermann, K. (1991).
The detection of CH3CO, C2H5, and CH3CHO by rempi/mass spectrometry and the application to the study of the reactions H + CH3CO and O + CH3CO
{\em Symp. Int. Combust. Proc. 23,} 131-138

\noindent {\bf Ba92.}
Baulch, D.L., Cobos, C.J., Cox, R.A., Esser, C., Frank, P., Just, Th., Kerr, J.A., Pilling, M.J., Troe, J., Walker, R.W., Warnatz, J. (1992). Evaluated kinetic data for combustion modeling. {\em J. Phys. Chem. Ref. Data. 21,} 411-429. 

\noindent {\bf Ba94.}
Baulch, D.L., Cobos, C.J., Cox, R.A., Frank, P., Hayman, G., Just, Th., Kerr, J.A., Murrells, T., Pilling, M.J., Troe, J., Walker, R.W., Warnatz, J. (1994).  Evaluated kinetic data for combusion modeling. {\em Supplement I. J. Phys. Chem. Ref. Data 23,} 847-1033.

\noindent {\bf Ba95.}
Bauerle, S., Klatt, M., Wagner, H.Gg. (2005). Recombination and decomposition of methylene radicals at high temperatures. {\em Ber. Bunsenges. Phys. Chem. 99,} 870-879.

\noindent {\bf Ba07.}
Ballester, M.Y., Caridade, P.JSB., Varandas, A.JC. (2007). Dynamics and kinetics of the H+SO2 reaction: A theoretical study. {\em Chem. Phys. Lett. 439,} 301-307.

\noindent {\bf Be78.}
Benson SW (1978). Thermochemistry and Kinetics of Sulfur-Containing Molecules and Radicals. {\em Chem.\ Rev.\ 78,} 23-35.

\noindent {\bf Be84.}
Berman, M.R. Lin, M.C. (1984). Kinetics and mechanisms of the reactions of CH and CD with H2 and D2. {\em J. Chem. Phys. 81,} 5743-5752.

\noindent {\bf Be92}
Becker, E.; Rahman, M.M.; Schindler, R.N. (1992). Determination of the rate constants for the gas phase reactions of NO3 with H, OH and HO2 radicals at 298 K. {\em Ber. Bunsenges. Phys. Chem. 96,} 776 - 783.

\noindent {\bf Be98}
Bergeat, A.; Calvo, T.; Daugey, N.; Loison, J.C.; Dorthe, G. (1998)
 Product branching ratios of the CH + NO reaction.
{\em J. Phys. Chem. A 102,} 8124 - 8130.

\noindent {\bf Be00.}
Becker KH, Kurtenbach R, Schmidt F, Weisen P (2000). Kinetics of the NCO radical reacting with atoms and selected molecules. {\em Combust. Flames 120,} 570-577.

\noindent {\bf Be02.}
 Berdyugina SV and W. C. Livingston WC (2002). Detection of the mercapto radical SH in the solar atmosphere. {\em Astron.\ Astrophys.\ 387,} L6-L9.

%\noindent Bi91.
% Billmers RI and Smith AL (1991). Ultraviolet-Visible Absorption Spectra of Equilibrium Sulfur Vapor: Molar Absorptlvlty
%Spectra of S$_3$ and S$_4$. {\em J.\ Phys.\ Chem.\ 95,} 4242-4245.

\noindent {\bf Bi06.}
Biczysko, M, Poveda, Varandas, A.J.C. (2006). Accurate MRCI study of ground-state N2H2 potential energy surface. {\em Chem Phys Lett 424,} 4653,

\noindent {\bf Bl00.}
Blitz, M.A., McKee, K.W., Pilling, M.J. (2000). Temperature dependence of the reaction of OH with SO. {\em Proc. Combust. Inst. 28,} 2491-2497.

\noindent {\bf Bl06.}
Blitz, M.A., Hughes, K.J., Pilling, M.J., Robertson, S.H. (2006).  Combined experimental and master equation investigation of the multiwell reaction H+SO2.  {\em J. Phys. Chem. A  110,} 2996-3009.

\noindent {\bf Bo85.}
Bohland, T., Dobe, S., Temps, F., Wagner, H.Gg. (1985).  Kinetics of the reactions between CH2(X$^3$B$_1$)-radicals and saturated hydrocarbons in the temperature range 296 K to 707 K (1985). {\em Ber. Bunsenges. Phys. Chem. 89,} 432.

\noindent {\bf Bo96.}
Bohn, B., Siese, M., Zetzsch, C. (1996). Kinetics of the OH + C2H2 reaction in the presence of O2. {\em J. Chem. Soc. Faraday Trans. 92,} 1459-1466.

\noindent {\bf Bo97}  Boughton, J.W.; Kristyan, S.; Lin, M.C. (1997).  Theoretical study of the reaction of hydrogen with nitric acid: ab initio MO and TST/RRKM calculations. {\em Chem. Phys. 214,} 219 - 227.

\noindent {\bf Bo04}
Bogdanchikov, G.A.; Bakanov, A.; Parker, D.H. (2004), The substitution reactions RH+O2->RO2+H: transition state theory calculations based on the ab initio and DFT potential energy surface.  {\em Chem. Phys. Lett. 385,} 486 - 490.

\noindent {\bf Bo20}
Bowman, M.C.; Burke, A.D.; Turney, J.M.; Schaefer, H.F. (2020).
Conclusive determination of ethynyl radical hydrogen abstraction energetics and kinetics.
{\em  Mol. Phys. 118.}

\noindent {\bf Br97.}
Brownsword, R.A., Canosa, A., Rowe, B.R., Sims, I.R., Smith, I.W.M., Stewart, D.W.A., Symonds, A.C., Travers, D. (1997). Kinetics over a wide range of temperature (13-744 K): rate constants for the reactions of CH($\nu$=$0$) with H2 and D2 and for the removal of CH($\nu$=$1$) by H2 and D2. {\em J. Chem. Phys. 106,} 7662-7677.

\noindent {\bf Br04.}
Bryukov, M.G.; Knyazev, V.D.; Lomnicki, S.M.; McFerrin, C.A.; Dellinger, B. (2004).  Temperature-dependent kinetics of the gas-phase reactions of OH with Cl2, CH4, and C3H8. {\em J. Phys. Chem. A 108,} 10464 - 10472.

\noindent {\bf Bu90.}
Burmeister, M., Roth P (1990).  {\em AIAA J.\  28,} 402-405.

\noindent {\bf Bu14}
Bunkan, A. J. C. (2014).
A theoretical and experimental study of atmospheric reactions of amines and their degradation products.
PhD thesis No.\ 1588.
Department of Chemistry, Faculty of Mathematics and Natural Sciences University of Oslo, Norway.

\noindent {\bf Bu16}
Bunkan, A. J. C. (2016).

\noindent {\bf Ca73}
Campbell, I.M.; Gray, C.N. (1973)
Rate Constants for O(3P) Recombination and Association with N(4S)
{\em Chem. Phys. Lett. 18}

\noindent {\bf Ca01.}
Campomanes, P., Menendez, I., Sordo, T.L. (2001). A Theoretical Study of the NCO + OH Reaction. {\em J. Phys. Chem. A 105,}  229 - 237.

\noindent {\bf Ca03.}
Carl, S.A., Sun, Q.,Teugels, L., Peeters, J.  (2003).  Experimental determination of the temperature dependence of the absolute rate coefficients of the HCCO+NO2 and HCCO+H2 reactions.  {\em Phys. Chem. Chem. Phys. 5,}  5424 - 5430.

\noindent {\bf Ca05.}
Carl, S.A., Nguyen, H.MT., Elsamra, R.MI., Nguyen, M.T., Peeters, J. (2005). Pulsed laser photolysis and quantum chemical-statistical rate study of the reaction of the ethynyl radical with water vapor.  {\em J. Chem. Phys. 122,} 114307.

\noindent {\bf Ca05b.}
Caridade PJSB, Rodrigues SPJ, Sousa F, and Varandas AJC. (2005). Unimolecular and Bimolecular Calculations for HN2. {\em J. Phys. Chem. A  109,} 2356-2363.

\noindent {\bf Ca08.}
Carvalho, E.FV.; Barauna, A.N.; Machado, F.BC.; Roberto, O. (2008).  Theoretical calculations of energetics, structures, and rate constants for the H+CH3OH hydrogen abstraction reactions. {\em Chem. Phys. Lett.  463,}  33 - 37.

\noindent {\bf Cd05.}
Caridade, P.JSB., Rodrigues, S.PJ., Sousa, F., Varandas, A.JC. (2005). Unimolecular and bimolecular calculations for HN2.   {\em J. Phys. Chem. A  109,} 2356-2363. 

\noindent {\bf Ch75.}
Chung, K.; Calvert, J.G.; Bottenheim, J.W. (1975).
The Photochemistry of Sulfur Dioxide Excited within its First Allowed Band (313 nm) and the ``Forbidden'' Band (370-400 nm)
{\em  Int. J. Chem. Kinet. 7}

\noindent {\bf Ch03.}
Che, C.-b., Zhang, H.; Zhang, X., Liu, Y., Liu, B. (2003). Ab Initio and Kinetic Study on CH3 Radical Reaction with H2CO.  {\em J. Phys. Chem. A 107,} 2929 - 2933.

\noindent {\bf Ch05}
Choi, Y.M.; Lin, M.C.  (2005). Kinetics and mechanisms for reactions of HNO with CH3 and C6H5 studied by quantum-chemical and statistical-theory calculations.   Inter. J. Chem. Kinet.   37,  261 - 274. 

\noindent {\bf Cl06.}
Cleary, P.A., Romero, M.TB., Blitz, M.A., Heard, D.E., Pilling, M.J., Seakins, P.W., Wang, L. (2006). Determination of the temperature and pressure dependence of the reaction OH + C2H4 from 200-400 K using experimental and master equation analyses. {\em Phys. Chem. Chem. Phys. 8,} 5633 - 5642.

\noindent {\bf Ch98.}
Chase MW (1998) {\em NIST-JANAF Themochemical Tables, Fourth Edition, J.\ Phys.\ Chem.\ Ref.\ Data, Monograph 9,} 1-1951.

\noindent {\bf Cl84.}
Clyne, M.A.A.; MacRobert, A.J.; Murrells, T.P.; Stief, L.J. (1984). Kinetics of the reactions of atomic chlorine with H2S, HS and OCS.
{\em J. Chem. Soc. Faraday Trans. 2, 80,} 877 - 886.

\noindent {\bf Cl06.}
Cleary, P.A.,Romero, M.TB.,Blitz, M.A.,Heard, D.E.,Pilling, M.J.,Seakins, P.W.,Wang, L. (2006). Determination of the temperature and pressure dependence of the reaction OH + C2H4 from 200-400 K using experimental and master equation analyses. {\em Phys. Chem. Chem. Phys. 8,} 5633-5642.

\noindent {\bf Co85.}
Cobos, C.J. Troe, J. (1985). Theory of thermal unimolecular reactions at high pressures. II. Analysis of experimental results. {\em  J. Chem. Phys. 83,} 1010-1015.

\noindent {\bf Co91.}
Cohen, N. Westberg, K.R. (1991).  Chemical kinetic data sheets for high-temperature reactions. Part II.
{\em J. Phys. Chem. Ref. Data  20,}  1211-1311.

\noindent {\bf Co92.}
Cooper, W.F., Hershberger, J.F. (1992).  An infrared laser study of the O($^3$P) + CS2 reaction.   {\em J. Phys. Chem. 96,} 5405-5410.

\noindent {\bf Co97.}
Corchado, J.C., Espinosa-Garcia, J. (1997). Analytical potential energy surface for the NH3+H=NH2+H2 reaction: application of variational transition-state theory and analysis of the equilibrium constants and kinetic isotope effects using curvilinear and rectilinear coordinates. {\em J. Chem. Phys. 106,} 4013-4021.

\noindent {\bf Co99.}
Cox, R.M.; Plane, J.M.C. (1999). An Experimental and Theoretical Study of the Reactions NaO + H2O(D2O) ?? NaOH(D) + OH(OD)
{\em Phys. Chem. Chem. Phys. 1,} 4713 - 4720

\noindent{\bf Cv87}
Cvetanovic, R.J. (1987) Evaluated chemical kinetic data for the reactions of atomic oxygen O(3P) with unsaturated hydrocarbons.
{\em J. Phys. Chem. Ref. Data 16,}

\noindent {\bf Cu06.}
Curran, H.J. (2006). Rate constant estimation for C-1 to C-4 alkyl and alkoxyl radical decomposition.  {\em Int. J. Chem. Kinet. 38,} 250-275.

\noindent {\bf Da90.}
Davidson DF, Kohse-Hoinghaus K, Chang AY, Hanson RK (1990). A pyrolysis mechanism for ammonia.  {\em Int J Chem Kin 22,} 513-535.

\noindent {\bf Da95.}
Darwin, D.C., Moore, C.B. (1995).  Reaction rate constants (295 K) for $^3$CH2 with H2S, SO2, and NO2: upper bounds for rate constants with less reactive partners. {\em J. Phys. Chem. 99,} 13467-13470.

\noindent {\bf De82.}
Demissy, M. Lesclaux, R. (1982). Absolute Rate Constants for the Reactions between Amino and Alkyl Radicals at 298 K.  {\em Int. J. Chem. Kinet. 14,} 1-12.

\noindent {\bf De91.}
Dean, A.J.; Hanson, R.K.; Bowman, C.T. (1991). A shock tube study of reactions of C atoms and CH with NO including product channel measurements.  {\em J. Phys. Chem. 95,} 3180-3189.

\noindent {\bf De92.}
Dean, A.J., Hanson, R.K. (1992).  CH and C-atom time histories in dilute hydrocarbon pyrolysis: measurements and kinetics calculations.  {\em Int. J. Chem. Kinet. 24,} 517-532.

\noindent {\bf De95.}
Devriendt, K., Van Poppel, M., Boullart, W., Peeters, J. (1995).  Kinetic investigation of the CH2(X$^3$B$_1$) + H $\rightarrow$ CH(X$^2$II) + H2 reaction in the temperature range $400{\rm~K}<T<1000$ K. {\em J Phys Chem 99,} 16953-16959.

\noindent {\bf De97.}
DeMore, W.B., Sander, S.P., Golden, D.M., Hampson, R.F., Kurylo, M.J., Howard, C.J., Ravishankara, A.R., Kolb, C.E., Molina, M.J. (1997). {\em Chemical kinetics and photochemical data for use in stratospheric modeling. Evaluation number 12.}  JPL Publication 97-4.

\noindent {\bf De98.} 
Deppe, J., Friedrichs, G., Ibrahim, A., Romming, H.-J., Wagner, H.Gg. (1998). The thermal decomposition of NH2 and NH radicals.   {\em Ber. Bunsenges. Phys. Chem. 102,} 1474-1485.

\noindent {\bf De00.} 
Dean, AM and Bozzelli, JW (2000). Combustion chemistry of nitrogen. In Gas-Phase Combustion Chemistry, WC Gardiner, Jr., Ed. Springer, pp 125-341.

\noindent {\bf Di95}
Diau, E.W.; Halbgewachs, M.J.; Smith, A.R.; Lin, M.C. (1995). Thermal reduction of NO by H2: kinetic measurement and computer modeling of the HNO + NO reaction.  Int. J. Chem. Kinet. 27, 867 - 881.  

\noindent {\bf Dob91.}
Dobe, S., Berces, T., Szilagyi, I. (1991). Kinetics of the reaction between methoxyl radicals and hydrogen atoms. {\em J. Chem. Soc. Faraday Trans. l:  87,} 2331-2336.

\noindent {\bf Do18.}
Douglas, K.; Blitz, M.A.; Feng, W.; Heard, D.E.; Plane, J.M.C.; Slater, E.; Willacy, K.; Seakins, P.W. (2018)
 Low temperature studies of the removal reactions of (CH2)-C-1 with particular relevance to the atmosphere of Titan
{\em Icarus 303,} 10 - 21.

\noindent {\bf Du08.}
 Du, S.Y., Francisco, J.S., Shepler, B.C., Peterson, K.A. (2008). Determination of the rate constant for sulfur recombination by quasiclassical trajectory calculations. {\em J. Chem. Phys. 128,} 204306.
 
\noindent{Ed92}
 Edelbuttel-Einhaus, J.; Hoyermann, K.; Rohde, G.; Seeba, J. (1992).
Title:   The detection of the hydroxyethyl radical by REMPI/mass-spectrometry and the application to the study of the reactions CH3CHOH + O and CH3CHOH + H
{\em Symp. Int. Combust. Proc.  24,}  661 - 668.

\noindent {\bf Ei98.}
Eiteneer, B., Yu, C.-L., Goldenberg, M., Frenklach, M. (1998). Determination of rate coefficients for reactions of formaldehyde pyrolysis and oxidation in the gas phase. {\em J. Phys. Chem. A  102,} 5196-5205.

\noindent {\bf Ei03.}
Eiteneer, B., Frenklach, M. (2003).  Experimental and Modeling Study of Shock-Tube Oxidation of Acetylene. {\em Int J. Chem. Kinet. 35,} 391-414.

\noindent {\bf Fa67.}
Fair, R.W.; Thrush, B.A. (1967).  Mechanism of S2 chemiluminescence in the reaction of hydrogen atoms with hydrogen sulphide. { \em Trans. Faraday Soc. 65,} 1208-1218.

\noindent {\bf Fa93}
Fagerstrom, K.; Jodkowski, J.T.; Lund, A.; Ratajczak, E. (1995). Kinetics of the self-reaction and the reaction with OH of the amidogen radical. Chem. Phys. Lett. 236, 103-110.

\noindent {\bf Fa95}
Fagerstrom, K.; Lund, A.; Mahmoud, G.; Jodkowski, J.T.; Ratajczak, E. (1993)
Kinetics of the gas-phase reaction between ethyl and hydroxyl radicals	{\em Chem. Phys. Lett. 208,} 321-327.
	
\noindent {\bf Fa00.}
Faravelli, T., Goldaniga, A., Zappella, L., Ranzi, E., Dagaut, P., Cathonnet, M. (2000). An experimental and kinetic modeling study of propyne and allene oxidation. {\em Proc. Combust. Inst. 28,} 2601 - 2608.

\noindent {\bf Fe98.}
Fernandez, A., Goumri, A., Fontijn, A. (1998). Kinetics of the reactions of N($^4$S) atoms with O2 and CO2 over wide temperatures ranges. {\em J. Phys. Chem. A 102,} 168-172.

\noindent {\bf Fl02.}
Fleurat-Lessard, P., Rayez, J.C., Bergeat, A., Loison, J.C. (2002). Reaction of methylidyne CH(X$^2\pi$) radical with CH4 and H2S: overall rate constant and absolute atomic hydrogen production. {\em Chem. Phys. 279,} 87-99. %products

\noindent {\bf Fo06.}
Fontijn, A., Shamsuddin, S.M., Crammond, D., Marshall, P., Anderson, W.R. (2006).   Kinetics of the NH reaction with H2 and reassessment of HNO formation from NH + CO2, H2O. {\em Combust. Flame 145,} 543-551.

\noindent {\bf Fr84.}
Frank, P., Just, Th. (1984). High temperature kinetics of ethylene-oxygen reaction. {\em  Proc. Int. Symp. Shock Tubes Waves 14,} 706.

\noindent {\bf Fr88.}
Frank, P., Bhaskaran, K.A., Just, Th. (1988). Acetylene oxidation: the reaction of C2H2 + O at high temperatures.  {\em Symp. Int. Combust. Proc. 21,} 885 - 893.

\noindent {\bf Fr02.} Friedrichs, G.; Herbon, J.T.; Davidson, D.F.; Hanson, R.K. (2002)   Quantitative Detection of HCO Behind Shock Waves: The Thermal Decomposition of HCO.  {\em Phys. Chem. Chem. Phys. 4,} 5778 - 5788.

\noindent {\bf Fu97a.}
Fulle, D., Hippler, H. (1997).  The temperature and pressure dependence of the reaction CH+H2 $\rightarrow$ CH3 $\rightarrow$ CH2+H.  {\em J. Chem. Phys. 106,} 8691-8698.

\noindent {\bf Fu97b.}
Fulle, D., Hamann, H.F., Hippler, H., Jansch, C.P.  (1997).  The high pressure range of the addition of OH to C2H2 and C2H4. {\em Ber. Bunsenges. Phys. Chem. 101,} 1443-1442.

\noindent {\bf Fu98.} Fulle D, Hippler H, Striebel F (1998). The high pressure range of the reaction CH+CO+M=HCCO+M. {\em J. Chem. Phys. 108,} 6709-6716.

\noindent {\bf Ga98.}
Garland NL (1998). Temperature dependence of the reaction: SO + O2. {\em Chem. Phys. Lett. 290,} 385-390.

\noindent {\bf Ga06.}
Gao, Y.D.; Alecu, I.M.; Hsieh, P.C.; Morgan, B.P.; Marshall, P.; Krasnoperov, L.N. (2006). Thermochemistry is not a lower bound to the activation energy of endothermic reactions: A kinetic study of the gas-phase reaction of atomic chlorine with ammonia.  {\em J. Phys. Chem. A 100,}  6844 - 6850.

\noindent {\bf Ga07.}
Gannon, K.L., Glowacki, D.R., Blitz, M.A., Hughes, K.J., Pilling, M.J., Seakins, P.W. (2007). H atom yields from the reactions of CN radicals with C2H2, C2H4, C3H6, trans-2-C4H8, and iso-C4H8. {\em J. Phys. Chem. A 111,}  6679-6692.

\noindent {\bf Ga08.}
Gannon, K.L.; Blitz, M.A.; Pilling, M.J.; Seakins, P.W.; Klippenstein, S.J.; Harding, L.B. (2008). Kinetics and product branching ratios of the reaction of CH2 (singlet) with H2 and D2.  {\em J. Phys. Chem. A 112,} 9575 - 9583.

\noindent {\bf Ga19.}
Garcia, E.; Jambrina, P.G.; Lagana, A. (2019).
Kinetics Of The H + CH2 $\rightarrow$ CH + H-2 Reaction At Low Temperature.
{\em J. Phys. Chem. A  123,} 7408 - 7419.

\noindent {\bf Ge99.}
Geiger, H., Wiesen, P., Becker, K.H. (1999).  A Product Study of the Reaction of CH Radicals with Nitric Oxide at 298 K.  {\em Phys. Chem. Chem. Phys. 1,} 5601 - 5606.

\noindent {\bf Gl00}
Glass, G.P., Kumaran, S.S., Michael, J.V. (2000).  Photolysis of Ketene at 193 nm and the Rate Constant for H + HCCO at 297 K.  {em J. Phys. Chem. A 104,} 8360 - 8367.

\noindent {\bf Gl08}
Glassman, I., Yetter, R.  (2008).  {\em Combustion.}  Academic Press. 800pp.

\noindent {\bf Gl15}
Glarborg, P.; Marshall, P.; Troe, J. (2015).
 Temperature and Pressure Dependence of the Reaction S plus CS ( plus M) $\rightarrow$ CS2 (+M).
{\em J. Phys. Chem. 119,} 7277 - 7281.

\noindent {\bf Gh88}
Ghibaudi, E.; Colussi, A.J.  (1988)
Kinetics and thermochemistry of the equilibrium 2 (acetylene) = vinylacetylene. Direct evidence against a chain mechanism.
{\em J. Phys. Chem. 92}

\noindent {\bf Go08.}
Golden DM (2008). Yet another look at the reaction CH3+H+M $\rightarrow$ CH4+M. {\em Int. J. Chem. Kinet. 40,} 310-319.

\noindent {\bf Gro89.}
Grotheer, H.; Riekert, G.; Walter, D.; Just, Th. (1989).
Reactions of hydroxymethyl and hydroxyethyl radicals with molecular and atomic oxygen
{\em Symp. Int. Combust. Proc.  22,} 963 - 972.

\noindent {\bf Gr94.}
Grussdorf, J., Nolte, J., Temps, F., Wagner, H.Gg. (1994).  Primary products of the elementary reactions of CH2CO with F, Cl, and OH in the gas phase.  {\em Ber. Bunsenges. Phys. Chem.  98,} 546 - 553.

\noindent {\bf Ha84.}
Hanson, R.K., Salimian, S. (1984). Survey of rate constants in the N/H/O system. In {\em Combustion Chemistry.} Ed. W.C. Gardiner,Jr., Springer-Verlag, NY, p.\ 361.

\noindent {\bf Ha93.}
Harding, L.B., Guadagnini, R., Schatz, G.C. (1993). Theoretical studies of the reactions H + CH $\rightarrow$ C + H2 and C + H2 $\rightarrow$ CH2 using an ab initio global ground-state potential surface for CH2. {\em J. Phys. Chem. 97,} 5472-5481.

\noindent {\bf Ha03}
Haworth, N.L.; Mackie, J.C.; Bacskay, G.B. (2003)  An Ab Initio Quantum Chemical and Kinetic Study of the NNH + O Reaction Potential Energy Surface: How Important Is This Route to NO in Combustion?   
{\em J. Phys. Chem. A  107,}  6792 - 6803.

\noindent {\bf Ha04}
Harrington

\noindent {\bf Ha05.}
Harding, L.B., Klippenstein, S.J., Georgievskii, Y. (2005). Reactions of oxygen atoms with hydrocarbon radicals: a priori kinetic predictions for the CH3+O, C2H5+O, and C2H3+O reactions. {\em Proc. Combust. Inst. 30,} 985-993.

\noindent {\bf Ha07.}
Harding, L.B.; Klippenstein, S.J.; Georgievskii, Y. (2007)
On the combination reactions of hydrogen atoms with resonance-stabilized hydrocarbon radicals
{\em J. Phys. Chem. A 111,} 3789 - 3801.

\noindent {\bf He80}
Herron, J.T.; Huie, R.E. (1980)
Title:   Rate Constants at 298 K for the Reactions SO + SO + M $\rightarrow$ (SO)$_2$ + M and SO + (SO)$_2$ $\rightarrow$ SO$_2$ + S$_2$O.
{\em  Chem. Phys. Lett., 76}

\noindent {\bf He88}
He, Y.; Sanders, W.A.; Lin, M.C. (1988).  Thermal Decomposition of Methyl Nitrite: Kinetic Modeling of Detailed Product Measurements by Gas-Liquid Chromatography and Fourier Transform Infrared Spectroscopy.   J. Phys. Chem. 92, 5474.

\noindent {\bf Her88}
Herron, J.T. (1988). Evaluated chemical kinetic data for the reactions of atomic oxygen O(3P) with saturated organic compounds in the gas phase
{\em J. Phys. Chem. Ref. Data 17,} 1988.

\noindent {\bf He92}
He, Y.; Lin, M.C. (1992).   Effects of nitric oxide on the thermal decomposition of methyl nitrite: overall kinetics and rate constants for the HNO + HNO and HNO + 2NO reactions.  {\em Int. J. Chem. Kinet. 24,} 743 - 760.

\noindent {\bf He93}
He, Y.; Liu, X.; Lin, M.C.; Melius, C.F. (1993).  Thermal reaction of HNCO with NO2 at moderate temperatures.  {\em Int. J. Chem. Kinet. 25,} 845 - 863.

\noindent {\bf He95.}
Hennig, G., Wagner, H.Gg. (1995). A study about the reactions of NH2(X$^2$B$_1$) radicals with unsaturated hydrocarbons in the gas phase.  {\em Ber. Bunsenges. Phys. Chem. 99,} 989-994.

\noindent {\bf He99.}
Evaluated Chemical Kinetics Data for Reactions of N(2D), N(2P), and N2(A$^3\Sigma^+_u$) in the Gas Phase
{\em Journal of Physical and Chemical Reference Data 28,} 1453 (1999)

\noindent {\bf Hi87.}
Hills, A.J., Cicerone, R.J., Calvert, J.G., Birks, J.W. (1987).  Kinetics of the reactions of S2 with O, O2, O3, N2O, NO, and NO2. {\em J. Phys. Chem. 91,} 1199-1204.

\noindent {\bf Hi89.}
Hidaka, Y., Oki, T., Kawano, H. (1989).  Thermal decomposition of methanol in shock waves.  {\em J. Phys. Chem. 93,} 7134 - 7139.

\noindent {\bf Hi96.}
Hidaka Y, Hattori K, Okuno T, Inami K, Abe T, Koike T (1996).  Shock tube and modeling study of acetylene pyrolysis and oxidation.  {\em Combust. Flames 107,} 401-417.

\noindent {\bf Hi00.}
Hidaka, Y.; Sato, K.; Yamane, M. (2000).  High-temperature pyrolysis of dimethyl ether in shock waves. {\em Combust. Flame 123,} 1-22.

\noindent {\bf Hi01.}
Hippler, H., Striebel, F., Viskolcz, B. (2001) A Detailed Experimental and Theoretical Study on the Decomposition of Methoxy Radicals.  {\em Phys. Chem. Chem. Phys.  3,} 2450 - 2458.

\noindent {\bf Hi07}
Hindiyarti, L.; Glarborg, P.; Marshall, P. (2007).  Reactions of SO3 with the O/H radical pool under combustion conditions. {\em  J. Phys. Chem. A 111,} 3984 - 3991.

\noindent {\bf Ho95}
Hoyermann, K.; Seeba, J. (1995).
 Mechanism and rate of the reaction of cyanomethyl radicals with oxygen atoms in the gas phase
{\em Z. Phys. Chem. (Munich) 188,} 215 - 226.

\noindent {\bf Ho97}
Hoobler, R.J.; Leone, S.R. (1997)
Rate coefficients for reactions of ethyl radical (C2H) with HCN and CH3CN: implications for the formation of complex nitriles on Titan.
{\em J. Geophys. Res. 102,} 28717 - 28723. 

\noindent {\bf Hu75.}
Husain, D. Young, A.N. (1975). Kinetic investigation of ground state carbon atoms, C(2$^3$P$_j$).  {\em J. Chem. Soc. Faraday Trans. 2, 71,} 525.

\noindent {\bf Hu85.}
Husain, D.; Plane, J.M.C.; Xiang, C.C. (1985).  Absolute rate data for Rb + OH + He determined by time-resolved molecular resonance-fluoroescence spectroscopy, OH(A$^2\Sigma$ - X$^2\pi$), coupled with steady atomic resonance-fluoroescence measurements, Rb(6$^2$P$_{\rm J}$ - 5$^2$S$_{1/2}$) {\em J. Chem. Soc. Faraday Trans. 2, 81,}

\noindent {\bf Hu86.}
Husain, D.; Marshall, P. (1986). Determination of absolute rate data for the reactions of atomic sodium, Na(3$^2$S$_{1/2}$), with CH$_3$F, CH$_3$Cl, CH$_3$Br, HCl, and HBr as a function of temperature by time-resolved atomic resonance spectroscopy. {\em Int. J. Chem. Kinet. 18,}

\noindent {\bf Hu92.}
Huebner, W. F.; Keady, J. J.; Lyon, S. P. (1992). Solar photo rates for planetary atmospheres and atmospheric pollutants.  {\em Astrophys. Space Sci., 195,} 1-294.

\noindent {\bf In99}
Inomata, S.; Washida, N.  (1999). Rate Constants for the Reactions of NH2 and HNO with Atomic Oxygen at Temperatures Between 242 and 473 K. {\em  J. Phys. Chem. A 103,} 5023 - 5031.

\noindent {\bf Ja70.}
Jamieson, J.W.S.; Brown, G.R.; Tanner, J.S. (1970)
The reaction of atomic hydrogen with methyl cyanide.
{\em Can. J. Chem. 48.}

\noindent {\bf Ja03.}
Javoy, S. Naudet, V. Abid, S. Paillard, C.E. (2003).  Elementary reaction kinetics studies of interest in H2 supersonic combustion chemistry.  {\em Expt. Thermal Fluid Sci. 27,}  371-377.

\noindent {\bf Ja07.}
Jasper, A.W., Klippenstein, S.J., Harding, L.B.  (2007). Secondary kinetics of methanol decomposition: Theoretical rate coefficients for (CH2)-C3+OH, (CH2)-C3+(CH2)-C3, and (CH2)-C3+CH3. {\em J Phys Chem A 111,} 8699-8707.

\noindent {\bf Ja09.}
Jasper, A.W., Klippenstein, S.J., Harding, L.B.  (2009). Theoretical rate coefficients for the reaction of methyl radical with hydroperoxyl radical and for methylhydroperoxide decomposition.
{\em Proc. Combust. Inst. 32,} 279 - 286.

\noindent {\bf Je82.}
Jensen, D.E.; Jones, G.A. (1982). Kinetics of flame inhibition by sodium. {\em J. Chem. Soc. Faraday Trans. 1, 78,}

\noindent {\bf Jo99.}
Jodkowski, J.T., Rayez, M.-T., Rayez, J.-C. (1999). Theoretical Study of the Kinetics of the Hydrogen Abstraction from Methanol. 3. Reaction of Methanol with Hydrogen Atom, Methyl, and Hydroxyl Radicals. {\em J. Phys. Chem. A 103,} 3750-3765. 
 
\noindent {\bf Ka89.}
Kasting JF, Zahnle KJ, Pinto JP, and Young AT (1989). Sulfur, ultraviolet radiation, and the early evolution of life. {\em Origins of Life 19,} 95-108.

\noindent {\bf Ka05.}
Karach, S.P., Osherov, V.I.  (2005). Ab Initio Analysis of the Transition States on the Lowest Triplet H2O2 Potential Surface. {\em J. Chem. Phys. 110,} 11918-11927.

%\noindent KD74.
%Klemm RB and Davis DD (1974). A Flash Photolysis-Resonance Fluorescence Kinetics Study of the Reaction S($^3$P) + OCS. {\em  J.\ Phys.\ Chem.\ 78,} 1137-1146.

\noindent {\bf Ke72.}
Kerr, J.A.; Parsonage, M.J. (1972). {\em Evaluated Kinetic Data on Gas Phase Addition Reactions. Reactions of Atoms and Radicals with Alkenes, Alkynes and Aromatic Compounds.} Butterworths, London.

\noindent {\bf Ke13.}
Keller-Rudek, H., Moortgat, G. K., Sander, R., S{\"o}rensen, R. (2013).
 The MPI-Mainz UV/VIS spectral atlas of gaseous molecules of atmospheric interest.
 {\em Earth Syst. Sci. Data, 5,} 365-373.
 
\noindent {\bf Kl05.}
Klippenstein, S.J.; Miller, J.A. 2005  The addition of hydrogen atoms to diacetylene and the heats of formation of i-C4H3 and n-C4H3
{\em  J. Phys. Chem. A 109,}  4285 - 4295

\noindent {\bf Kl09.}
Klippenstein, S.J.; Harding, L.B.; Ruscic, B.; Sivaramakrishnan, R.; Srinivasan, N.K.; Su, M.C.; Michael, J.V. (2009).
Thermal Decomposition of NH2OH and Subsequent Reactions: Ab Initio Transition State Theory and Reflected Shock Tube Experiments
{\em J. Phys. Chem. A 113,} 10241 - 10259.

\noindent {\bf Kn88.}
Knipovich, O.M.; Rubtsova, E.A.; Nekrasov, L.I. (1988).   Volume recombination of nitrogen atoms in the afterglow of a condensed discharge. {\em Russ. J. Phys. Chem. (Engl. Transl.) 62,} 867-870.

\noindent {\bf Kny96.}
Knyazev, V.D., Bencsura, A., Stoliarov, S.I., Slagle, I.R. (1996).  Kinetics of the C2H3 + H2 = H + C2H4 and CH3 + H2 = H + CH4 reactions. {\em J. Phys. Chem. 100,} 11346-11354.

\noindent {\bf Kny17.}
Knyazev, V.D. (2017)  Kinetics and mechanism of the reaction of recombination of vinyl and hydroxyl radicals.
{\em Chem. Phys. Lett. 685,} 165 - 170.

\noindent {\bf Ko00.}
Koike, T., Kudo, M., Yamada, H. (2000). Rate Constants of CH4 + M = CH3 + H + M and CH3OH + M = CH3 + OH + M over 1400 - 2500 K.  {\em Int. J. Chem. Kin. 32,} 1-6.

%\noindent Kr87.
%Krasnopolsky VA (1987).  S3 and S4 absorption cross sections in the range of 340 to 600 nm and evaluation of the S3 abundance in the lower atmosphere of Venus.
%  {\em Adv. Space Phys.\ 7,} 25-27.

\noindent {\bf Kr97.}
Kruse, T.; Roth, P. (1997). Kinetics of C2 reactions during high-temperature pyrolysis of acetylene. {\em J. Phys. Chem. A 101,}  2138-2146.

\noindent {\bf Kr13.}
Krasnopolsky, V.A. (2013).
S3 and S4 abundances and improved chemical kinetic model for the lower atmosphere of Venus.
{\em Icarus 225,} 570-580.

\noindent {\bf Kr18.}
Kroupnov, A.A.; Pogosbekian, M.J. (2018)
DFT calculation-based study of the mechanism for CO2 formation in the interaction of CO and NO2 molecules
{\em Chem. Phys. Lett. 710,} 90 - 95.

\noindent {\bf Ku95.}
Kurbanov, M.A., Mamedov, Kh.F.  (1995). The role of the reaction CO + SH $\rightarrow$ COS + H in hydrogen formation in the course of interaction between CO and H2S. {\em Kinet. Catal.  36,} 455-457.

\noindent {\bf La72.}
Langford, R.B.; Oldershaw, G.A. (1972). Mechanism of Sulfur Formation in the Flash Photolysis of Carbonyl Sulphide. {\em J. Chem. Soc. Faraday Trans. 1, 69,}1550-1559.

\noindent {\bf La90.}
Lander, D.R., Unfried, K.G., Glass, G.P., Curl, R.F. (1990). Rate constant measurements of C2H with CH4, C2H6, C2H4, D2, and CO.  {\em J. Phys. Chem. 94,} 7759-7763. 

\noindent {\bf La92}
Lang, V.I. (1992)  Rate constants for reactions of hydrazine fuels with O(3P).  {\em J. Phys. Chem. 96,} 3047-3050.

\noindent {\bf Lai92}
Lai, L-H., Hsu, Y-C., Lee, Y-P. (1992). The enthalpy change and the detailed rate coefficients of the equilibrium reaction OH+C2H2 + M = HOC2H2 + M over the temperature range 627-713K.  {\em J. Chem. Phys. 97,} 3092 - 3099.

\noindent {\bf La04.}
Laufer, A.H., Fahr, A. (2004). Reactions and kinetics of unsaturated C2 hydrocarbon radicals. {\em Chem. Rev. 104,} 2813-2832.

\noindent {\bf Le77.}
Lee, J.H., Stief, L.J., Timmons, R.B. (1977). Absolute Rate Parameters for the Reaction of Atomic Hydrogen with Carbonyl Sulfide and Ethylene Episulfide,  {\em J. Chem. Phys. 67,} 1705-1714.

\noindent {\bf Li84.}
Lichtin, D.A. Berman, M.R., Lin, M.C. (1984). NH(A$^3\pi \rightarrow$ X$^3\Sigma^-$) Chemiluminescence from the CH(X$^3\pi$) + NO reaction.  {\em Chem. Phys. Lett. 108,} 18-24.

\noindent {\bf Li91.}
Lifshitz, A., Michael, J.V. (1991). Rate constants for the reaction, O + H2O $\rightarrow$? OH + OH, over the temperature range, 1500-2400 K, by the flash photolysis-shock tube technique: a further consideration of the back reaction. {\em Symp. Int. Combust. Proc. 23,}  59-67.

\noindent {\bf Li96.}
Li, S.C., Williams, F.A. (1996). Experimental and numerical studies of two-stage methanol flames.  {\em Symp. Int. Combust. Proc. 26,} 1017-1024.

\noindent {\bf Lin93}
Lin, M.C.; He, Y.; Melius, C.F. (1993). Theoretical aspects of product formation from the NCO + NO reaction.  {\em J. Phys. Chem. 97,}  9124 - 9128.

\noindent {\bf Li96b.}
Linder DP, Duan X, Page M (1996). Thermal rate constant for R+N2H2 $\rightarrow$ RH+NNH (R=H,OH,NH2) determined from multireference configuration interaction and variational transition state theory calculations. {\em J. Chem Phys. 104,} 6298-6306.

\noindent {\bf Li03.}
Liu, J.Y., Li, Z.S., Wu, J.Y., Wei, Z.G., Zhang, G., Sun, C.C. (2003). Theoretical study and rate constant calculation of the CH2O + CH3 reaction.  {\em J. Chem. Phys. 119,} 7214-7221.

\noindent {\bf Li04.}
Li, Q.S., Zhang, Y., Zhang, S.W. (2004). Direct ab initio dynamics study on the rate constants and kinetics isotope effects of CH3O + H $\rightarrow$ CH2O + H2 reaction. {\em J. Chem. Phys. 121,}  9474-9480

\noindent {\bf Li06.}
Li, Q.S.; Zhang, X. (2006).   Direct dynamics study on the hydrogen abstraction reactions N2H4 + R ? N2H3 + RH (R=NH2,CH3).  J. Chem. Phys. 125,

\noindent {\bf Li07.}
Li, J.,Zhao, Z.W., Kazakov, A., Chaos, M.,Dryer, F.L., Scire, J.J. (2007). A comprehensive kinetic mechanism for CO, CH2O, and CH3OH combustion. {\em  Int. J. Chem. Kinet. 39,}109-136.

\noindent{\bf Li14}
Li, S. J.; Davidson, D. F.; Hanson, R. K. (2014)
Shock tube study of the pressure dependence of monomethylhydrazine pyrolysis
{\em Combust. Flame 161,} 16 - 22.

\noindent {\bf Liu22.}
Liu, S.; Fan, W.; Wang, X.; Chen, J.; Guo, H. (2022). Improvement of kinetic parameters and modeling of the N2O chemical reaction in combustion.
{\em Energy 247}

\noindent {\bf Lo84.}
Louge, M.Y.; Hanson, R.K. (1984).  Shock tube study of NCO kinetics.
{\em  Symp. Int. Combust. Proc.} 1984

\noindent {\bf Lo04.}
Louis, F.; Gonzalez, C.A.; Sawerysyn, J.P. (2004).
Direct combined ab initio/transition state theory study of the kinetics of the abstraction reactions of halogenated methanes with hydrogen atoms
{\em  J. Phys. Chem. A 108,} 10586 - 10593.

\noindent {\bf Lo15.}
Loison, J.-C. (2015).

\noindent {\bf Lu03.}
Lu, C.W., Wu, Y.J., Lee, Y.P., Zhu, R.S., Lin, M.C.(2003). Experiments and calculations on rate coefficients for pyrolysis of SO2 and the reaction O plus SO at high temperatures. {\em J. Phys. Chem. A. 107,} 11020-11029.

\noindent {\bf Lu04.}
Lu, C.W., Wu, Y.J., Lee, Y.P., Zhu, R.S., Lin, M.C. (2004). Experimental and theoretical investigations of rate coefficients of the reaction S($^3$P) + O2 in the temperature range 298-878 K.  {\em J Chem Phys. 121,} 8271-8278

\noindent {\bf Lu06.}
Lu, C.W.,Wu, Y.J.,Lee, Y.P.,Zhu, R.S.,Lin, M.C. (2006).  Experimental and theoretical investigation of rate coefficients of the reaction S($^3$P) + OCS in the temperature range of 298-985 K.  {\em J. Chem. Phys. 125,} 164329.

\noindent {\bf Ma66.}
Mayer, S.W.; Schieler, L.(1966). Computed high-temperature rate constants for hydrogen-atom transfers involving light atoms.
{\em  J. Chem. Phys. 45,}

\noindent {\bf Ma67.}
Mayer, S.W.; Schieler, L.; Johnston, H.S. (1967). Computation of high-temperature rate constants for bimolecular reactions of combustion products. {\em Symp. Int. Combust. Proc. 11,} 837 - 844.

\noindent {\bf Ma83.}
 Martinez RI and Herron JT (1983).  Methyl thiirane: Kinetic gas-phase titration of sulfur atoms in S$_x$O$_y$ systems.  {\em Int.\ J.\ Chem.\ Kinet.\ 15,} 1127-1140.

\noindent {\bf Ma89.}
Marston, G., Nesbitt, F.L., Stief, L.J. (1989). Branching ratios in the N + CH3 reaction: Formation of the methylene amidogen (H2CN) radical.  {\em J. Chem. Phys. 91,} 3483.

\noindent {\bf Ma98.}
Marinov, N.M.; Pitz, W.J.; Westbrook, C.K.; Vincitore, A.M.; Castaldi, M.J.; Senkan, S.M. (1998).
Aromatic and Polycyclic Aromatic Hydrocarbon Formation in a Laminar Premixed n-butane Flame.  {\em Combust. Flame 114,} 192 - 213.

\noindent {\bf Mar98.}
Marchand, N.; Rayez, J.C.; Smith, S.C.
Theoretical study of the reaction CH(X$^2\Pi$) + NO(X$^2\Pi$). 3. Determination of the branching ratios
{\em J. Phys. Chem. A 102,} 3358 - 3367.

\noindent {\bf Ma06.}
Matus MH, Arduengo AJ 3rd, Dixon DA. (2006).  The heats of formation of diazene, hydrazine, N2H3+, N2H5+, N2H, and N2H3 and the Methyl Derivatives CH3NNH, CH3NNCH3, and CH3HNNHCH3. {\em J Phys Chem A. 110,} 10116-10121.

\noindent {\bf Ma11.}
Matsugi, A.; Suma, K.; Miyoshi, A. (2011)
Deuterium kinetic isotope effects on the gas-phase reactions of C2H with H-2(D-2) and CH4(CD4)
{\em Phys. Chem. Chem. Phys. 13,}  4022 - 4031

\noindent {\bf Ma18.}
Mazarei, E.; Mousavipour, S.H. (2018).  Theoretical Study on the Dynamics and Kinetics of the Reaction of CH2OH with OH
{\em  J. Phys. Chem. A 122,}  9761 - 9777.

\noindent {\bf Me81.}
Messing, I., Filseth, S.V., Sadowski, C.M.; Carrington, T. (1981). Absolute Rate Constants for the Reactions of CH with O and N Atoms.  {\em J. Chem. Phys. 74,} 3874.

\noindent {\bf Me91.}
Mertens, J.D., Kohse-Hoinghaus, K., Hanson, R.K., Bowman, C.T. (1991).  A shock tube study of H + HNCO $\rightarrow$ NH$_2$ + CO.  {\em Int. J. Chem. Kinet. 23,} 655 - 668.

\noindent {\bf Me93.}
Meads, R.F., Maclagan, R.G.A.R., Phillips, L.F. (1993) Kinetics, energetics, and dynamics of the reactions of CN with NH3 and ND3.   {\em J. Phys. Chem.  97,} 3257-3265.

\noindent {\bf Me96}
Mebel, A.M.; Lin, M.C.; Morokuma, K.; Melius, C.F. (1996).  Theoretical study of reactions of N$_2$O with NO and OH radicals.  {\em Int. J. Chem. Kinet.  28,}  693 - 703.

\noindent {\bf Meb96}
Mebel, A.M.; Diau, E.W.G.; Lin, M.C.; Morokuma, K. (1996). Theoretical rate constants for the NH$_3$ + NO$_x$ $\rightarrow$? NH$_2$ + HNO$_x$ (x = 1, 2) reactions by ab initio MO/VTST calculations {\em J. Phys. Chem. 100,} 7517 - 7525.

\noindent {\bf Mer96.} Mertens, J.D., Hanson, R.K. (1996). A shock tube study of H + HNCO $\rightarrow$ H2 + NCO and the thermal decomposition of NCO.  {\em Symp. Int. Combust. Proc. 26,} 551 - 558.

\noindent {\bf Me00}
Meagher, N.E.; Anderson, W.R. (2000).  Kinetics of the O($^3$P) + N$_2$O Reaction. 2. Interpretation and Recommended Rate Coefficients. {\em J. Phys. Chem. A  104,} 6013 - 6031.

\noindent {\bf Mi88.}
Miller, J.A., Melius, C.F. (1988). A theoretical analysis of the reaction between hydroxyl and acetylene. {\em Symp. Int. Combust. Proc. 22,} 1031-1039.

\noindent {\bf Mi92.}
Miller JA, Melius CF (1992). A theoretical study of the reaction between hydrogen atoms and isocyanic acid.  {\em Int. J. Chem. Kin. 24,} 421-432.

\noindent {\bf Mi97.}
Millar TJ, Farquhar PRA, and Willacy K (1997). The UMIST database for astrochemistry 1995. {\em Astron. Astrophys. Suppl. Ser. 121,} 139-185.

\noindent {\bf Mi98.}
Mills (1998). 

\noindent {\bf Mi01.}
Michelsen, H.A.; Simpson, W.R. (2001). Relating State-Dependent Cross Sections to Non-Arrhenius Behavior for the Cl + CH4 Reaction. {\em J. Phys. Chem. A 105,} 1476 - 1488

\noindent {\bf Mi03.}
Miller, J.A., Klippenstein, S.J. , Glarborg, P. (2003).   A kinetic issue in reburning: The fate of HCNO.  {\em Combust. Flame 135,} 357 - 362.

\noindent {\bf Mi05.}
Michael, J.V., Su, M.C., Sutherland, J.W., Harding, L.B., Wagner, A.F.  (2005). Rate constants for D+C2H4 $\rightarrow$ C2H3D+H at high temperature: implications to the high pressure rate constant for H+C2H4 $\rightarrow$C2H5.  {\em Proc. Combust. Inst. 30,} 965-973.

\noindent {\bf Mo73.}
Morris, E.D., Jr.; Niki, H. (1973).
Reaction of Methyl Radicals with Atomic Oxygen
{\em Int. J. Chem. Kinet. 5}

\noindent {\bf Mo77.}
Moortgat, G.K., Slemr, F., Warneck, P. (1977). Kinetics and Mechanism of the Reaction H + CH3ONO. {\em  Int. J. Chem. Kinet.  9,} 267.

\noindent {\bf Mo81.}
Molina, L.T., Lamb, J.J., Molina, M.J.  (1981). Temperature dependent UV absorption cross sections for carbonyl sulfide. {\em Geophys.\ Res.\ Lett.\ 8,} 1008-1011.

\noindent {\bf Mo95a.}
Moses, J.I. Allen, M., Gladstone, G.R. (1995).	Post-SL9 sulfur photochemistry on Jupiter. {\em Geophys.\ Res.\ Lett.\ 22,} 1597-1600.

\noindent {\bf Mo95b.}
Moses, J.I. Allen, M., Gladstone, G.R. (1995).	Nitrogen and oxygen photochemistry following SL9. {\em Geophys.\ Res.\ Lett.\ 22,} 1601-1604.

\noindent {\bf Mo96.}  
Moses, J.I. 

\noindent {\bf Mo02.}
Moses JI, Zolotov MY, and Fegley BJ (2002). Photochemistry of a Volcanically Driven Atmosphere on Io: Sulfur and Oxygen Species from a Pele-Type Eruption. {\em Icarus 156,} 76-106.

\noindent {\bf Mu87.}
Mulenko, S.A. (1987). The application of an intracavity laser spectroscopy method for elementary processes study in gas-phase reactions. {\em Rev. Roum. Phys. 32,} 173.

\noindent{\bf Na05}
Naidoo, J.; Goumri, A.; Marshall, P. (2005).
 A kinetic study of the reaction of atomic oxygen with SO2
{\em Proc. Combust. Inst. 30,} 1219 - 1225.

\noindent {\bf Ni79.}
Nicholas, J.E., Amodio, C.A., Baker, M.J. (1979).  Kinetics and Mechanism of the Decomposition of H2S, CH3SH and (CH3)$_2$S in a Radio-frequency Pulse Discharge. {\em J. Chem. Soc. Faraday Trans. 1, 75,} 1868-1880.

\noindent {\bf Ni03.}
Nizamov, B., Dagdigian, P.J. (2003).  Spectroscopic and Kinetic Investigation of Methylene Amidogen by Cavity Ring-Down Spectroscopy. {\em J. Phys. Chem. A 107,} 2256-2263.

\noindent {\bf Ne90.}
Nesbitt, F.L.; Marston, G.; Stief, L.J. (1990). Kinetic studies of the reactions of H2CN and D2CN radicals with N and H.  {\em J. Phys. Chem. 94,} 4946.

\noindent {\bf Ng96}
Nguyen, M.T.; Sengupta, D.; Vereecken, L.; Peeters, J.; Vanquickenborne, L.G. (1996).
 Reaction of isocyanic acid and hydrogen atom (H + HNCO): theoretical characterization.  
 {\em J. Phys. Chem. 100,} 1615 - 1621.

\noindent {\bf Ng04}
Nguyen, H.MT.; Zhang, S.W.; Peeters, J.; Truong, T.N.; Nguyen, M.T. (2004).  Direct ab initio dynamics studies of the reactions of HNO with H and OH radicals. 
{\em Chem. Phys. Lett.   388,} 94 - 99.

\noindent {\bf No89.}
Norton, T.S., Dryer, F.L. (1989). Some new observations on methanol oxidation chemistry.  {\em Combust. Sci. Technol.  63,} 107-129.

\noindent {\bf Nu19.}
Nunez-Reyes, D.; Loison, J.-C.; Hickson, K.M.; Dobrijevic, M. (2019).
Rate constants for the N(D-2) + C2H2 reaction over the 50-296 K temperature range
{\em Phys. Chem. Chem. Phys. 21,} 22230 - 22237.

\noindent {\bf Oe92.}
Oehlers, C., Temps, F., Wagner, H.Gg., Wolf, M. (1992). Kinetics of the reaction of OH radicals with CH2CO. {\em Ber. Bunsenges. Phys. Chem. 96,} 171 - 175.

\noindent {\bf Ohm90}
Ohmori, K.; Miyoshi, A.; Matsui, H.; Washida, N.
Studies on the reaction of acetaldehyde and acetyl radicals with atomic hydrogen
{\em J. Phys. Chem. 94,} 3253 - 3255

\noindent {\bf Ok78.}
Okabe H (1978) {\em The Photochemistry of Small Molecules.} Wiley-Interscience, New York, 431 pp.

\noindent {\bf Olm16.}
Olm, C.; Varga, T.; Valko, E.; Hartl, S.; Hasse, C.; Turanyi, T. (2016).
Development of an Ethanol Combustion Mechanism Based on a Hierarchical Optimization Approach
{\em Int. J. Chem. Kinet. 48,} 423 - 441.


\noindent {\bf Oy94.}
Oya, M., Shiina, H., Tsuchiya, K., Matsui, H. (1994). Thermal decomposition of COS. {\em Bull. Chem. Soc. Japan 67,} 2311-2313.

\noindent {\bf Pa79.}
Pagsberg, P.B. Eriksen, J. Christensen, H.C. (1979).  Pulse Radiolysis of Gaseous Ammonia-Oxygen Mixtures.  {\em J. Phys. Chem. 83,} 582.

\noindent {\bf Pa93}
Park, J.; Hershberger, J.F. (1993).  Kinetics and product branching ratios of the CN+NO2 reaction.
{\em J. Chem. Phys.  99,}  3488 - 3493

\noindent {\bf Pa96.}
Payne, W.A., Monks, P.S., Nesbitt, F.L., Stief, L.J. (1996). The reaction between N($^4$S) and C2H3: rate constant and primary reaction channels.  {\em J. Chem. Phys. 104,} 9808-9815.

\noindent {\bf Pa08.}
Paramo, A., Canosa, A., Le Picard, S.D., Sims, I.R. (2008). Rate coefficients for the reactions of C2 with various hydrocarbons (CH4, C2H2, C2H4, C2H6, and C3H8): A gas-phase experimental study over the temperature range 24-300 K.  {\em J. Phys. Chem. A 112, } 9591-9600.


\noindent {\bf Pe99.}
Pen, J., Hu, X., Marshall, P. (1999).  Experimental and ab Initio Investigations of the Kinetics of the Reaction of H Atoms with H2S.  {\em J. Phys. Chem. A, 103,} 5307-5311.

\noindent {\bf Pe85.}
Perry, R.A., Melius, C.F. (1985).  The rate and mechanism of the reaction of HCN with oxygen atoms over the temperature range 540-900 K.  {\em Symp. Int. Combust. Proc. 20,} 20639.

\noindent {\bf Pe88.}
Perrin, D., Richard, C., Martin, R. (1988). Etude cinetique de la reaction thermique du pentene-2 cis vers 500$^{\circ}$C. III - Influence de H2S. {\em  J. Chim. Phys. 85,} 185-192.

\noindent {\bf Pe95.}
Peeters, J., Boullart, W., Devriendt, K. (1995).  CH formation in the reaction between ketenyl radicals and oxygen atoms. Determination of the CH yield between 405 and 960 K.  {\em J. Phys. Chem. 99,} 3583 - 3591.

\noindent {\bf Po91.}
Pople JA, Curtiss LA (1991). The energy of N2H2 and related compounds.  {\em J. Chem. Phys. 95,} 4385-4388.

\noindent {\bf Po13.}
Polino, D.; Klippenstein, S. J.; Harding, L. B.; Georgievskii, Y. (2013)
Predictive Theory for the Addition and Insertion Kinetics of (CH2)-C1 Reacting with Unsaturated Hydrocarbons
{\em J. Phys. Chem. A  117,} 12677 - 12692.

\noindent {\bf Pr77.}
Pratt, G.; Rogers, D. (1977). Homogeneous Isotope Exchange Reactions. Part 3. H2S + D2.  {\em J. Chem. Soc. Faraday Trans. 1, 73,} 54.

\noindent {\bf Ra03.}
Rauk, A., Boyd, R.J., Boyd, S.L., Henry, D.J., Radom, L. (2003).  Alkoxy radicals in the gaseous phase: beta-scission reactions and formation by radical addition to carbonyl compounds.  {\em Can. J. Chem. 81,} 431 - 442.

\noindent {\bf Ra11}
 Rao, H.B.; Zeng, X.Y.; He, H.; Li, Z.R. (2011).
Theoretical Investigations on Removal Reactions of Ethenol by H Atom
{\em J. Phys. Chem. A 115,} 1602-1608.

\noindent {\bf Re94}
Reiner, T.; Arnold, F. (1994).  Laboratory investigations of gaseous sulfuric acid formation via SO3+H2O+M??H2SO4+M: measurement of the rate constant and product identification. {\em J. Chem. Phys. 101,} 7399 - 7407.

\noindent {\bf Ri99.}
Rim, K.T., Hershberger, J.F. (1999).  Temperature dependence of the product branching ratio of the CN + O2 reaction. {\em J. Phys. Chem. A. 103,}  372-3725.

\noindent {\bf Ro94.}
Rohrig, M., Wagner, H.G. (1994). The reactions of NH(X$^3\Sigma^-$) with the water gas components CO2, H2O, and H2.  {\em Symp. Int. Combust. Proc.  25,} 975-981.

\noindent {\bf Rod96.}
Rodgers, A.S., Smith, G.P. (1996). Pressure and temperature dependence of the reactions of CH with N2. {\em Chem. Phys. Lett. 253,} 313-321.

\noindent {\bf Rom96.}
Romming, H.J., Wagner, H.Gg. (1996). A kinetic study of the reactions of NH(X$^3\Sigma^-$) with O2 and NO in the temperature range from 1200 to 2200 K.  {\em Symp. Int. Combust. Proc. 26,} 559-566.

\noindent {\bf Sa80.}
Saito, K., Toriyama, Y., Yokubo, T., Higashihara, T., Murakami, I. (1980). A Measurement of the Thermal Decomposition of CS2 behind Reflected Shock Waves. {\em Bull. Chem. Soc. Japan 53,} 1437-1438.

\noindent {\bf Sa99.}
Sato, K.; Misawa, K.; Kobayashi, Y.; Matsui, M.; Tsunashima, S.; Kurosaki, Y.; Takayanagi, T. (1999).
Measurements of Thermal Rate Constants for the Reactions of N(2D,2P) with C2H4 and C2D4 Between 225 and 292 K
{\em  J. Phys. Chem. A 103,}  8650 - 8656.

\noindent {\bf Sa03.}
Sander SP, Friedl RR,  Ravishankara AR, Golden DM, Kolb CE, Kurylo MJ, Huie RE, Orkin VL, Molina MJ, Moortgat GK, and Finlayson-Pitts BJ (2003).
{\em Chemical Kinetics and Photochemical Data for Use in Atmospheric Studies. Evaluation Number 14.} JPL Publication 02-25.

\noindent {\bf Sa19.}
Savchenkova, A.S.; Semenikhin, A.S.; Chechet, I.V.; Matveev, S.G.; Konnov, A.A.; Mebel, A.M. (2019).
Mechanism and rate constants of the CH2 + CH2CO reactions in triplet and singlet states: A theoretical study.
{\em J. Comput. Chem. 40,} 387 - 399.

\noindent {\bf Sc73.}
Schofield, K.  (1973).  Evaluated chemical kinetic rate constants for various gas phase reactions.  {\em J. Phys. Chem. Ref. Data 2,} 25-84.

\noindent {\bf Se02.} 
Sendt, K.; Jazbec, M.; Haynes, B.S. (2002).
Chemical kinetic modeling of the H/S system: H2S thermolysis and H-2 sulfidation
{\em Proc. Combust. Inst. 29,} 2439 - 2446.

\noindent {\bf Se06a.}  Senosiain, J.P., Klippenstein, S.J., Miller, J.A. (2006). Pathways and rate coefficients for the decomposition of vinoxy and acetyl radicals.  {\em J. Phys. Chem. A 110,} 5772 - 5781.

\noindent {\bf Se06b.}
Senosiain JP, Klippenstein SJ, Miller JA (2006).  Reaction of ethylene hydroxyl radicals.  {\em J. Phys. Chem. A 110,} 6960-6970.

\noindent {\bf Sem18.}
Semenikhin, A.S.; Shubina, E.G.; Savchenkova, A.S.; Chechet, I.V.; Matveev, S.G.; Konnov, A.A.; Mebel, A.M. (2018)
Mechanism and Rate Constants of the CH3+ CH2CO Reaction: A Theoretical Study
{\em Int. J. Chem. Kinet. 50,}  273 - 284.

\noindent {\bf Sh85.}
Shum, L.G.S.; Benson, S.W. (1985). The pyrolysis of dimethyl sulfide, kinetics and mechanism.  {\em Int. J. Chem. Kinet. 17,} 749.

\noindent {\bf Sh96.}
Shiina, H., Oya, M., Yamashita, K., Miyoshi, A., Matsui, H. (1996). Kinetic studies on the pyrolysis of H2S. {\em  Phys. Chem. 100,} 2136-2140.

\noindent {\bf Sh98.}
Shiina, H., Miyoshi, A., Matsui, H. (1998). Investigation on the insertion channel in the S(3P) + H2 reaction.  {\em J. Phys. Chem. A 102,} 3556 - 3559.

\noindent {\bf Si84.}
Silver, J.A.; Stanton, A.C.; Zahniser, M.S.; Kolb, C.E. (1984). Gas-phase reaction rate of sodium hydroxide with hydrochloric acid.  {\em J. Phys. Chem.  88,}

\noindent {\bf Si88.}
Singleton, D.L., Cvetanovic, R.J.  (1988). Evaluated chemical kinetic data for the reactions of atomic oxygen O(3P) with sulfur containing compounds.  {\em J. Phys. Chem. Ref. Data 17,} 1377-1399.

\noindent {\bf Si10.}
Sivaramakrishnan, R.; Su, M.C.; Michael, J.V.; Klippenstein, S.J.; Harding, L.B.; Ruscic, B. (2010)

\noindent {\bf Sk18.} 
Skouteris, D.; Balucani, N.; Ceccarelli, C.; Vazart, F.; Puzzarini, C.; Barone, V.; Codella, C.; Lefloch, B. (2018).
The Genealogical Tree of Ethanol: Gas-phase Formation of Glycolaldehyde, Acetic Acid, and Formic Acid
{\em Astrophys. J.  854,}  2018.

\noindent {\bf So01.}
Song, S., Hanson, R.K., Bowman, C.T., Golden, D.M.  (2001). Shock Tube Determination of the Overall Rate of NH2 + NO -> Products in the Thermal De-NOx Temperature Window.  {\em Int J. Chem. Kinet. 33,} 715-721.

\noindent {\bf So03.}
Song, S., Golden, D.M., Hanson, R.K., Bowman, C.T., Senosiain, J.P., Musgrave, C.B., Friedrichs, G. (2003). A Shock Tube Study of the Reaction NH2 + CH4 $\rightarrow$ NH3 + CH3 and Comparison With Transition State Theory.  {\em Int. J. Chem. Kinet. 35,} 304-309.

\noindent {\bf So15.}
Sommerer, J.; Olzmann, M. (2015)
 Kinetics of the Reactions of Hydroxyl Radicals with Diacetylene and Vinylacetylene
{\em Z. Phys. Chem. 229,}495 - 505.

\noindent {\bf Sp00.}
Spencer JR, Jessup, KL, McGrath MA, Ballester GE, Yelle RV (2000). Discovery of Gaseous S$_2$ in Io's Pele Plume. {\em Science 288,} 1208-1210.

\noindent {\bf Sri05}
Srinivasan, N.K.; Su, M.C.; Sutherland, J.W.; Michael, J.V. (2005).  Reflected shock tube studies of high-temperature rate constants for OH+CH4 -> CH3+H2O and CH3+NO2 $\rightarrow$ CH3O+NO.  {\em J. Phys. Chem. A  109,}  1857 - 1863.

\noindent {\bf St68.}
Strachan, A. N., Thornton, D. E. (1968). Photolysis of ketene.  {\em  Can.\ J.\ Chem.\  46,} 2353-2360.

\noindent {\bf St87.}
Stachnik R., Molina M. J.  (1987). Kinetics of the reactions of SH radicals with NO$_2$ and O$_2$. {\em  J.\ Phys.\ Chem.\  91,} 4603-4611.

\noindent {\bf St88.}
Stief, L.J., Marston, G., Nava, D.F., Payne, W.A., Nesbitt, F.L. (1988). Rate constant for the reaction of N($^4$S) with CH3 at 298 K.  {\em Chem. Phys. Lett. 147,}

\noindent {\bf St95.}
Stief, L.J.; Nesbitt, F.L.; Payne, W.A.; Kuo, S.C.; Tao, W.; Klemm, R.B. (1995). Rate constant and reaction channels for the reaction of atomic nitrogen with the ethyl radical. {\em  J. Chem. Phys. 102,} 5309-5316.

\noindent {\bf St95.}
Stothard N, Humpfer R, Grotheer H-H (1995). The multichannel reaction NH2+NH2 at ambient temperature and low pressures. {\em Chem. Phys. Lett. 240,} 474-480.

\noindent {\bf Sto00.}
Stoliarov, S.I.; Knyazev, V.D.; Slagle, I.R. (2000)
Experimental Study of the Reaction Between Vinyl and Methyl Radicals in the Gas Phase. Temperature and Pressure Dependence of Overall Rate Constants and Product Yields
{\em J. Phys. Chem. A 104,} 9687 - 9697.

\noindent {\bf Sut02.}
Sutherland, J.W., Su, M.-C., Michael, J.V. (2002). Rate Constants for H + CH4, CH3 + H2, and CH4 Dissociation at High Temperature. {\em Int. J. Chem. Kinet. 33,} 669-684.

\noindent {\bf Su02.}
Su, M.-C.; Kumaran, S.S.; Lim, K.P.; Michael, J.V.; Wagner, A.F.; Harding, L.B.; Fang, D.-C. (2002).   
Rate Constants, 1100<T<2000K, for H + NO2 -> OH + NO Using Two Shock Tube Techniques: Comparison of Theory to Experiment.  
{\em J. Phys. Chem. A   106,} 8261 - 8270.

\noindent {\bf Su10.}
 Sun, J.Y.; Tang, Y.Z.; Jia, X.J.; Wang, F.; Sun, H.; Feng, J.D.; Pan, X.M.; Hao, L.Z.; Wang, R.S. (2010).
Theoretical study for the reaction of CH3CN with O(P-3)
{\em J. Chem. Phys. 132}

\noindent {\bf Sy01.}
Syrstad, E.A.; Turecek, F. (2001).
Hydrogen Atom Adducts to the Amide Bond. Generation and Energetics of the Amino(hydroxy)methyl Radical in the Gas Phase
{\em J. Phys. Chem. A 105,} 11144 - 11155.

\noindent {\bf Sz84.}
Szekely, A.; Hanson, R.K.; Bowman, C.T. (1984)
Shock tube study of the reaction between hydrogen cyanide and atomic oxygen
{\em Symp. Int. Combust. Proc.  20,} 1984

\noindent {\bf Te90.}
Tesner, P.A., Nemirovskii, M.S., Motyl, D.N. (1990). Kinetics of the thermal decomposition of hydrogen sulfide at 600-1200$^{\circ}$C.  {\em Kinet. Catal. 31,} 1081-1083.

\noindent {\bf Th86.}
Thielen, K., Roth, P. (1986).  N atom measurements in high-temperature N2 dissociation kinetics. {\em AIAA J.  24,} 1102-1105

\noindent {\bf To84.}
Toby, S.; Sheth, S.; Toby, F.S. (1984).  Reaction of carbon monoxide with ozone with oxygen atoms. {\em Int. J. Chem. Kinet. 16,} 149.

\noindent {\bf To03.}
Tomeczek, J., Gradon, B. (2003). The role of N2O and NNH in the formation of NO via HCN in hydrocarbon flames. {\em  Combust. Flame 133,} 311-322.

\noindent {\bf Tr05.}
Troe J (2005). Theory of multichannel thermal unimolecular reactions. 2. Application to the thermal dissociation of formaldehyde. {\em J. Phys. Chem. A 109,} 8320-8328.

\noindent {\bf Ts81.}
Tsuboi, T., Katoh, M., Kikuchi, S., Hashimoto, K. (1981). Thermal Decomposition of Methanol behind Shock Waves.  {\em Japan J. Appl. Phys. 20,} 985.

\noindent {\bf Ts86.}
Tsang, W., Hampson, R.F. (1986). Chemical kinetic data base for combustion chemistry. Part I. Methane and related compounds.  {\em J. Phys. Chem. Ref. Data 15,} 1087-1280.

\noindent {\bf Ts87.}
Tsang, W. (1987).  Chemical kinetic data base for combustion chemistry. Part 2. Methanol.  {\em J. Phys. Chem. Ref. Data 16,} 471-509.

\noindent {\bf Ts90.}
Tsai C.P. McFadden, D.L. (1990). Gas-phase atom-radical kinetics of atomic hydrogen, nitrogen, and oxygen reactions with fluoromethylene radicals.  {\em J. Phys. Chem.94,} 3298.

\noindent {\bf Ts91.}
Tsang W., Herron J. (1991).
Chemical Kinetic Data Base for Propellant Combustion I. Reactions Involving NO, NO2, HNO, HNO2, HCN and N2.
{\em J. Phys. Chem. Ref. Data 20,} 609-664.

\noindent {\bf Ts97.}
Tsuchiya, K.; Kamiya, K.; Matsui, H.(1997). 
Studies on the oxidation mechanism of H2S based on direct examination of the key reactions.
{\it Int. J. Chem. Kinet. 29,} 57 - 66.

\noindent {\bf Va77.}
Vandooren, J., Van Tiggelen, P.J. (1977). Reaction Mechanisms of Combustion in Low Pressure Acetylene-Oxygen Flames. {\em Symp. Int. Combust. Proc. 16,} 165.

\noindent {\bf Va95.}
Vaghjiani, G.L. (1995).  Laser photolysis studies of hydrazine vapor: 193 and 222-nm H-atom primary quantum yields at 296 K, and the kinetics of H + N2H4.   

\noindent {\bf Va01a.}
Vaghjiani, G.L. (2001).  Gas Phase Reaction Kinetics of O Atoms with (CH3)2NNH2, CH3NHNH2, and N2H4, and Branching Ratios of the OH Product. 
{\em J. Phys. Chem. A 105,} 4682-4690.

\noindent {\bf Va01b.}
Vaghjiani, G.L. (2001).  Kinetics of OH Reactions with N2H4, CH3NHNH2 and (CH3)2NNH2 in the Gas Phase. {\em Int J. Chem. Kinet. 33,} 354-362.

\noindent {\bf Va16.}
Vanuzzo, G.; Balucani, N.; Leonori, F.; Stranges, D.; Nevrly, V.; Falcinelli, S.; Bergeat, A.; Casavecchia, P. (2016).
Reaction Dynamics of O(P-3) + Propyne: I. Primary Products, Branching Ratios, and Role of Intersystem Crossing from Crossed Molecular Beam Experiments
{\em J. Phys. Chem. A 120,} 4603 - 4618.

\noindent {\bf vo1971.}
von Gehring, M.; Hoyermann, K.; Wagner, H.Gg.; Wolfrum, J. (1971).  Die Reaktion von Atomarem Wasserstoff mit Hydrazin. 
{\em Ber. Bunsenges. Phys. Chem. 75.}

%\noindent Va01.
%Van der Heijden and Van der Mullen (2001). {\em J.\ Phys.\ B. Atom.\ Mol.\ Opt.\ Phys.\ 34,} 4183-4201.

\noindent {\bf Wa75.}
Walkauskas, Kaufman (1975). 

\noindent {\bf Wa84.}
Warnatz, J. (1984).  Rate coefficients in the C/H/O system.  In {\em Combustion Chemistry.} ed. W.C. Gardiner,Jr.  Springer-Verlag NY p.\ 197.

%\noindent {\bf Wa03a.}
%Wang, B.S., Hou, H., Yoder, L.M., Muckerman, J.T., Fockenberg, C. (2003). Experimental and theoretical investigations on the methyl-methyl recombination reaction. {\em J. Phys. Chem. A 107,} 11414-11426.

\noindent {\bf Wa89.}
Wategaonkar, S.J.; Setser, D.W. (1989)
 Infrared chemiluminescence studies of H atom reactions with Cl2O, CINO, F2O, CF3OF, ClO2, NO2, and ClO
{\em J. Chem. Phys. 90,}  251 - 264.

\noindent {\bf Wa03.}
Wang, L.; Liu, J.-y.; Li, Z.-s.; Huang, X.-r.; Sun, C.-c. (2003).  Theoretical Study and Rate Constant Calculation of the Cl + HOCl and H + HOCl Reactions. {\em J. Phys. Chem. A 107,} 4921 - 4928.  % products

\noindent {\bf Wh83.}
Whyte, A.R., Phillips, L.F. (1983).  Rate of reaction of N with CN($\mu$=0,1). {\em Chem. Phys. Lett.  98,} 590.

\noindent {\bf Wh84.}
Whyte, A.R., Phillips, L.F. (1984).  Products of reaction of nitrogen atoms with NH2. {\em J. Phys. Chem.  88,} 5670.

\noindent{\bf Wo94.}
Woods, I.T., Haynes, B.S. (1994). C1/C2 chemistry in fuel-rich post-flame gases: detailed kinetic modelling. {\em Symp. Int. Combust. Proc. 25,} 909 - 917.

\noindent {\bf Woo95.}
Wooldridge, S.T., Hanson, R.K., Bowman, C.T. (1995). Simultaneous laser absorption measurements of CN and OH in a shock tube study of HCN + OH $\rightarrow$ products.  {\em Int. J. Chem. Kinet. 27,} 1075-1087.

\noindent {\bf Wo96a.}
Wooldridge, M.S., Hanson, R.K., Bowman, C.T. (1996). A shock tube study of CO + OH $\rightarrow$ CO2 + H and HNCO + OH $\rightarrow$ products via simultaneous laser absorption measurements of OH and CO2.  {\em Int. J. Chem. Kinet. 28,} 361-372.

\noindent {\bf Wo96b.}
Wooldridge, S.T., Hanson, R.K., Bowman, C.T. (1996). A shock tube study of reactions of CN with HCN, OH, and H2 using CN and OH laser absorption.   {\em Int. J. Chem. Kinet. 28,} 24 -258.

\noindent {\bf Woi94.}
Woiki, D.; Roth, P. (1994). Kinetics of the high-temperature H2S decomposition.  {\em J. Phys. Chem. 98,} 12958 - 12963

\noindent {\bf Woi95a.}
Woiki, D.; Roth, P. (1995). A shock tube study of the reaction S + H2 = SH + H in pyrolysis and photolysis systems. {\em Int. J. Chem. Kinet. 27,}  547-553.

\noindent {\bf Woi95b.}
Woiki, D., Roth, P. (1995). Oxidation of S and SO by O2 in high-temperature pyrolysis and photolysis reaction systems. {\em Int. J. Chem. Kinet. 27,} 5 -71.

\noindent{\bf Wu07}
Wu, C.W.; Lee, Y.P.; Xu, S.C.; Lin, M.C. (2007).
Experimental and theoretical studies of rate coefficients for the reaction O(3P) plus C2H5OH at high temperatures
{\em  J. Phys. Chem. A 111,} 6693 - 6703.

\noindent {\bf Xu99.}
Xu, Z.-F.; Li, S.-M.; Yu, Y.-X.; Li, Z.-S.; Sun, C.-C. (1999).  Theoretical Studies on the Reaction Path Dynamics and Variational Transition-State Theory Rate Constants of the Hydrogen-Abstraction Reactions of the NH(X$^3\Sigma^-$) Radical with Methane and Ethane. {\em J. Phys. Chem. A 103,} 4910-4917.

\noindent {\bf Xu10.}
Xu, S.C.; Lin, M.C. (2010)
Ab Initio Chemical Kinetics for Singlet CH2 Reaction with N2 and the Related Decomposition of Diazomethane
{\em  J. Phys. Chem. A  114,} 5195 - 5204.

\noindent {\bf Xu11.}
Xu, Z.F.; Xu, K.; Lin, M.C. (2011).
Thermal Decomposition of Ethanol. 4. Ab Initio Chemical Kinetics for Reactions of H Atoms with CH3CH2O and CH3CHOH Radicals.
{\em J. Phys. Chem. A 115,} 3509 - 3522.

\noindent {\bf Xu15.}
Xu, Z.F.; Raghunath, P.; Lin, M.C. (2015).
Ab Initio Chemical Kinetics for the CH3 + O(3P) Reaction and Related Isomerization-Decomposition of CH3O and CH2OH Radicals.
{\em J. Phys. Chem. A 119,} 7404-7417. 

\noindent {\bf Ya95.}
Yaws CL (1995).  {\em Handbook of Vapor Pressure Volume 4.}
Gulf Publishing Company.

\noindent {\bf Ya97.}
Yamada, T. Bozzelli, J.W. Lay, T. (1999). Kinetic and Thermodynamic Analysis on OH Addition to Ethylene: Adduct Formation, Isomerization, and Isomer Dissociations.
{\em J. Phys. Chem. A  103,} 7646-7655.

\noindent {\bf Ya05.}
Yang, Y.; Zhang, W.J.; Pei, S.X.; Shao, H.; Huang, W.; Gao, X.M. (2005). Theoretical study on the mechanism of the N($^4$S)+C2H5 reaction.  {\em J. Mol. Struct. (Theochem) 725,} 133-138.

\noindent {\bf Ya08.}
Yasunaga, K., Kubo, S., Hoshikawa, H.,Kamesawa, T., Hidaka, Y. (2008). Shock-tube and modeling study of acetaldehyde pyrolysis and oxidation
{\em Int. J. Chem. Kinet. 40,} 73 - 102.

\noindent {\bf Ya218.}
Yang, X.; Shen, X.; Zhao, P.; Law, C.K. (2021).
Statistical Analysis on Rate Parameters of the H-2-O-2 Reaction System
{\em J. Phys. Chem. A 125,} 10223 - 10234.

\noindent {\bf Yu98.}
Yu, Y-X., Li, S-M., Xu, Z-F., Li, Z-S., Sun, C-C. (1998). An ab initio study on the reaction NH2 + CH4 $\rightarrow$ NH3 + CH3. {\em Chem. Phys. Lett. 296,} 131-136.

\noindent {\bf Za89.}
Zabarnick, S.; Lin, M.C. (1989).
Kinetics of CN(X$^2\Sigma^+$) radical reactions with HCN, BrCN and CH3CN
{\em Chem. Phys. 134,} 185 - 191.

\noindent {\bf Za09.}
Zahnle, K., Marley, M.M., Lodders, K., and Fortney, J.J. (2009). Atmospheric sulfur chemistry on hot Jupiters. {\em Astrophys. J. Lett. 701,} L20-L24.

\noindent {\bf Za17}
Zador, J.; Fellows, M.D.; Miller, J.A. (2017)
Initiation Reactions in Acetylene Pyrolysis.
{\em  J. Phys. Chem. A 121,} 4203 - 4217.

\noindent {\bf Zh03.}
Zhu, R.S.; Lin, M.C. (2003).
 Ab initio studies of ClOx reactions. VIII. Isomerization and decomposition of ClO2 radicals and related bimolecular processes
{\em J. Chem. Phys. 119,}  2075 - 2082.

\noindent {\bf Zh05.}
Zhu, R.S., Park, J., Lin, M.C. (2005). Ab initio kinetic study on the low-energy paths of the HO+C2H4 reaction.  {\em Chem. Phys. Lett. 408,} 25-30.

\noindent {\bf Zu97.}
Zu, Z-F., Fang, D-C., Fu, X-Y. (1997). Ab initio study on the reaction 2NH(X$^3\Sigma^-$)  $\rightarrow$ NH2(X$^2$B$_1$) + N($^4$S).  {\em Chem. Phys. Lett. 275,} 386-391.

\end{document}

Sulfanes (H$_2$S$_n$, hydropolysulfides) are liquids at room temperature (Steudel 2003). The smaller ones will be present and possibly rather abundant at hot Jupiter conditions.  Sulfanes absorb VUV between 260 nm and 330 nm (Steudel 2003), wavelengths that are absorbed by the more abundant S$_2$ and HS.  Two groups have used theoretical tools to study the reactions of HS$_2$  and H$_2$S$_2$ in combustion (Sendt et al 2002, Sendt and Haynes 2005, Cerru et al 2006, Zhou et al 2008).  Reaction rates have been estimated by fitting complex systems.  The resulting set of reaction rates are rather puzzling, but fast enough that thermochemical equilibrium is important.  Unfortunately, thermochemical data are incomplete.  For H$_2$S$_2$(g), $H_{298}^{\circ}=16$ kJ/mol (Feher and Winkhaus 1957).  From computer modeling, Denis (2006) estimates that $H_{298}^{\circ}=105$ kJ/mol for HS$_2$.   

 \multicolumn{6}{l}{\bf Na, NaH, NaOH}\\
 \refstepcounter{reaction}R\arabic{reaction} & NaH  + H    &$\!\!\!\rightarrow$ &   Na   +   H$_2$   & $ 1.2\!\times\! 10^{-10} \left(T/298 \right)^{0.69} e^{-2350/T}$  & Ma66\\  
 \refstepcounter{reaction}R\arabic{reaction} & Na  + HCO    &$\!\!\!\rightarrow$ &   NaH   +   CO   & $ 2.7\!\times\! 10^{-14} e^{-3860/T}$  & Ma67 \\  
 \refstepcounter{reaction}R\arabic{reaction} & NaO  + O    &$\!\!\!\rightarrow$ &   Na   +   O$_2$   & $ 3.7\!\times\! 10^{-10} $  & De97\\  
 \refstepcounter{reaction}R\arabic{reaction} & Na  + N$_2$O    &$\!\!\!\rightarrow$ &   NaO   +   N$_2$   & $ 2.8\!\times\! 10^{-11} e^{-1600/T}$  & De97\\  
 \refstepcounter{reaction}R\arabic{reaction} & NaO  + H$_2$    &$\!\!\!\rightarrow$ &   NaOH  +   H   & $ 2.6\!\times\! 10^{-11} $  & De97 \\  
 \refstepcounter{reaction}\label{RNaOH}R\arabic{reaction} & Na  +    OH + M &$\!\!\!\rightarrow$ &      NaOH + M & $ 3.9\!\times\! 10^{-30} \left(T/298 \right)^{-1.6}  $   &  Hu85 \\     
          & Na  +    OH  &$\!\!\!\rightarrow$ &   NaOH  & $ 1.0\!\times\! 10^{-11} $    &  assumed\\  
 \refstepcounter{reaction}R\arabic{reaction} & NaH  + OH    &$\!\!\!\rightarrow$ &   NaOH  +   H   & $ 1.0\!\times\! 10^{-10} $  & \\  
 \refstepcounter{reaction}R\arabic{reaction} & NaO  + H$_2$O    &$\!\!\!\rightarrow$ &   NaOH  +   OH   & $ 4.4\!\times\! 10^{-10} e^{-505/T} $  & Co99 \\  
 \refstepcounter{reaction}\label{RNaO}R\arabic{reaction} & NaO  +    H + M &$\!\!\!\rightarrow$ &      NaOH + M & $ 3.9\!\times\! 10^{-30} \left(T/298 \right)^{-1.6}  $   &  \\     
          & NaO  +    H  &$\!\!\!\rightarrow$ &   NaOH  & $ 1.0\!\times\! 10^{-11} $    &  \\  
 \refstepcounter{reaction}R\arabic{reaction} & NaOH  + H    &$\!\!\!\rightarrow$ &   Na  +   H$_2$O   & $ 1.8\!\times\! 10^{-11} e^{-990/T} $  & Je82\\  

 \multicolumn{6}{l}{\bf NaCl}\\
 \refstepcounter{reaction}\label{RNaCl}R\arabic{reaction} & Na  +    Cl + M &$\!\!\!\rightarrow$ &      NaCl + M & $ 3.9\!\times\! 10^{-30} \left(T/298 \right)^{-1.6}  $   &  CHECK \\     
          & Na  +    Cl  &$\!\!\!\rightarrow$ &   NaCl  & $ 1.0\!\times\! 10^{-11} $    &  \\  
 \refstepcounter{reaction}\label{RNaClb}R\arabic{reaction} & Na  + HCl   &$\!\!\!\rightarrow$ &   NaCl  +   H   & $ 4.0\!\times\! 10^{-10} e^{-4090/T} $  & Hu86 \\  
 \refstepcounter{reaction}R\arabic{reaction} & Na  + Cl$_2$   &$\!\!\!\rightarrow$ &   NaCl  +   Cl   & $ 7.3\!\times\! 10^{-10} $  & De97\\  
 \refstepcounter{reaction}R\arabic{reaction} & NaH  + Cl   &$\!\!\!\rightarrow$ &   NaCl  +   H   & $ 1.0\!\times\! 10^{-10}  $  & \\  
% \refstepcounter{reaction}R\arabic{reaction} & Na  + ClCO  &$\!\!\!\rightarrow$ &   NaCl  +   CO   & $ 1.0\!\times\! 10^{-10}  $  & \\  
 \refstepcounter{reaction}R\arabic{reaction} & NaO  + HCl   &$\!\!\!\rightarrow$ &   NaCl  +   OH  & $ 2.8\!\times\! 10^{-10} $  & Si84 \\  
 \refstepcounter{reaction}R\arabic{reaction} & NaOH  + HCl   &$\!\!\!\rightarrow$ &   NaCl  +   H$_2$O   & $ 2.8\!\times\! 10^{-10} $  & De97 \\  

 \multicolumn{6}{l}{\bf NaCN}\\
 \refstepcounter{reaction}\label{RNaCN}R\arabic{reaction} & Na  +    CN + M &$\!\!\!\rightarrow$ &      NaCN + M & $ 3.9\!\times\! 10^{-30} \left(T/298 \right)^{-1.6}  $   &  \\     
          & Na  +   CN  &$\!\!\!\rightarrow$ &   NaCN  & $ 1.0\!\times\! 10^{-11} $    &  \\  
 \refstepcounter{reaction}\label{RNaCNb}R\arabic{reaction} & Na  + HCN   &$\!\!\!\rightarrow$ &   NaCl  +   H   & $ 4.0\!\times\! 10^{-10} e^{-9000/T} $  & \\  
 \refstepcounter{reaction}R\arabic{reaction} & NaOH  + HCN  &$\!\!\!\rightarrow$ &   NaCN  +   H$_2$O   & $ 1.0\!\times\! 10^{-10}  $  & \\  
 \refstepcounter{reaction}R\arabic{reaction} & NaCN  + HCl  &$\!\!\!\rightarrow$ &   NaCl  +   HCN   & $ 1.0\!\times\! 10^{-10}  $  & \\  


\noindent {\bf R\ref{RNaOH}.}  Ya95 gives the vapor pressure (in torr) as $\log_{10}{\left(P\right)} =-48.23-1934/T+17\log_{10}{\left(T\right)}+3.74\times 10^{-4}T-8.75\times 10^{-7}T^2$  for $596<T<1830$ K. 

\noindent {\bf R\ref{RNaO}.}  Assumes the same rate as Na + OH. 

\noindent {\bf R\ref{RNaCl}.}  Assumes the same rate as Na + OH.  Ya95 gives the vapor pressure (in torr) as $\log_{10}{\left(P\right)} = 22.43-11358/T-4.2\log_{10}{\left(T\right)}+2.96\times 10^{-11}T-3.95\times 10^{-12}T^2$ for $1074<T<1738$ K. 

\noindent {\bf R\ref{RNaCN}.}  We include NaCN because we have thermodynamic data and HCN can be abundant out of equilibrium. Ya95 give the vapor pressure (in torr) as $\log_{10}{\left(P\right)} =-2.23-8202/T+3.4\log_{10}{\left(T\right)}-2.43\times 10^{-3}T+3.93\times 10^{-7}T^2$ for $867<T<1769$ K.
\noindent {\bf R\ref{RNaCNb}.}  By analogy to R\ref{RNaClb}, with the energy barrier scaled by the different heats of formation.

